\section{Jordan--Chevalley for Locally Finite Endomorphisms}





\subsection{Locally Finite Endomorphisms}


\begin{lemma}
  \label{characterization of locally finite endomorphisms}
  For an endomorphism~$g \colon V \to V$ of a~{\kvs}~$V$ the following conditions are equivalent:
  \begin{enumerate}
    \item
      \label{every v in a fd subspace}
      Every~$v \in V$ is contained in some {\fd}~\dash{$g$}{invariant} subspace.
    \item
      \label{union of fd subspaces}
      $V$ is the union of some of its {\fd}~\dash{$g$}{invariant} subspaces.
    \item
      \label{sum of fd subspaces}
      $V$ is the sum of some of its {\fd}~\dash{$g$}{invariant} subspaces.
    \item
      \label{union of all fd subspaces}
      $V$ is the union of all of its {\fd}~\dash{$g$}{invariant} subspaces.
    \item
      \label{sum of all fd subspaces}
      $V$ is the sum of all of its {\fd}~\dash{$g$}{invariant} subspaces.
  \end{enumerate}
\end{lemma}


\begin{proof}
  \leavevmode
  \begin{description}
    \item[\ref*{every v in a fd subspace}~$\iff$~\ref*{union of fd subspaces}:]
      This is just a reformulation.
    \item[\ref*{union of fd subspaces}~$\iff$~\ref*{union of all fd subspaces}, and \ref*{sum of fd subspaces}~$\iff$~\ref*{sum of all fd subspaces}:]
      These is clear.
    \item[\ref*{sum of all fd subspaces}~$\iff$~\ref*{union of all fd subspaces}]
      The set~$S$ of all~{\fd}~\dash{$g$}{invariant} subspaces of~$V$ is directed with respect to the inclusion~$\subseteq$, meaning that for any two~$W_1, W_2 \in S$ there exists some~$W \in S$ with~$W_1, W_2 \subseteq W$ (for example~$W = W_1 + W_2$).
      It follows that~$\sum_{W \in S} W = \bigcup_{W \in S} W$.
    \qedhere
  \end{description}
\end{proof}


\begin{definition}
  An endomorphism~$g \colon V \to V$ of a~{\kvs}~$V$ which satisfies the equivalent conditions from \cref{characterization of locally finite endomorphisms} is \emph{locally finite}\index{locally!finite}.
\end{definition}


% \begin{lemma}
%   \label{ss iff diagbar for locally finite}
%   An endomorphism~$g \colon V \to V$ of a~{\kvs}~$V$ is locally finite and semisimple if and only if it is diagonalizable.
% \end{lemma}
% 
% 
% \begin{proof}
%   If~$g$ is diagonalizable then~$V$ is the sum of \dash{one}{dimensional}~\dash{$g$}{invariant} subspaces (each of which is spanned by an eigenvector for~$g$) which shows that~$g$ is both locally finite and semisimple.
%   
%   If~$g$ is locally finite then there exists a cover~$V = \bigcup_{i \in I} W_i$ by~\dash{$g$}{invariant} {\fd} subspaces~$W_i \subseteq V$.
%   If~$g$ is also semisimple then it follows from \cref{properties of ss endomorphisms} that the restriction~$\restrict{g}{W_i}$ is semisimple for every~$i \in I$, and thus diagonalizable by \cref{ss iff diagbar for fd}.
%   This then shows that every~$W_i$, and therefore also~$V$, is spanned by eigenvectors for~$g$.
% \end{proof}
% 
% 
% \begin{corollary}
%   If~$g_1, g_2 \colon V \to V$ are two locally finite and semisimple endomorphisms which commute with each other then the sum~$g_1 + g_2$ and the composition~$g_1 \circ g_2$ are again locally finite and semisimple.
% \end{corollary}
% 
% 
% \begin{proof}
%   This holds because~$g_1$ and~$g_2$ are simultaneously diagonalizable.
% \end{proof}


\begin{example}
  Semisimple endomorphisms are locally finite because for every semisimple endomorphism~$g \colon V \to V$ the vector space~$V$ is a sum \dash{one}{dimensional}~\dash{$g$}{invariant} subspaces (namely the ones spanned by eigenvectors).
\end{example}





\subsection{Locally Nilpotent and Locally Unipotent Endomorphisms}


\begin{definition}
  An endomorphism~$g \colon V \to V$ of a~{\kvs}~$V$ is \emph{locally nilpotent}\index{locally!nilpotent} (resp.\ \emph{locally unipotent}\index{locally!unipotent}) if it acts nilpotent (resp.\ unipotent) on every~$v \in V$, i.e.\ if there exists for every~$v \in V$ some power~$n$ such that~$g^n(v) = 0$ (resp.~$(g - \id_V)^n(v) = 0$).
\end{definition}


\begin{remark}
  An endomorphism~$g \colon V \to V$ of a~{\kvs}~$V$ is locally unipotent if and only if~$g - \id_V$ is locally nilpotent.
\end{remark}


\begin{lemma}
  \label{characterization of locally potent}
  For an endomorphism~$g \colon V \to V$ of a {\kvs}~$V$ the following conditions are equivalent:
  \begin{enumerate}
    \item
      \label{locally potent}
      The endomorphism~$g$ is locally nilpotent (resp.\ locally unipotent).
    \item
      \label{locally finite and all fd subspaces}
      The endomorphism~$g$ is locally finite and the restriction~$\restrict{g}{W}$ is nilpotent (resp.\ unipotent) for every {\fd}~\dash{$g$}{invariant} subspace~$W \subseteq V$.
    \item
      \label{a covering by fd subspaces}
      There exists a covering~$V = \bigcup_{i \in I} W_i$ by~{\fd}~\dash{$g$}{invariant} subspaces~$W_i \subseteq V$ such that the restriction~$\restrict{g}{W_i}$ is nilpotent (resp.\ unipotent) for every~$i \in I$.
  \end{enumerate}
\end{lemma}


\begin{proof}
  \label{characterization of locally potent endomorphisms}
  We first consider the (locally) nilpotent case:
  \begin{description}
    \item[\ref*{locally potent}~$\implies$~\ref*{locally finite and all fd subspaces}]
      Suppose that~$g$ is locally nilpotent.
      Then for every~$v \in V$ there exists some~$n \geq 1$ with~$g^n(v) = 0$ and it follows that~$\gen{ v, g(v), \dotsc, g^{n-1}(v) }_k$ is a {\fd}~\dash{$g$}{invariant} subspace of~$V$ which contains~$v$.
      This shows that~$g$ is locally finite.
      
      If~$W \subseteq V$ is {\fd}~\dash{$g$}{invariant} subspace with finite generating set~$w_1, \dotsc, w_r \in W$ then there exist~$n_1, \dotsc, n_r \geq 1$ with~$g^{n_i}(w_i) = 0$ for all~$i$. It follows for~$n \defined \max(n_1, \dotsc, n_r)$ that~$(\restrict{g}{W})^n = 0$, which shows that~$\restrict{g}{W}$ is nilpotent.
    \item[\ref*{locally finite and all fd subspaces}~$\implies$~\ref*{a covering by fd subspaces}]
      One can choose the covering by all {\fd}~\dash{$g$}{invariant} subspaces.
    \item[\ref*{a covering by fd subspaces}~$\implies$~\ref*{locally potent}]
      There exists for every~$v \in V$ some~$i \in I$ with~$v \in W_i$, and it follows from~$\restrict{g}{W_i}$ being nilpotent that~$g$ acts nilpotent on~$v$.
  \end{description}
  The (locally) unipotent case follows from the (locally) unipotent one by using that~$g$ is locally unipotent if and only if~$g - \id_V$ is locally nilpotent, and that~$g$ and~$g - \id_V$ have the same invariant subspaces.
\end{proof}


\begin{corollary}
  Let~$g \colon V \to V$ is a locally nilpotent (resp.\ locally unipotent) endomorphism of a {\fd}~{\kvs}~$V$ then~$g$ is nilpotent (resp.\ unipotent).
  \qed
\end{corollary}


\begin{lemma}
  \label{restriction of ss and locally potent endomorphisms }
  Let~$g \colon V \to V$ be an endomorphism of a~{\kvs}~$V$ and let~$W \subseteq V$ be a~\dash{$g$}{invariant} subspace.
  If~$g$ is locally finite, locally nilpotent or locally unipotent then the restriction~$\restrict{g}{W}$ has the same property.
\end{lemma}


\begin{proof}
  If~$g$ is locally finite then there exists for every~$w \in W$ a {\fd}~\dash{$g$}{invariant} subspace~$W' \subseteq V$ with~$w \in W'$.
  Then~$W \cap W'$ is a {\fd}~\dash{$(\restrict{g}{W})$}{invariant} subspace of~$W$ which contains~$w$.
  f 
  If~$g$ is locally nilpotent (resp.\ locally unipotent) then~$g$ acts nilpotent (resp.\ unipotent) on every~$v \in V$ and therefore also on every~$v \in W$.
\end{proof}


\begin{lemma}
  \label{combination of locally potent endomorphisms}
  Let~$g_1, g_2 \colon V \to V$ be two endomorphisms of a~{\kvs}~$V$ which commute with each other.
  \begin{enumerate}
    \item
      \label{sum of locally nilpotent}
      If both~$g_1$ and~$g_2$ are locally nilpotent then the sum~$g_1 + g_2$ is again locally nilpotent.
    \item
      \label{composition of locally nilpotent}
      If~$g_1$ or~$g_2$ is locally nilpotent then the composition~$g_1 \circ g_2$ is again locally nilpotent.
    \item
      \label{composition of locally uinpotent}
      If both~$g_1$ and~$g_2$ are locally unipotent then the composition~$g_1 \circ g_2$ is again locally unipotent.
  \end{enumerate}
\end{lemma}


\begin{proof}
  \leavevmode
  \begin{enumerate}
    \item
      There exist for every~$v \in V$ powers~$n$ and~$m$ with~$g_1^n(v) = 0$ and~$g_2^m(v) = 0$.
      It follows that
      \[
          (g_1 + g_2)^{n+m}(v)
        = \sum_{\ell=0}^{n+m} \left( g_1^\ell \circ g_2^{n+m-\ell} \right)(v)
        =   \sum_{\ell=0}^n g_1^\ell \bigl( \underbrace{g_2^{n+m-\ell}(v)}_{=0} \bigr)
          + \sum_{\ell=n+1}^{n+m} g_2^{n+m-\ell} \bigl( \underbrace{g_1^\ell(v)}_{=0} \bigr)
        = 0
      \]
    \item
      We may assume that~$g_2$ is locally nilpotent.
      For every~$v \in V$ there then exists some power~$n$ with~$g_2^n(v) = 0$, for which it then follows that
      \[
          (g_1 \circ g_2)^n(v)
        = g_1^n\bigl( \underbrace{g_2^n(v)}_{=0} \bigr)
        = 0 \,.
      \]
    \item
      It holds that
      \[
          g_1 \circ g_2 - \id_V
        = (g_1 - \id_V) \circ g_2 + (g_2 - \id_V)
      \]
      The terms~$g_1 - \id_V$ and~$g_2 - \id_V$ are locally nilpotent, and it follows from part~\ref*{composition of locally nilpotent} that the composition~$(g_1 - \id_V) \circ g_2$ is again locally nilpotent.
      It further follows from part~\ref*{sum of locally nilpotent} that~$(g_1 - \id_V) \circ g_2 + (g_2 - \id_V)$ is locally nilpotent.
      This shows that~$g_1 \circ g_2 - \id_V$ is locally nilpotent, and therefore that~$g_1 \circ g_2$ is locally unipotent.
    \qedhere
  \end{enumerate}
\end{proof}


 
% \subsection{Locally Nilpotent Endomorphisms}
% 
% 
% \begin{lemma}
%   \label{characterization of locally nilpotent endomorphisms}
%   For an endomorphism~$g \colon V \to V$ of a~{\kvs}~$V$ the following conditions are equivalent:
%   \begin{enumerate}
%     \item
%       The endomorphism~$g$ acts nilpotent on every~$v \in V$, i.e.\ there exists for every~$v \in V$ some~$n$ with~$g^n(v) = 0$.
%     \item
%       Every~$v \in V$ is contained in a {\fd}~\dash{$g$}{invariant} subspace~$W \subseteq V$ such that~$\restrict{g}{W}$ is nilpotent, i.e.\ the vector space~$V$ is a union~$V = \bigcup_{i \in I} W_i$ of {\fd}~\dash{$g$}{invariant} subspaces~$W_i \subseteq V$ such that~$\restrict{g}{W_i}$ is nilpotent for every~$i \in I$.
%     \item
%       The vector space~$V$ is a sum~$V = \sum_{i \in I} W_i$ of {\fd}~\dash{$g$}{invariant} subspaces~$W_i \subseteq V$ such that~$\restrict{g}{W_i}$ is nilpotent for every~$i \in I$.
%   \end{enumerate}
% \end{lemma}
% 
% 
% \begin{definition}
%   An endomorphim~$g \colon V \to V$ of a~{\kvs} which satisfies the equivalent conditions from \cref{characterization of locally nilpotent endomorphisms} is \emph{locally nilpotent}\index{locally!nilpotent}.
% \end{definition}
% 
% 
% \begin{lemma}
%   Let~$g \colon V \to V$ be an endomorphism of a~{\kvs}~$V$.
%   \begin{enumerate}
%     \item
%       If~$g$ is locally nilpotent then~$\restrict{g}{W}$ is nilpotent for every {\fd}~\dash{$g$}{invariant} subspace~$W \subseteq V$.
%     \item
%       If~$g$ is locally finite and the restriction~$\restrict{g}{W}$ is nilpotent for every {\fd}~\dash{$g$}{invariant} subspace~$W \subseteq V$ then~$g$ is locally nilpotent.
%   \end{enumerate}
% \end{lemma}
% 
% 
% 
% \subsection{Locally Unipotent Endomorphsims}
% 
% 
% \begin{lemma}
%   \label{characterization of locally unipotent endomorphisms}
%   For an endomorphism~$g \colon V \to V$ of a~{\kvs}~$V$ the following conditions are equivalent:
%   \begin{enumerate}
%     \item
%       The endomorphism~$g - \id_V$ is locally nilpotent.
%     \item
%       The endomorphism~$g$ acts unipotent on every~$v \in V$, i.e.\ there exists for every~$v \in V$ some~$n$ with~$(g - \id_V)^n(v) = 0$.
%     \item
%       Every~$v \in V$ is contained in a {\fd}~\dash{$g$}{invariant} subspace~$W \subseteq V$ such that~$\restrict{g}{W}$ is unipotent, i.e.\ the vector space~$V$ is a union~$V = \bigcup_{i \in I} W_i$ of {\fd}~\dash{$g$}{invariant} subspaces~$W_i \subseteq V$ such that~$\restrict{g}{W_i}$ is unipotent for every~$i \in I$.
%     \item
%       The vector space~$V$ is a sum~$V = \sum_{i \in I} W_i$ of {\fd}~\dash{$g$}{invariant} subspaces~$W_i \subseteq V$ such that~$\restrict{g}{W_i}$ is unipotent for every~$i \in I$.
%   \end{enumerate}
% \end{lemma}
% 
% 
% \begin{definition}
%   An endomorphim~$g \colon V \to V$ of a~{\kvs} which satisfies the equivalent conditions from \cref{characterization of locally unipotent endomorphisms} is \emph{locally unipotent}\index{locally!unipotent}.
% \end{definition}
% 
% 
% \begin{lemma}
%   Let~$g \colon V \to V$ be an endomorphism of a~{\kvs}~$V$.
%   \begin{enumerate}
%     \item
%       If~$g$ is locally unipotent then~$\restrict{g}{W}$ is unipotent for every {\fd}~\dash{$g$}{invariant} subspace~$W \subseteq V$.
%     \item
%       If~$g$ is locally finite and the restriction~$\restrict{g}{W}$ is unipotent for every {\fd}~\dash{$g$}{invariant} subspace~$W \subseteq V$ then~$g$ is locally unipotent.
%   \end{enumerate}
% \end{lemma}





\subsection{Additive Jordan--Chevalley Decomposition}


\begin{lemma}
  \label{ss and locally np is zero}
  The only endomorphism~$g \colon V \to V$ which is both semisimple and locally nilpotent is the zero endomorphism.
\end{lemma}


\begin{proof}
  It follows from~$g$ being locally nilpotent that~$0$ is the only (possible) eigenvalue of~$g$.
\end{proof}


\begin{proposition}[Additive {\JCD}]
  \label{local jcd}
  Let~$g \colon V \to V$ be a locally finite endomorphism of a~{\kvs}~$V$.
  \begin{enumerate}
    \item
      \label{the local jcd itself}
      There exists unique endomorphisms~$g_s, g_n \colon V \to V$ with~$g = g_s + g_n$ such that~$g_s$ is semisimple,~$g_n$ is locally nilpotent and~$g_s$ and~$g_n$ commute with each other.
    \item
      \label{commuting via local jcd}
      An endomorphism~$h \colon V \to V$ commutes with~$g$ if and only if it already commutes with both~$g_s$ and~$g_n$.
    \item
      \label{local jcd via cover}
      If~$V = \bigcup_{i \in I} W_i$ is any covering by {\fd}~\dash{$g$}{invariant} subspaces then the endomorphisms~$g_s$ and~$g_n$ are uniquely determined by the property that every~$W_i$ is also~\dash{$g_s$}{invariant} and~\dash{$g_n$}{invariant} and that~$\restrict{g}{W_i} = \restrict{g_s}{W_i} + \restrict{g_n}{W_i}$ is the {\JCD} of~$\restrict{g}{W_i}$ for every~$i \in I$.
  \end{enumerate}
\end{proposition}


\begin{proof}
  \label{local jcd proof}
  It holds for the set
  \[
              \mc{F}
    \defined  \left\{
                W \subseteq V
              \suchthat*
                \text{$W$ is a {\fd}~\dash{$g$}{invariant} subspace of~$V$}\ 
              \right\}
  \]
  that~$V = \bigcup_{W \in \mathcal{F}} W$ because~$g$ is locally finite.
  For every~$W \in \mc{F}$ let~$\restrict{g}{W} = g_{s,W} + g_{n,W}$ be the {\JCD} of~$\restrict{g}{W}$.
  It holds for all~$W_1, W_2 \in \mc{F}$ that
  \[
      \restrict*{g_{s,W_1}}{W_2}
    = g_{s, W_1 \cap W_2}
    = \restrict*{g_{s,W_2}}{W_1}
  \]
  by \cref{properties of jcd}.
  It follows that there exists a unique map~$g_s \colon V \to V$ with~$\restrict{g_s}{W} = g_{s,W}$ for every~$W \in \mc{F}$.
  This map is~\dash{$k$}{linear} since there exists for all~$v_1, v_2 \in V$ some~$W \in \mc{F}$ with~$v_1, v_2 \in W$, with the restriction~$\restrict{g_s}{W} = g_{s,W}$ being~\dash{$k$}{linear}.
  It follows from \cref{properties of ss endomorphisms} that the endomorphism~$g_s$ is semisimple
  
  It follows similarly that there exists a unique map~$g_n \colon V \to V$ with~$\restrict{g_n}{W} = g_{n,W}$ for all~$W \in \mc{F}$, which is then~\dash{$k$}{linear} and by \cref{characterization of locally potent} locally nilpotent.
  It holds that~$g = g_s + g_n$ and the endomorphisms~$g_s$ and~$g_n$ commute with each other, because this holds on every subspace~$W \in \mc{F}$ and~$V$ is covered by them.
  This shows the existence for part~\ref*{the local jcd itself}.
  
  To show part~\ref*{commuting via local jcd} suppose that~$h$ commutes with~$g$.
  We show that the claimed equalities hold at every point~$v \in V$.
  There exists a {\fd}~\dash{$g$}{invariant} subspace~$W \subseteq V$ with~$v \in W$ because~$g$ is locally finite, and the subspace~$h(W)$ is again~\dash{$g$}{invariant} because~$g$ and~$h$ commmute.
  The subspaces~$W$,~$h(W)$ together with the actions of~$g$ and~$h$ on them fit into the following commutative diagram:
  \[
    \begin{tikzcd}[sep = large]
        W
        \arrow{r}[above]{\restrict{g}{W}}
        \arrow{d}[left]{\restrict{h}{W}}
      & W
        \arrow{d}[right]{\restrict{h}{W}}
      \\
        h(W)
        \arrow{r}[above]{\restrict{g}{h(W)}}
      & h(W)
    \end{tikzcd}
  \]
  It follows from \cref{properties of jcd} that the restriction~$\restrict{h}{W}$ satisfies the equations
  \[
      \restrict*{h}{W} \circ \left( \restrict*{g}{W} \right)_s
    = \left( \restrict*{g}{h(W)} \right)_s \circ \restrict{h}{W}
    \quad\text{and}\quad
      \restrict*{h}{W} \circ \left( \restrict*{g}{W} \right)_n
    = \left( \restrict*{g}{h(W)} \right)_n \circ \restrict{h}{W} \,.
  \]
  By evaluating these equations at~$v \in W$ and using the above construction of~$g_s$ and~$g_n$ it follows that
  \[
      h(g_s(v))
    = g_s(h(v))
    \quad\text{and}\quad
      h(g_n(v))
    = g_n(h(v)) \,,
  \]
  as desired.

  To show the uniqueness for part~\ref*{the local jcd itself} let~$g = g'_s + g'_n$ be another decomposition with~$g'_s$ semisimple and~$g'_n$ locally nilpotent such that~$g'_s$ and~$g'_n$ commute.
  It then follows from part~\ref{commuting via jcd} that~$g'_s$ and~$g'_n$ commute with~$g_s$ and~$g_n$ because they commute with~$g$.
  It then follows from
  \begin{gather*}
      g_s + g_n 
    = g
    = g'_s + g'_n
  \shortintertext{that}
      g_s - g'_s
    = g'_n - g_n \,.
  \end{gather*}
  The endomorphism~$g_s - g'_s$ is again semisimple by \cref{combination of ss endomorphisms} and the endomorphism~$g'_n - g_n$ is again locally nilpotent by \cref{combination of locally potent endomorphisms}.
  It follows from~\cref{ss and locally np is zero} that
  \[
    g_s - g'_n = g'_n - g_n = 0
  \]
  and therefore that~$g_s = g'_s$ and~$g_n = g'_n$.
  
  To show part~\ref*{local jcd via cover} let~$V = \bigcup_{i \in I} W_i$ be a cover of~$V$ by {\fd}~\dash{$g$}{invariant} subspaces.
  Then for every~$i \in I$ the restriction~$\restrict{g_s}{W_i} = g_{s,W_i}$ is semisimple and the restriction~$\restrict{g_n}{W_i} = g_{n,W_i}$ is nilpotent.
  It follows for every~$i \in I$ from~$g = g_s + g_n$ that~$\restrict{g}{W_i} = \restrict{g_s}{W_i} + \restrict{g_n}{W_i}$, and it follows that~$\restrict{g_s}{W_i}$ and~$\restrict{g_n}{W_i}$ commute because~$g_s$ and~$g_n$ commute.
  Together this shows that~$\restrict{g}{W_i} = \restrict{g_s}{W_i} + \restrict{g_n}{W_i}$ is the {\JCD} of~$\restrict{g}{W_i}$ for every~$i \in I$.
  The maps~$g_s$ and~$g_n$ are uniquely determined by this because their actions on the~$W_i$ are uniquely determined and these subspaces cover~$V$.
\end{proof}


\begin{definition}
  \label{def local jcd}
  If~$g \colon V \to V$ is a locally finite endomorphism of a~{\kvs}~$V$ then the decomposition~$g = g_s + g_n$ from \cref{local jcd} is the \emph{\textup(additive\textup) Jordan--Chevalley decomposition}\index{Jordan--Chevalley decomposition!for locally finite endomorphisms!additive} of~$g$.
  The summand~$g_s$ is the \emph{semisimple part}\index{semisimple!part of!a locally finite endomorphism} of~$g$ and the summand~$g_n$ is the \emph{nilpotent part}\index{nilpotent!part of!a locally finite endomorphism} of~$g$.
\end{definition}


\begin{lemma}
  \label{properties of local jcd}
  Let~$g \colon V \to V$ be a locally finite endomorphism of a~{\kvs}~$V$ and let~$W \subseteq V$ be a~\dash{$g$}{invariant} subspace.
  \begin{enumerate}
    \item
      The subspace~$W$ is also~\dash{$g_s$}{invariant} and~\dash{$g_n$}{invariant}
    \item
      \label{restriction of local jcd}
      The {\JCD} of the restriction~$\restrict{g}{W}$ is given by~$\restrict{g}{W} = \restrict{g_s}{W} + \restrict{g_n}{W}$.
    \item
      \label{induced local jcd}
      The {\JCD} of the induced endomorphism~$\induced{g} \colon V/W \to V/W$ is given by~$\induced{g} = \induced{g_s} + \induced{g_n}$.
    \item
      Let~$g' \colon V' \to V'$ be an endomorphism of another~{\kvs}~$V'$ and let~$f \colon V \to V'$ be a~\dash{$k$}{linear} map with~$f \circ g = g' \circ f$, i.e.\ such that the diagram
      \[
        \begin{tikzcd}
            V
            \arrow{r}[above]{g}
            \arrow{d}[left]{f}
          & V
            \arrow{d}[right]{f}
          \\
            V'
            \arrow{r}[above]{g'}
          & V'
        \end{tikzcd}
      \]
      commutes.
      Then~$f \circ g_s = g'_s \circ f$ and~$f \circ g_n = g'_n \circ f$, i.e.\ the following diagrams commute:
      \[
        \begin{tikzcd}
            V
            \arrow{r}[above]{g_s}
            \arrow{d}[left]{f}
          & V
            \arrow{d}[right]{f}
          \\
            V'
            \arrow{r}[above]{g'_s}
          & V'
        \end{tikzcd}
        \qquad
        \begin{tikzcd}
            V
            \arrow{r}[above]{g_n}
            \arrow{d}[left]{f}
          & V
            \arrow{d}[right]{f}
          \\
            V'
            \arrow{r}[above]{g'_n}
          & V'
        \end{tikzcd}
      \]
  \end{enumerate}
\end{lemma}


\begin{proof}
  \leavevmode
  \begin{enumerate}
    \item
      Let~$V = \bigcup_{i \in I} V_i$ be a cover by {\fd}~\dash{$g$}{invariant} subspaces, and for every~$i \in I$ let~$W_i \defined V \cap W_i$.
      Then~$W = \bigcup_{i \in I} W_i$ is a cover of~$W$ by {\fd}~\dash{$g$}{invariant} subspaces because
      \[
          W 
        = W \cap V
        = W \cap \bigcup_{i \in I} V_i
        = \bigcup_{i \in I} (W \cap V_i)
        = \bigcup_{i \in I} W_i \,.
      \]
      It then holds for every~$i \in I$ that~$\restrict{g}{W_i} = \restrict{g_s}{W_i} + \restrict{g_n}{W_i}$ is the {\JCD} of~$\restrict{g}{W_i}$.
      It follows in particular that every~$W_i$ is both~\dash{$g_s$}{invariant} and~\dash{$g_n$}{invariant} and therefore that~$W = \bigcup_{i \in I} W_i$ is both~\dash{$g_s$}{invariant} and~\dash{$g_n$}{invariant}.
    \item
      By continuing the above line of thought we find that $\restrict{(\restrict{g_s}{W})}{W_i} = (\restrict{g}{W_i})_s$ for every~$i \in I$.
      It follows from \cref{local jcd} that~$\restrict{g_s}{W} = ( \restrict{g}{W} )_s$ and similarly for the nilpotent part.
    \item
      It follows from~$g = g_s + g_n$ that~$\induced{g} = \induced{g_s} + \induced{g_n}$, the restrictions~$\induced{g_s}$ and~$\induced{g_n}$ commute because~$g_s$ and~$g_n$ commute, the restriction~$\induced{g_s}$ is again semisimple by \cref{properties of ss endomorphisms} and the restriction~$\induced{g_n}$ is again locally nilpotent.
    \item
      If~$f$ is injective then the claim holds by part~\ref*{restriction of local jcd} and if~$f$ is surjective then it holds by part~\ref*{induced local jcd}.
      The claim follows for general~$f$ as in the \hyperref[properties of jcd proof]{proof of \cref{properties of jcd}}.
    \qedhere
  \end{enumerate}
\end{proof}


\begin{lemma}
  If~$g_1, g_2 \colon V \to V$ are two locally finite endomorphisms of a~{\kvs}~$V$ which commute with each other then
  \[
      (g_1 + g_2)_s
    = (g_1)_s + (g_2)_s
    \quad\text{and}\quad
      (g_1 + g_2)_n
    = (g_1)_n + (g_2)_n \,.
  \]
\end{lemma}


\begin{proof}
  It holds that
  \[
      g_1 + g_2
    = (g_1)_s + (g_1)_n + (g_2)_s + (g_2)_n
    = \bigl( (g_1)_s + (g_2)_s \bigr) + \bigl( (g_1)_n + (g_2)_n \bigr) \,.
  \]
  The summands~$(g_1)_s + (g_2)_s$ and~$(g_1)_n + (g_2)_n$ commute with each other because~$g_1$ and~$g_2$ commute, the summand~$(g_1)_s + (g_2)_s$ is semisimple by \cref{combination of ss endomorphisms} and the summand~$(g_1)_n + (g_2)_n$ is locally nilpotent by \cref{combination of locally potent endomorphisms}.
\end{proof}





\subsection{Multiplicative Jordan--Chevalley Decomposition}


\begin{lemma}
  \label{invertibility via fd}
  Let~$g$ be an endomorphism of a~{\kvs}~$V$.
  \begin{enumerate}
    \item
      \label{restriction of invertible}
      If~$g$ is invertible then for every {\fd}~\dash{$g$}{invariant} subspace~$W \subseteq V$ the restriction~$\restrict{g}{W}$ is again invertible.
      It then holds that~$(\restrict{g}{W})^{-1} = \restrict{g^{-1}}{W}$.
    \item
      If $V = \bigcup_{i \in I} W_i$ is a cover by~\dash{$g$}{invariant} subspaces then~$g$ is invertible if and only if the restriction~$\restrict{g}{W_i}$ is invertible for every~$i \in I$.
      It then holds that~$( \restrict{g}{W_i} )^{-1} = \restrict{g^{-1}}{W_i}$ for every~$i \in I$.
  \end{enumerate}
\end{lemma}


\begin{proof}
  \leavevmode
  \begin{enumerate}
    \item
      If~$g$ is an invertible then the restriction~$\restrict{g}{W}$ is injective and thus an isomorphism by the \dash{finite}{dimensionality} of~$W$.
      The equality~$(\restrict{g}{W})^{-1} = \restrict{g^{-1}}{W}$ holds because both~$(\restrict{g}{W})^{-1}$ and~$g^{-1}$ act the same on~$W$.
    \item
      If~$g$ is invertible then every restriction~$\restrict{g}{W_i}$ is invertible by part~\ref*{restriction of invertible}.
      It holds on the other hand that
      \[
          \ker(g)
        = \bigcup_{i \in I} \ker\left( \restrict*{g}{W_i} \right)
        \quad\text{and}\quad
          \im(g)
        = \bigcup_{i \in I} \im\left( \restrict*{g}{W_i} \right) \,,
      \]
      from which it follows that~$g$ is injective if and only if every restriction~$\restrict{g}{W_i}$ is injective, and that~$g$ is surjective if every restriction~$\restrict{g}{W_i}$ is surjective.
    \qedhere
  \end{enumerate}
\end{proof}


\begin{corollary}
  Let~$g \colon V \to V$ be endomorphism of a~{\kvs}~$V$.
  \begin{enumerate}
    \item
      If~$g$ is locally unipotent then~$g$ is invertible.
    \item
      If~$g$ is locally finite then~$g$ is invertible if and only if its semisimple part~$g_s$ is invertible.
  \end{enumerate}
\end{corollary}


\begin{proof}
  Let~$V = \bigcup_{i \in I} W_i$ be a cover of~$V$ by {\fd}~\dash{$g$}{invariant} subspaces.
  \begin{enumerate}
    \item
      The restriction~$\restrict{g}{W_i}$ is unipotent and thus invertible for every~$i \in I$, so the invertibility of~$g$ follows from \cref{invertibility via fd}.
    \item
      Let~$g = g_s + g_n$ be the {\JCD} of~$g$.
      Then the {\JCD} decompositon of~$\restrict{g}{W_i}$ is for every~$i \in I$ given by~$\restrict{g}{W_i} = \restrict{g_s}{W_i} + \restrict{g_n}{W_i}$.
      It follows from \cref{invertibility via fd} and \cref{invertible iff ss part fd} that
      \begin{align*}
              & \text{$g$ is invertible}  \\
        \iff{}& \text{$\restrict{g}{W_i}$ is invertible for every~$i \in I$}  \\
        \iff{}& \text{$\restrict{g_s}{W_i}$ is invertible for every~$i \in I$}  \\
        \iff{}& \text{$g_s$ is invertible},
      \end{align*}
      as claimed.
    \qedhere
  \end{enumerate}
\end{proof}


\begin{proposition}[Multiplicative {\JCD}]
  \label{local mjcd}
  Let~$V$ be a~{\kvs} and let~$g \in \GL(V)$ be locally finite.
  \begin{enumerate}
    \item
      \label{the local mjcd itself}
      There exist unique endomorphisms~$g_s, g_u \colon V \to V$ with~$g = g_s g_u$ such that the factor~$g_s$ is sempisimple, the factor~$g_u$ is locally~unipotent and~$g_s$ and~$g_u$ commute with each other.
    \item
      \label{invertibility of local factors}
      The factors~$g_s$ and~$g_u$ are invertible.
    \item
      \label{global description of local mjcd}
      The factor~$g_s$ is the semisimple part of~$g$, and the factor~$g_u$ is given by~$g_u = \id_V + g_s^{-1} g_n$.
    \item
      \label{commutativity via local mjcd}
      An endomorphism~$h \colon V \to V$ commutes with~$g$ if and only if it already commutes with both~$g_s$ and~$g_u$.
    \item
      \label{local mjcd via cover}
      If~$V = \bigcup_{i \in I} W_i$ is any covering by {\fd}~\dash{$g$}{invariant} subspaces then the factors~$g_s$ and~$g_u$ are uniquely determined by the property that every~$W_i$ is also~\dash{$g_s$}{invariant} and~\dash{$g_u$}{invariant} and that~$\restrict{g}{W_i} = \restrict{g_s}{W_i} \, \restrict{g_u}{W_i}$ is the multiplicative {\JCD} of~$\restrict{g}{W_i}$ for every~$i \in I$.
  \end{enumerate}
\end{proposition}


\begin{proof}
  Part~\ref*{the local mjcd itself}, part~\ref*{invertibility of local factors} and part~\ref*{global description of local mjcd} can be shown in same way as in the \hyperref[mjcd]{{\fd} case} by using that~$g_s$ in invertible.
  Part~\ref*{commutativity via local mjcd} follows from part~\ref*{global description of local mjcd}.
  
  To show part~\ref*{local mjcd via cover} let~$V = \bigcup_{i \in I} W_i$ be a cover by {\fd}~\dash{$g$}{invariant} subspaces.
  Then it holds for every~$i \in I$ that the additive {\JCD} of~$\restrict{g}{W_i}$ is given by~$\restrict{g}{W_i} = \restrict{g_s}{W_i} + \restrict{g_n}{W_i}$.
  It follows that the unipotent part of~$\restrict{g}{W_i}$ is given by
  \[
      \left( \restrict*{g}{W_i} \right)_u
    =   \id_{W_i}
      + \left( \restrict*{g}{W_i} \right)_s^{-1}
        \left( \restrict*{g}{W_i} \right)_n
    = \restrict*{ \left( \id_V + g^{-1} g_n \right) }{W_i}
    = \restrict*{ g_u }{W_i}
  \]
  as desired.
  That~$g$ is uniquely determined by its restrictions~$\restrict{g}{W_i}$ holds because the~$W_i$ cover~$V$.
\end{proof}


% \begin{proof}
%   We proceed as in the \hyperref[local jcd proof]{proof of the additive Jordan--Chevalley decomposition}:
%   
%   It holds for the set
%   \[
%               \mc{F}
%     \defined  \left\{
%                 W \subseteq V
%               \suchthat*
%                 \text{$W$ is a {\fd}~\dash{$g$}{invariant} subspace of~$V$}\ 
%               \right\}
%   \]
%   that~$V = \bigcup_{W \in \mc{F}} W$ because~$g$ is locally finite.
%   It follows from \cref{invertibility via fd} that the restriction~$\restrict{g}{W}$ is invertible for every~$W \in \mc{F}$.
%   For every~$W \in \mc{F}$ let~$\restrict{g}{W} = g_{s,W} g_{u,W}$ be the multiplicative {\JCD} of~$\restrict{g}{W}$.
%   It follows from \cref{properties of mjcd} for all~$W_1, W_2 \in \mc{F}$ that
%   \[
%       \restrict*{g_{s,W_1}}{W_2}
%     = g_{s, W_1 \cap W_2}
%     = \restrict*{g_{s,W_2}}{W_1} \,.
%   \]
%   It follows that there exists a unique map~$g_s \colon V \to V$ with~$\restrict{g_s}{W} = g_{s,W}$ for every~$W \in \mc{F}$.
%   The map~$g_s$ is~\dash{$k$}{linear} since there exists for all~$v_1, v_2 \in V$ some~$W \in \mc{F}$ with~$v_1, v_2 \in W$, for which it then follows that~$\restrict{g}{W}$ is~\dash{$k$}{linear}.
%   The endomorphismp~$g_s$ is invertible by \cref{invertibility via fd} and semisimple by \cref{properties of ss endomorphisms}.
%   
%   It follows similarly that there exists a unique map~$g_u \colon V \to V$ with~$\restrict{g_u}{W} = g_{u,W}$ for every~$W \in \mc{F}$, which is~$g_u$ is~\dash{$k$}{linear}, and it follows from \cref{invertibility via fd} that~$g_u$ is invertible and from \cref{characterization of locally potent endomorphisms} that~$g_u$ is locally unipotent.
%   It holds that~$g = g_s g_u$ and that~$g_s$ and~$g_u$ commute with each other because this holds on every subspace~$W \in \mc{F}$ and~$V$ is covered by them.
%   This shows the existence for part~\ref*{the local mjcd itself}.
%   
%   To show part~\ref*{local mjcd via cover} let~$V = \bigcup_{i \in I} W_i$ be a cover by {\fd}~\dash{$g$}{invariant} subspaces.
%   It then follows for every~$i \in I$ that~$\restrict{g_s}{W_i} = g_{s,W_i}$ is semisimple and~$\restrict{g_u}{W_i} = g_{u,W_i}$ is unipotent.
% \end{proof}


\begin{definition}
  If~$V$ is a~{\kvs} and~$g \in \GL(V)$ is locally finite then the decomposition~$g = g_s g_u$ from \cref{local mjcd} is the \emph{\textup(multiplicative\textup) {\JCD} decompositon}\index{Jordan--Chevalley decomposition!for locally finite endomorphsims!multiplicative} of~$g$.
  The factor~$g_u$ is the \emph{unipotent part}\index{unipotent!part of!a locally finite endomorphism} of~$g$.
\end{definition}


% \begin{lemma}
%   Let~$V$ be a~{\kvs} an let~$g \in \GL(V)$ be locally finite.
%   If~$W \subseteq V$ is a~\dash{$g$}{invariant} subspace then~$W$ is also~\dash{$g_s$}{invariant} and~\dash{$g_u$}{invariant}, and the restriction~$\restrict{g}{W}$ is again invertible with
%   \[
%     \left( \restrict*{g}{W} \right)_s = \restrict*{g_s}{W}
%     \quad\text{and}\quad
%     \left( \restrict*{g}{W} \right)_u = \restrict*{g_u}{W} \,.
%   \]
% \end{lemma}
% 
% 
% \begin{proof}
%   The first equality is known from \cref{properties of local jcd} and the second follows because
%   \[
%       \left( \restrict*{g}{W} \right)_u
%     =   \id_W
%       + \left( \restrict*{g}{W} \right)_s^{-1}
%         \left( \restrict*{g}{W} \right)_n
%     = \restrict*{ \left( \id_V + g_s^{1} g_n \right) }{W}
%     = \restrict*{ g_u }{W}
%   \]
%   by again using \cref{properties of local jcd}.
% \end{proof}


\begin{lemma}
  \label{properties of local mjcd}
  Let~$V$ be a~{\kvs} and let~$g \in \GL(V)$ be locally finite.
  Let~$W \subseteq V$ be a~\dash{$g$}{invariant subspace}.
  \begin{enumerate}
    \item
      The subspace~$W$ is also~\dash{$g_s$}{invariant} and~\dash{$g_u$}{invariant}.
    \item
      \label{restriction of local mjcd}
      The restriction~$\restrict{g}{W}$ is again invertible and it holds that~$(\restrict{g}{W})_s = \restrict{g_s}{W}$ and~$(\restrict{g}{W})_u = \restrict{g_u}{W}$.
    \item
      \label{induced local mjcd}
      It holds for the induced endomorphisms~$\induced{g}, \induced{g_s}, \induced{g_u} \colon V \to V$ that~$\induced{g}$,~$\induced{g_s}$, and~$\induced{g_u}$ are invertible and locally finite, and the {\JCD} decompositon of~$\induced{g}$ is given by~$\induced{g} = \induced{g_s} \circ \induced{g_u}$.
    \item
      Let~$V'$ be another~{\kvs}, let~$g' \in \GL(V')$ be locally finite and let~$f \colon V \to V'$ be a~\dash{$k$}{linear} map with~$f \circ g = g' \circ f$, i.e.\ such that the diagram
      \begin{equation}
        \label{another endomorphism of local finite kx modules}
        \begin{tikzcd}
            V
            \arrow{r}[above]{g}
            \arrow{d}[left]{f}
          & V
            \arrow{d}[right]{f}
          \\
            V'
            \arrow{r}[above]{g'}
          & V'
        \end{tikzcd}
      \end{equation}
      commutes.
      Then it holds that~$f \circ g_s = g_s \circ f'$ and~$f \circ g_u = g'_u \circ f$, so that the diagram~\eqref{another endomorphism of local finite kx modules} factorizes as follows:
      \[
        \begin{tikzcd}
            {}
          & {}
          & {}
          \\
            V
            \arrow[rounded corners,
              to path = { -- ([yshift=1em] \tikztostart.north)
                          -- ([yshift=1em] \tikztotarget.north) \tikztonodes
                          -- (\tikztotarget)
                        }
            ]{rr}[above]{g}
            \arrow{r}[above]{g_u}
            \arrow{d}[left]{f}
          & V
            \arrow{r}[above]{g_s}
            \arrow{d}[right]{f}
          & V
            \arrow{d}[right]{f}
          \\
            V'
            \arrow{r}[above]{g'_u}
            \arrow[rounded corners,
              to path = { -- ([yshift=-1em] \tikztostart.south)
                          -- ([yshift=-1em] \tikztotarget.south) \tikztonodes
                          -- (\tikztotarget)
                        }
            ]{rr}[below]{g}
          & V'
            \arrow{r}[above]{g'_s}
          & V'
        \end{tikzcd}
      \]
  \end{enumerate}
\end{lemma}


\begin{proof}
  \leavevmode
  \begin{enumerate}[start=2]
    \item
      The restriction~$\restrict{g}{W}$ is again locally finite and so there exists a cover~$W = \bigcup_{i \in I} W_i$ by {\fd}~\dash{$g$}{invariant} subspaces~$W_i \subseteq V$.
      The restriction~$\restrict{g}{W_i}$ is invertible with~$(\restrict{g}{W_i})^{-1} = \restrict{g^{-1}}{W_i}$ for every~$i \in I$, which shows that very~$W_i$ is~\dash{$(g^{-1})$}{invariant}.
      It follows that~$W$ is~\dash{$(g^{-1}) $}{invariant}, and therefore that~$\restrict{g}{W}$ is invertible with~$(\restrict{g}{W})^{-1} = \restrict{g^{-1}}{W}$.
      
      If~$g = g_s + g_n$ is the additive {\JCD} of~$g$ then it follows from \cref{properties of local jcd} that the subspace~$W$ is invariant under both~$g_s$ and~$g_n$ and that the additive {\JCD} of the restriction~$\restrict{g}{W}$ is given by~$\restrict{g}{W} = \restrict{g_s}{W} + \restrict{g_n}{W}$.
      It follows from the invertibility of~$\restrict{g}{W}$ that~$(\restrict{g}{W})_s = \restrict{g_s}{W}$ is invertible, and it holds that~$(\restrict{g_s}{W})^{-1} = \restrict{g_s^{-1}}{W}$.
      It follows that
      \[
          \left( \restrict*{g}{W} \right)_u
        =   \id_W
          + \left( \restrict*{g_s}{W} \right)^{-1}
            \left( \restrict*{g_n}{W} \right)
        = \restrict*{ \left( \id_V + g_s^{-1} g_n \right) }{W}
        = \restrict*{ g_u }{W} \,.
      \]
    \addtocounter{enumi}{-2}
    \item
      This follows from the above.
    \addtocounter{enumi}{1}
    \item
      It follows from~$g = g_s g_u$ that~$\induced{g} = \induced{g_s} \, \induced{g_n}$, the induced endomorphisms~$\induced{g_s}$ and~$\induced{g_n}$ commute because~$g_s$ and~$g_n$ commute, the endomorphism~$\induced{g_s}$ is again semisimple by \cref{properties of ss endomorphisms} and the endomorphism~$\induced{g_u}$ is again locally unipotent.
    \item
      If~$f$ is injective then the claim follows from part~\ref*{restriction of local mjcd} and if~$f$ is surjective then the claim follows from part~\ref*{induced local mjcd}.
      The claim follows for general~$f$ as in the \hyperref[properties of jcd proof]{proof of \cref{properties of jcd}}.
    \qedhere
  \end{enumerate}
\end{proof}


\begin{lemma}
  \label{local mjcd for commuting endomorphisms}
  Let~$V$ be a~{\kvs} and let~$g_1, g_2 \in \GL(V)$ be locally finite and commuting with each other.
  Then
  \[
      ( g_1 g_2 )_s
    = ( g_1 )_s ( g_2 )_s
    \quad\text{and}\quad
      ( g_1 g_2 )_u
    = ( g_1 )_u ( g_2 )_u \,.
  \]
\end{lemma}


\begin{proof}
  The factors~$(g_1)_s$,~$(g_1)_u$,~$(g_2)_s$ and~$(g_2)_u$ commute with each other because~$g_1$ and~$g_2$ commute.
  It therefore holds that
  \[
      g_1 g_2
    = (g_1)_s (g_1)_u (g_2)_s (g_2)_u
    = \bigl( (g_1)_s (g_2)_s \bigr) \bigl( (g_1)_u (g_2)_u \bigr) \,.
  \]
  The factors~$(g_1)_s (g_2)_s$ and~$(g_1)_u (g_2)_u$ commute with each other, the factor~$(g_1)_s (g_2)_s$ is again semisimple by \cref{combination of ss endomorphisms} and the factor~$(g_1)_u (g_2)_u$ is again locally unipotent by \cref{combination of locally potent endomorphisms}.
\end{proof}




