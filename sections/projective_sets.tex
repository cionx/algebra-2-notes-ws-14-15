\section{(Quasi-)Projective Sets}


\begin{definition}
  The \emph{projective~\dash{$n$}{space}} over~$k$ is~$\Proj^n \defined \Proj^n_k \defined ( k^{n+1} \setminus \{0\} )/{\sim}$ where~$\sim$ is the equivalence relation defined by
  \[
          x \sim y
    \iff  \exists\, \lambda \in k^\times : x = \lambda y
    \iff  \gen{x}_k = \gen{y}_k \,.
  \]
  The equivalence class of~$(x_0, \dotsc, x_n) \in k^{n+1} \setminus \{0\}$ in~$\Proj^n$ is denoted by~$[x_0 \hd \dotsb \hd x_n]$, and the tupel~$(x_0, \dotsc, x_n)$ are \emph{homogeneous coordinates of~$[x_0 \hd \dotsb \hd x_n]$}.
\end{definition}


\begin{lemma}
  \label{characterization of homogeneous polynomials}
  For a polynomial~$f \in k[x_0, \dotsc, x_n]$ and a degree~$d \geq 0$ the following conditions are equivalent:
  \begin{enumerate}
    \item
      $f$ contains only monomials of degree~$d$.
    \item
      $f(\lambda x_0, \dotsc, \lambda x_n) = \lambda^d f(x_0, \dotsc, x_n)$ for every~$\lambda \in k$.
    \item
      $f(\lambda x) = \lambda^d f(x)$ for all~$\lambda \in k$ and~$x \in k^{n+1}$.
  \end{enumerate}
\end{lemma}


\begin{definition}
  A polynomial~$f \in k[x_0, \dotsc, x_n]$ which satisfies one (and thus all) of the conditions from \cref{characterization of homogeneous polynomials} is \emph{homogeneous of degree~$d$}.
  The set of homogeneous polynomials~$f \in k[x_0, \dotsc, x_n]$ of degree~$d$ is denoted by~$k[x_0, \dotsc, x_n]^d$.
\end{definition}


\begin{lemma}
  Let~$f, g\in k[x_0, \dotsc, x_n]$.
  \begin{enumerate}
    \item
      If~$f,g$ are homogeneous of degree~$d \geq 0$ then the linear combination~$\lambda f + \mu g$ is again homogeneous of degree~$d$ for all~$\lambda, \mu \in k$.
    \item
      If~$f$ is homogeneous of degree~$d_1$ and~$g$ is homogeneous of degree~$d_2$ then~$f \cdot g$ is homogeneous of degree~$d_1 + d_2$.
    \item
      It holds that~$k[x_0, \dotsc, x_n] = \bigoplus_{d \geq 0} k[x_0, \dotsc, x_n]_d$.
  \end{enumerate}
  Together this shows that~$k[x_0, \dotsc, x_n]$ is an~\dash{($\mathbb{N}$}{)graded}~\dash{$k$}{algebra} with homegenous components~$k[x_0, \dotsc, x_n]^d$.
\end{lemma}


\begin{definition}
  For~$f \in k[x_0, \dotsc, x_n]$ the unique polynomials~$f_d \in k[x_0, \dotsc, x_n]^d$ for~$d \in \Natural$ with~$f = \sum_{d \geq 0} f_d$ are the \emph{homogeneous parts of~$f$}.
\end{definition}


\begin{lemma}
  \label{characterization of homogeneous ideals}
  For an ideal~$I \idealleq k[x_0, \dotsc, x_n]$ the following conditions are equivalent:
  \begin{enumerate}
    \item
      $I$ is generated by homogeneous elements.
    \item
      $I = \bigoplus_{d \geq 0} I \cap k[x_0, \dotsc, x_n]^d$.
    \item
      $I = \bigoplus_{d \geq 0} I_d$ for some~\dash{$k$}{linear} subspaces~$I_d \moduleleq k[x_0, \dotsc, x_n]^d$.
  \end{enumerate}
\end{lemma}


\begin{definition}
  An ideal~$I \idealleq k[x_0, \dotsc, x_n]$ is \emph{homogeneous} if it satisfies one (and thus all) of the conditions in \cref{characterization of homogeneous ideals}.
\end{definition}


\begin{fluff}
  For a polynomial~$f \in k[x_0, \dotsc, x_n]$ and a point~$[x] \in \Proj^n$ the value~$f([x])$ is not well-defined, but if~$f$ is homogeneous then the condition~$f(x) = 0$ is well-defined.
  That allows us to define vanishing sets in projective space.
\end{fluff}


\begin{definition}
  \leavevmode
  \begin{enumerate}
    \item
      Let~$S \subseteq k[x_0, \dotsc, x_n]$ consist of homogeneous polynomials.
      Then the set
      \[
                  \vset_{\Proj}(S)
        \defined  \{
                    [x] \in \Proj^n
                  \suchthat
                    \text{$f(x) = 0$ for every~$f \in S$}
                  \}
      \]
      is the \emph{vanishing locus} of~$S$ or \emph{projective set} given by~$S$.
    \item
      For any homogeneous ideal~$I \idealleq k[x_0, \dotsc, x_n]$ the set
      \begin{align*}
                    \vset_{\Proj}(I)
        \defined{}& \{
                      [x] \in \Proj^n
                    \suchthat
                      \text{$f(x) = 0$ for every homogeneous~$f \in I$}
                    \} \\
        ={}&        \{
                      [x] \in \Proj^n
                    \suchthat
                      \text{$f(x) = 0$ for every~$f \in I$}
                    \}
      \end{align*}
      is the \emph{vanishing locus} of~$I$ or \emph{projective set} given by~$I$.
  \end{enumerate}
\end{definition}




