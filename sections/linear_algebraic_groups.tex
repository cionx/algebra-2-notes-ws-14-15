\chapter{Affine Varities}





% \addtocounter{section}{-1}
% \section{Basic Definition}
% 
% 
% \begin{conventions}
%   Throughout these notes $k$ denotes an algebraically closed field.
% \end{conventions}
% 
% 
% \begin{definition}
%   A \emph{linear algebraic group} is a subgroup~$G \groupleq \GL_n(k)$ (for some~$n$) which is cut out by polynomials, i.e.\ there exist~$f_1, \dotsc, f_r \in k[x_{11}, \dotsc, x_{nn}]$ such that
%   \[
%       G
%     = \{
%         A \in \GL_n(k)
%       \suchthat
%         f_1(A) = \dotsb = f_n(A) = 0
%       \} \,.
%   \]
% \end{definition}
% 
% 
% \begin{example}
%   \leavevmode
%   \begin{enumerate}
%     \item
%       The \emph{general linear group} $\GL_n(k)$.
%     \item
%       The \emph{multiplicative group}~$\Gmult \defined \GL_1(k) = k^\times$.
%     \item
%       The \emph{additive group}~$\Gadd \defined (K,{+})$ can be regarded as a linear algebraic group by identifying it with
%       \[
%                   \Gadd
%         \cong     \left\{
%                     \begin{pmatrix}
%                       1 & x \\
%                       0 & 1
%                     \end{pmatrix}
%                   \suchthat*
%                     x \in k
%                   \right\}
%         \groupleq \GL_2(k) \,.
%       \]
%     \item
%       The \emph{special linear group}~$\SL_n(k) = \{ A \in \GL_n(k) \suchthat \det A = 1 \}$.
%     \item
%       The \emph{orthogonal groups}~$\Orth_n(k) = \{ A \in \GL_n(k) \suchthat A^T A = I \}$.
%     \item
%       The \emph{special orthogonal group}~$\SOrth_n(k) = \SL_n(k) \cap \Orth_n(k)$.
%     \item
%       The \emph{symplectic group}~$\Symp_{2n}(k) = \{ A \in \GL_n(k) \suchthat A^T J A = J \}$ where~$
%           J
%         = \begin{psmallmatrix}
%              0    & I_n \\
%             -I_n  & 0
%           \end{psmallmatrix}
%       $.
%     \item
%       Every finite subgroup of $\GL_n(k)$ is a linear algebraic group.
%     \item
%       The \emph{group of diagonal matrices}~$
%           \Diag_n(k)
%         = \left\{
%             \begin{psmallmatrix}
%               * &        &    \\
%                 & \ddots &    \\
%                 &        & *
%             \end{psmallmatrix}
%             \in \GL_n(k)
%           \right\}
%       $.
%     \item
%       The \emph{group of upper triangular matrices}~$
%           \Triag_n(k)
%         = \left\{
%             \begin{psmallmatrix}
%               * & \cdots & *      \\
%                 & \ddots & \vdots \\
%                 &        & *
%             \end{psmallmatrix}
%             \in \GL_n(k)
%           \right\} \,.
%       $
%     \item
%       The \emph{group of unipotent matrices}~$
%           \Uni_n(k)
%         = \left\{
%             \begin{psmallmatrix}
%               1 & \cdots & *      \\
%                 & \ddots & \vdots \\
%                 &        & 1
%             \end{psmallmatrix}
%             \in \GL_n(k)
%           \right\} \,.
%       $
%   \end{enumerate}
% \end{example}





\section{Affine Sets}
\label{affine sets}


\begin{conventions}
  Throughout these notes~$k$ denotes an algebraically closed field.
\end{conventions}


\begin{fluff}
  In this \nameCref{affine sets} we recall the notion of an affine set and some of the basic theorems about them.
  The missing proofs can, for example, be found in the \cite{algebra1notes}
\end{fluff}





\subsection{Definition}


\begin{definition}
  The \emph{affine~\dash{$n$}{space}}\index{affine!space} over~$k$ is~$\Aff^n \defined \Aff^n_k \defined k^n$.
\end{definition}


\begin{fluff}
  It follows from~$k$ being infinite that~$f = g$ for all~$f, g \in k[x_1, \dotsc, x_n]$ with~$f(x) = g(x)$ for every~$x \in \Aff^n$.
  We will therefore regard the polynomial ring~$k[x_1, \dotsc, x_n]$ as the ring of polynomial functions on~$\Aff^n$.
\end{fluff}


\begin{definition}
  \label{definition of affine sets}
  For every subset~$S \subseteq k[x_1, \dotsc, x_n]$ the set
  \[
              \vset(S)
    \defined  \{
                x \in \Aff^n
              \suchthat
                \text{$f(x) = 0$ for every~$f \in S$}
              \}
  \]
  is the \emph{affine set}\index{affine!set} given by~$S$.
\end{definition}


\begin{lemma}
  \label{properties of vanishing sets}
  \leavevmode
  \begin{enumerate}
    \item
      It holds for every subset~$S \subseteq k[x_1, \dotsc, x_n]$ that~$\vset(S) = \vset( (S) )$.
    \item
      It holds for all ideals~$I_1 \idealleq I_2 \idealleq k[x_1, \dotsc, x_n]$ that~$\vset(I_1) \supseteq \vset(I_2)$.
    \item
      It holds for all ideals~$I_1, I_2 \idealleq k[x_1, \dotsc, x_n]$ that
      \[
          \vset(I_1) \cup \vset(I_2)
        = \vset(I_1 \cap I_2)
        = \vset(I_1 \cdot I_2) \,.
      \]
    \item
      It holds for every family~$(I_\lambda)_{\lambda \in \Lambda}$ of ideals~$I_\lambda \idealleq k[x_1, \dotsc, x_n]$ that
      \[
          \bigcap_{\lambda \in \Lambda} \vset(I_\lambda)
        = \vset\left( \bigcup_{\lambda \in \Lambda} I_\lambda \right)
        = \vset\left( \sum_{\lambda \in \Lambda} I_\lambda \right) \,.
      \]
    \item
      It holds that~$\vset(1) = \emptyset$.
    \item
      It holds that~$\vset(0) = \Aff^n$.
    \qed
  \end{enumerate}
\end{lemma}


\begin{corollary}
  \label{existence of zariski topology}
  There exists a unique topology on~$\Aff^n$ whose closed subsets are the subsets of the form~$\vset(S)$ for subsets~$S \subseteq k[x_1, \dotsc, x_n]$.
  \qed
\end{corollary}


\begin{definition}
  The topology from \cref{existence of zariski topology} is the \emph{Zariski topology}\index{Zariski topology} on~$\Aff^n$.
  The induced subspace topology on a subset~$X \subseteq \Aff^n$ is the \emph{Zariski topology}\index{Zariski topology!for affine sets} on~$X$.
\end{definition}


\begin{theorem}[Hilbert’s basis theroem]
  \index{Hilbert!basis theorem}
  Every ideal~$I \idealleq k[x_1, \dotsc, x_n]$ is finitely generated.
  \qed
\end{theorem}


\begin{corollary}
  It holds for every subset~$S \subseteq k[x_1, \dotsc, x_n]$ that
  \[
      \vset(S)
    = \vset(f_1, \dotsc, f_m)
    = \vset(f_1) \cap \dotsb \cap \vset(f_m) \,.
  \]
  for some finitely many~$f_1, \dotsc, f_m \in S$.
  \qed
\end{corollary}





\subsection{Coordinate Rings}


\begin{definition}
  For every subset~$X \subseteq \Aff^n$ the set
  \[
              \videal(X)
    \defined  \{
                f \in k[x_1, \dotsc, x_n]
              \suchthat
                \text{$f(x) = 0$ for every~$x \in X$}
              \}
  \]
  is the \emph{vanishing ideal}\index{vanishing!ideal} of~$X$.
\end{definition}


\begin{lemma}
  \label{properties of vanishing ideals}
  \leavevmode
  \begin{enumerate}
    \item
      For every subset~$X \subseteq \Aff^n$ the vanishing ideal~$\videal(X)$ is an ideal in~$k[x_1, \dotsc, x_n]$.
    \item
      It holds for all subsets~$X_1 \subseteq X_2 \subseteq \Aff^n$ that~$\videal(X_1) \idealgeq \videal(X_2)$.
    \item
      It holds for every family~$(X_\lambda)_{\lambda \in \Lambda}$ of subsets~$X_\lambda \subseteq \Aff^n$ that
      \[
          \videal\left( \bigcup_{\lambda \in \Lambda} X_\lambda \right)
        = \bigcap_{\lambda \in \Lambda} \videal(X_\lambda) \,.
      \]
    \item
      It holds that~$\videal(\emptyset) = (1) = k[x_1, \dotsc, x_n]$.
    \item
      It holds that~$\videal(\Aff^n) = 0$ (because~$k$ is infinite).
    \qed
  \end{enumerate}
\end{lemma}


\begin{definition}
  The \emph{coordinate ring}\index{coordinate ring!of affine sets} of an affine set~$X \subseteq \Aff^n$ is the~\dash{$k$}{algebra}
  \[
              \coord(X)
    \defined  k[x_1, \dotsc, x_n] / {\videal(X)} \,.
  \]
\end{definition}
% TODO:: Make quotients into binary operators.


\begin{fluff}
  For an affine set~$X \subseteq \Aff^n$ any two polyomial functions~$f, g \in k[x_1, \dotsc, x_n]$ coincide on~$X$ in the sense that~$f(x) = g(x)$ for all~$x \in X$ if and only if~$f - g \in \videal(X)$, i.e.\ if and only if~$f$ and~$g$ are identified in~$\coord(X)$.
  We will therefore regard the coordinate ring~$\coord(X)$ as the ring of polynomial functions on~$X$.
  
  We can then define for every ideal~$I \idealleq \coord(X)$ the \emph{vanishing set}\index{vanishing!set}
  \[
              \vset(I)
    \defined  \vset_X(I)
    \defined  \{
                x \in X
              \suchthat
                \text{$f(x) = 0$ for every~$f \in I$}
              \}
  \]
  and can define for every subset~$Y \subseteq X$ the \emph{vanishing ideal}\index{vanishing!ideal}
  \[
              \videal(Y)
    \defined  \videal_X(Y)
    \defined  \{
                f \in \coord(X)
              \suchthat
                \text{$f(y) = 0$ for every~$y \in Y$}
              \} \,.
  \]
  For~$X = \Aff^n$ this agrees with the previous definitions of vanishing sets and vanishing ideals.
  The properties from \cref{properties of vanishing sets} and \cref{properties of vanishing ideals} also hold for this more general definitions.
\end{fluff}


\begin{lemma}
  The Zariski closed subsets of an affine set~$X$ are precisely the subsets of the form~$\vset(I)$ with~$I \idealleq \coord(X)$.
  \qedhere
\end{lemma}


\begin{corollary}
  \label{characterization of zariski dense}
  Let~$X$ be an affine set.
  For a subset~$X' \subseteq X$ the following conditions are equivalent:
  \begin{enumerate}
    \item
      The set is Zariski dense in~$X$.
    \item
      It follows for every~$f \in \coord(X)$ from~$\restrict{f}{X'} = 0$ that~$f = 0$.
    \item
      It follows for all~$f, g \in \coord(X)$ from~$\restrict{f}{X'} = \restrict{g}{X'}$ that~$f = g$.
  \end{enumerate}
\end{corollary}


\begin{proof}
  It holds that
  \begin{align*}
        {}& \text{$X'$ is Zariski dense in~$X$}                                                 \\
    \iff{}& \text{if~$C \subseteq X$ is a Zariski closed with~$X' \subseteq C$ then~$C = X$}    \\
    \iff{}& \text{if~$f \in \coord(X)$ with~$X' \subseteq \vset(f)$ then~$\vset(f) = X$}        \\
    \iff{}& \text{if~$f \in \coord(X)$ with~$\restrict{f}{X'} = 0$ then~$f = 0$}                \\
    \iff{}& \text{if~$f, g \in \coord(X)$ with~$\restrict{(f-g)}{X'} = 0$ then~$f-g = 0$}       \\
    \iff{}& \text{if~$f, g \in \coord(X)$ with~$\restrict{f}{X'} = \restrict{g}{X'}$ then~$f = g$}
  \end{align*}
  as claimed.
\end{proof}


\begin{corollary}
  \label{standard basis of zariski topology}
  If~$X$ is an affine set then the sets
  \[
              \Dopen(f)
    \defined  \Dopen_X(f)
    \defined  \{
                x \in X
              \suchthat
                f(x) \neq 0
              \}
  \]
  with~$f \in \coord(X)$ form a basis for the Zariski topology of~$X$.
\end{corollary}



\begin{proof}
  The sets~$\Dopen(f)$ are Zariski open because~$\Dopen(f) = X \setminus \vset(f)$.
  If~$U \subseteq X$ is any Zariski open subset then the complement~$X \setminus U$ is Zariski closed and thus of the form~$X \setminus U = \vset(I)$ for some ideal~$I \idealleq \coord(X)$.
  If~$f_\lambda \in I$,~$\lambda \in \Lambda$ is a generating set of~$I$ then it follows from
  \begin{gather*}
      \vset(I)
    = \vset(f_\lambda \suchthat \lambda \in \Lambda)
    = \bigcap_{\lambda \in \Lambda} \vset(f_\lambda)
  \shortintertext{that}
      U
    = X \setminus \vset(I)
    = X \setminus \bigcap_{\lambda \in \Lambda} \vset(f_\lambda)
    = \bigcup_{\lambda \in \Lambda} \big( X \setminus \vset(f_\lambda) \big)
    = \bigcup_{\lambda \in \Lambda} \Dopen(f_\lambda)
  \end{gather*}
  as desired.
\end{proof}


\begin{definition}
  For an affine set~$X$ the open subsets~$\Dopen(f) \subseteq X$ with~$f \in \coord(X)$ are the \emph{standard open subsets}\index{standard open subset} of~$X$.
\end{definition}


\begin{notation}
  If more generally~$X'$ is any set,~$f \colon X' \to k$ is a function and~$X \subseteq X'$ is a subset then we will use the notation
  \[
              \Dopen_X(f)
    \defined  \{
                x \in X
              \suchthat
                f(x) \neq 0
              \} \,.
  \]
\end{notation}


\begin{definition}
  Let~$R$ be a commutative ring.
  \begin{enumerate}
    \item
      The ring~$R$ is \emph{reduced}\index{reduced ring} if~$0 \in R$ is the only nilpotent element of~$R$.
    \item
      The \emph{radical}\index{radical!of an ideal} of an ideal~$I \idealleq R$ is
      \[
                  \rad{I}
        \defined  \{
                    f \in R
                  \suchthat
                    \text{$f^n \in I$ for some~$n \geq 0$}
                  \} \,.
      \]
    \item
      An ideal~$I \idealleq R$ is \emph{radical}\index{radical!ideal} if~$I = \rad{I}$.
  \end{enumerate}
\end{definition}


\begin{lemma}
  Let~$R$ be a commutative ring.
  \begin{enumerate}
    \item
      For every ideal~$I \idealleq R$ its radical~$\rad{I}$ is again an ideal in~$R$.
    \item
      An ideal~$I \idealleq R$ is radical if and only if the quotient~$R/I$ is reduced.
    \qed
  \end{enumerate}
\end{lemma}


\begin{lemma}
  The ideal~$\videal(X) \idealleq k[x_1, \dotsc, x_n]$ is radical for every subset~$X \subseteq \Aff^n$.
  \qed
\end{lemma}


\begin{corollary}
  \label{coordinate ring is fg commutative reduced}
  For every affine set~$X$ its coordinate ring~$\coord(X)$ is a finitely generated, commutative, reduced~\dash{$k$}-algebra.
  \qed
\end{corollary}


\begin{theorem}[Hilbert’s~Nullstellensatz, version~1]
  \index{Hilbert!Nullstellensatz}
  \label{nullstellensatz 1}
  If~$X$ is an affine set then holds for every ideal~$I \idealleq \coord(X)$ that~$\videal( \vset( I ) ) = \rad{I}$.
  \qed
\end{theorem}


\begin{theorem}[Hilbert’s~Nullstellensatz, version~2]
  \index{Hilbert!Nullstellensatz}
  \label{nullstellensatz 2}
  If~$X$ is an affine set and~$I \ideallneq \coord(X)$ is a proper ideal then its vanishing set~$\vset(I)$ is nonempty.
  \qed
\end{theorem}



\subsection{Irreducibility}


\begin{definition}
  Let~$X$ be a topological space.
  \begin{enumerate}
    \item
      The space~$X$ is \emph{reducible}\index{reducible topological space} if~$X = C_1 \cup C_2$ for some proper closed subsets~$C_1, C_2 \subsetneq X$.
    \item
      The space~$X$ is \emph{irreducible}\index{irreducible!topological space} if it is nonempty and not reducible.
  \end{enumerate}
\end{definition}


\begin{lemma}
  For a nonempty topological space~$X$ the following conditions are equivalent:
  \begin{enumerate}
    \item
      The space~$X$ is irreducible.
    \item
      Every two nonempty open subsets of~$X$ intersect nontrivially.
    \item
      Every nonemtpy open subset of~$X$ is dense.
    \qed
  \end{enumerate}
\end{lemma}


\begin{lemma}
  \label{irreducible is connected}
  Every irreducible space is connected.
  \qed
\end{lemma}


\begin{example}
  The affine set~$\vset(xy) \subseteq \Aff^2$ is connected but reducible, which shows that the converse to \cref{irreducible is connected} does not hold.
\end{example}


\begin{proposition}
  \label{existence of irreducible components}
  For every topological space~$X$ there exist a unique collection~$( C_\lambda )_{\lambda \in \Lambda}$ of closed irreducible subsets~$C_\lambda \subseteq X$ with~$X = \bigcup_{\lambda \in \Lambda} C_\lambda$ and~$C_\lambda \nsubseteq C_\mu$ for~$\lambda \neq \mu$.
  \qed
\end{proposition}


\begin{definition}
  The sets~$C_\lambda$,~$\lambda \in \Lambda$ from \cref{existence of irreducible components} are the \emph{irreducible components}\index{irreducible!component} of~$X$.
\end{definition}


\begin{definition}
  A topological space~$X$ is \emph{noetherian}\index{noetherian topological space} if every descending sequence
  \[
              C_1
    \supseteq C_2
    \supseteq C_3
    \supseteq \dotsb
  \]
  of closed subsets~$C_i \subseteq X$ stabilizes, or equivalently if every ascending sequence
  \[
              U_1
    \subseteq U_2
    \subseteq U_3
    \subseteq \dotsb
  \]
  of open subsets~$U_i \subseteq X$ stabilizes.
\end{definition}


\begin{lemma}
  Subspaces of noetherian topological spaces are again noetherian.
  \qed
\end{lemma}


\begin{lemma}
  Any affine set is noetherian.
  \qed
\end{lemma}


\begin{lemma}
  A noetherian topological space has only finitely many irreducible components.
  \qed
\end{lemma}


\begin{corollary}
  An affine set has only finitely many irreducible components.
  \qed
\end{corollary}


\begin{theorem}[Hilbert’s~Nullstellensatz, version~3]
  \index{Hilbert!Nullstellensatz}
  \label{nullstellensatz 3}
  For every affine set~$X$, the maps~$\vset_X, \videal_X$ restrict to the following bijections:
  \[
    \begin{matrix}
        \left\{
          \begin{tabular}{@{}c@{}}
              affine  algebraic \\
              sets~$Y \subseteq X$
          \end{tabular}
        \right\}
      & \begin{tikzcd}[column sep = large]
            {}
            \arrow[shift left]{r}{\videal_X}
          & {}
            \arrow[shift left]{l}{\vset_X}
        \end{tikzcd}
      & \left\{
          \begin{tabular}{@{}c@{}}
            radical ideals \\
            $I \idealleq \coord(X)$
          \end{tabular}
        \right\}
      \\
        {}
      & {}
      & {}
      \\
        \rotatebox[origin=c]{90}{$\subseteq$}
      & {}
      & \rotatebox[origin=c]{90}{$\subseteq$}
      \\
        {}
      & {}
      & {}
      \\
        \left\{
          \begin{tabular}{@{}c@{}}
              irreducible affine \\
              algebraic set \\
              $Y \subseteq X$
          \end{tabular}
        \right\}
      & \begin{tikzcd}[column sep = large]
            {}
            \arrow[shift left]{r}{\videal_X}
          & {}
            \arrow[shift left]{l}{\vset_X}
        \end{tikzcd}
      & \left\{
          \begin{tabular}{@{}c@{}}
            prime ideals \\
            $\mf{p} \idealleq \coord(X)$
          \end{tabular}
        \right\}
      \\
        {}
      & {}
      & {}
      \\
        \rotatebox[origin=c]{90}{$\subseteq$}
      & {}
      & \rotatebox[origin=c]{90}{$\subseteq$}
      \\
        {}
      & {}
      & {}
      \\
        \left\{
          \begin{tabular}{@{}c@{}}
            points~$p \in X$
          \end{tabular}
        \right\}
      & \begin{tikzcd}[column sep = large]
            {}
            \arrow[shift left]{r}{\videal_X}
          & {}
            \arrow[shift left]{l}{\vset_X}
        \end{tikzcd}
      & \left\{
          \begin{tabular}{@{}c@{}}
            maximal ideals \\
            $\mf{m} \idealleq \coord(X)$
          \end{tabular}
        \right\}
    \end{matrix}
  \]
  For every point~$p = (p_1, \dotsc, p_n) \in X$ the corresponding maximal ideal~$\mf{m}_p \idealleq \coord(X)$ is given by~$\mf{m}_p = (\class{x_1} - p_1, \dotsc, \class{x_n} - p_n)$.
  \qed
\end{theorem}


\begin{corollary}
  \label{containment of D}
  If~$X$ is an affine set then it holds for all~$f, g \in \coord(X)$ that~$\Dopen(f) \subseteq \Dopen(g)$ if and only if~$f \in \rad{\genideal{g}}$.
\end{corollary}


\begin{proof}
  It holds that
  \[
          \Dopen(f) \subseteq \Dopen(g)
    \iff  \vset(f) \supseteq \vset(g)
    \iff  \videal(\vset(f)) \subseteq \videal(\vset(g))
    \iff  \rad{\genideal{f}} \subseteq \rad{\genideal{g}}
    \iff  f \in \rad{\genideal{g}} \,,
  \]
  as desired.
\end{proof}





\subsection{Morphisms of Affine Sets}


\begin{definition}
  \label{regular for affine}
  Let~$X,Y$ be affine sets with~$Y \subseteq \Aff^m$.
  \begin{enumerate}
    \item
      A function~$f \colon X \to k = \Aff^1$ is \emph{regular}\index{regular!for affine sets} if it is an element of~$\coord(X)$.
    \item
      A map~$f \colon X \to \Aff^n$ is \emph{regular}\index{regular!for affine sets} if it is regular in each coordinate.
    \item
      A map~$f \colon X \to Y$ is \emph{regular}\index{regular!for affine sets} if it is the restriction of a regular map~$X \to \Aff^m$.
  \end{enumerate}
  A map~$X \to Y$ is a \emph{morphism}\index{homomorphism@(homo)morphism!of affine sets}\index{morphism|see {(homo)morphism}} of affine sets if it is regular.
  The set of morphisms~$X \to Y$ is denoted by~$\Mor(X,Y)$.
\end{definition}


\begin{lemma}
  Let~$X, Y, Z$ be affine sets.
  \begin{enumerate}
    \item
      The identity map~$\id_X \colon X \to X$ is a morphism.
    \item
      For every two morphisms~$f \colon X \to Y$ and~$g \colon Y \to Z$ their composition~$g \circ f \colon X \to Z$ is again morphism.
    \qed
  \end{enumerate}
\end{lemma}


\begin{lemma}
  Let~$X,Y$ be affine sets and let~$f \colon X \to Y$ be a morphism of affine sets.
  \begin{enumerate}
    \item
      It holds for every~$\varphi \in \coord(Y)$ that~$f^{-1}(\Dopen_Y(\varphi)) = \Dopen_X(\varphi \circ f)$.
    \item
      The map~$f$ is continuous with respect to the Zariski topologies on~$X, Y$.
    \qed
  \end{enumerate}
\end{lemma}


\begin{lemma}
  \label{fuctoriality of the coordinate ring}
  Let~$X, Y, Z$ be affine sets.
  \begin{enumerate}
    \item
      If~$f \colon X \to Y$ is a morphism of affine sets then the map
      \[
                f^*
        \colon  \coord(Y)
        \to     \coord(X) \,,
        \quad   \varphi
        \mapsto \varphi \circ f
      \]
      is a well-defined homomorphism of~\dash{$k$}{algebras}.
    \item
      It holds that~$\id_X^* = \id_{\coord(X)}$.
    \item
      It holds for any two composable morphisms of affine sets~$f \colon X \to Y$,~$g \colon Y \to Z$ that~$(g \circ f)^* = f^* \circ g^*$.
    \qed
  \end{enumerate}
\end{lemma}


\begin{fluff}[Finite sets as affine sets]
  If~$X$ is a finite affine set then every function~$X \to k$ is regular.
  It follows that for every affine set~$Y$ every map~$X \to Y$ is regular.
  If~$X'$ is another finite affine set with~$\card{X} = \card{X'}$ then it follows that every bijection~$X \to X'$ is already an isomorphism of affine sets.


  This shows that any two affine sets of the same cardinality are isomorphic as affine sets.
  This allows us to regard every finite set as an affine set by identifying it with an (up to isomorphicsm unique) affine set of suitable cardinality.
\end{fluff}


\begin{lemma}
  Let~$X$,~$Y$ be affine sets and let~$f \colon X \to Y$ be a morphism of affine sets with image~$X' \defined \im(f)$.
  The induced morphism~$f^*$ is injective if and only if the image~$X'$ is dense in~$Y$.
\end{lemma}


\begin{proof}
  \leavevmode
  It holds that
  \begin{align*}
        &{} \text{$f^*$ in injective} \\
    \iff&{} \text{it follows for all~$\varphi \in \coord(G)$ from~$f^*(\varphi) = 0$ that~$\varphi = 0$} \\
    \iff&{} \text{it follows for all~$\varphi \in \coord(G)$ from~$\varphi \circ f = 0$ that~$\varphi = 0$} \\
    \iff&{} \text{it follows for all~$\varphi \in \coord(G)$ from~$\restrict{\varphi}{X'} = 0$ that~$\varphi = 0$} \\
    \iff&{} \text{$X'$ is dense in~$Y$}
  \end{align*}
  by \cref{characterization of zariski dense}.
\end{proof}



\begin{proposition}
  \label{coordinate ring is fully faithful}
  For any two affine sets~$X,Y$ the map
  \[
            \Mor(X, Y)
    \to     \Hom_{\cAlg{k}}( \coord(Y), \coord(X) ) \,,
    \quad   f
    \mapsto f^*
  \]
  is a well-defined bijection.
\end{proposition}


\begin{proof}
  We prove the claim by constructing an inverse to~$(-)^*$.
  
  With~$X \subseteq \Aff^n$ and~$Y \subseteq \Aff^m$ the coordinate rings~$\coord(X)$ and~$\coord(Y)$ are given by
  \[
      \coord(X)
    = k[x_1, \dotsc, x_n]/{\videal(X)}
    \quad\text{and}\quad
      \coord(Y)
    = k[y_1, \dotsc, y_m]/{\videal(Y)} \,.
  \]
  For any homomorphism of~\dash{$k$}{algebras}~$F \colon \coord(Y) \to \coord(X)$ we associate a morphism of affine sets~$\tilde{F}^\circ \colon X \to \Aff^m$ with coordinates~$\tilde{F}^\circ = (\tilde{F}^\circ_1, \dotsc, \tilde{F}^\circ_m)$ given by
  \[
        \tilde{F}^\circ_j
    =   F(\class{y_j})
    \in \coord(X)
  \]
  for every~$j = 1, \dotsc, m$.
  
  The morphism~$\tilde{F}^\circ \colon X \to \Aff^m$ restrict to a morphism~$F^\circ \colon X \to Y$:
  The affine set~$Y$ is given by~$Y = \vset(\videal(Y))$ so needs to be shown that~$p( \tilde{F}^\circ(x) ) = 0$ for all~$p \in \videal(Y)$,~$x \in X$.
  For this we calculate
  \begin{align}
        p(\tilde{F}^\circ(x))
    &=  p( \tilde{F}^\circ_1(x), \dotsc, \tilde{F}^\circ_m(x) )
        \label{equation: definiton of F circ} \\
    &=  p( F(\class{y_1})(x), \dotsc, F(\class{y_m})(x) )
        \label{equation: definiton of F circ j} \\
    &=  p( F(\class{y_1}), \dotsc, F(\class{y_m}) )(x)
        \label{equation: pulling x out} \\
    &=  F( p(\class{y_1}, \dotsc, \class{y_m}) )(x)
        \label{equation: pulling F out} \\
    &=  F\left( \class{p(y_1, \dotsc, y_m)} \right)(x)
        \label{equation: putting p in}\\
    &=  F(\class{p})(x)
        \nonumber \\
    &=  F(0)(x)
        \label{equation: p vanishes}  \\
    &=  0
        \nonumber \,.
  \end{align}
  \cref{equation: definiton of F circ} uses the definition of~$\tilde{F}^\circ$, \cref{equation: definiton of F circ j} uses the definiton of the components~$\tilde{F}^\circ_j$, \cref{equation: pulling x out} uses that the~\dash{$k$}{algebra} structure on~$\coord(X)$ is given pointwise, \cref{equation: pulling F out} uses that~$F$ is a~\dash{$k$}{algebra} homomorphism, \cref{equation: putting p in} uses that~$\class{(-)}$ is a~\dash{$k$}{algebra} homomorphism, and \cref{equation: p vanishes} uses that~$p \in \videal(Y)$.
  
  The constructions~$(-)^*$ and~$(-)^\circ$ are mutually inverse:
  If~$f \colon X \to Y$ is a morphism of affine sets with coordinates~$f = (f_1, \dotsc, f_m)$ then
  \[
      (f^*)^\circ_j
    = (f^*)(\class{y_j})
    = \class{y_j} \circ f
    = f_j
  \]
  for every~$j = 1, \dotsc, m$ and therefore~$(f^*)^\circ = f$.
  If~$F \colon \coord(Y) \to \coord(X)$ is a homomorphism of~\dash{$k$}{algebras} then
  \[
      (F^\circ)^*(\class{y_j})
    = \class{y_j} \circ F^\circ
    = F^\circ_j
    = F(\class{y_j})
  \]
  for every~$j = 1, \dotsc, m$ and therefore~$(F^\circ)^* = F$.
\end{proof}


\begin{lemma}
  \label{coordinate ring is dense}
  For every finitely generated, commutative, reduced~\dash{$k$}{algebra}~$A$ there exists an affine set~$X$ with~$A \cong \coord(X)$ as~\dash{$k$}{algebras}.
\end{lemma}


\begin{proof}
  Let~$a_1, \dotsc, a_n \in A$ be generating set of~$A$ as a~\dash{$k$}{algebra}.
  Then there exists a unique homomorphisms of~\dash{$k$}{algebras}~$f \colon k[x_1, \dotsc, x_n] \to A$ with~$f(x_i) = a_i$ for every~$i = 1, \dotsc, n$, and~$f$ is surjective.
  It follows that~$f$ induces an isomorphism of~\dash{$k$}{algebras}~$k[x_1, \dotsc, x_n]/I \to A$ for~$I \defined \ker(f)$.
  
  The ideal~$I$ is a radical ideal because the quotient~$k[x_1, \dotsc, x_n]/I \cong A$ is reduced.
  It follows from the \hyperref[nullstellensatz 3]{second version of Hilbert’s~Nullstellensatz} that~$\videal(X) = I$ for the affine set~$X \defined \vset(I)$.
  It follows that
  \[
          A
    \cong k[x_1, \dots, x_n]/I
    =     k[x_1, \dots, x_n]/{\videal(X)}
    =     \coord(X)
  \]
  as desired.
\end{proof}


\begin{corollary}
\label{equivalence for affine sets}
  The coordinate ring~$\coord(-)$ gives rise to a contravariant equivalence
  \begin{align*}
    \{
      \text{affine sets}
    \}
    &\longto
    \left\{
      \begin{tabular}{@{}c@{}}
        finitely generated, \\
        commutative,        \\
        reduced~\dash{$k$}{algebras}
      \end{tabular}
    \right\} \,,
    \\
    X
    &\longmapsto
    \coord(X) \,,
    \\
    f
    &\longmapsto
    f^* \,.
  \end{align*}
\end{corollary}


\begin{proof}
  It follows from \cref{coordinate ring is fg commutative reduced} and \cref{fuctoriality of the coordinate ring} that~$\coord(-)$ defines a functor as claimed.
  It follows from \cref{coordinate ring is fully faithful} that~$\coord(-)$ is fully faithful and it follows from \cref{coordinate ring is dense} that~$\coord(-)$ is dense.
\end{proof}


\begin{corollary}
  \label{induced is isomorphism or surjective}
  Let~$X$,~$Y$ be affine sets
  \begin{enumerate}
    \item
      \label{affine sets isomorphic iff coordinate rings isomorphic}
      The affine sets~$X, Y$ are isomorphic if and only if their coordinate rings~$\coord(X)$,~$\coord(Y)$ are.
  \end{enumerate}
  Let~$f \colon X \to Y$ be a morphism of affine sets with image~$X' \defined \im(f)$.
  \begin{enumerate}[resume]
    \item
      \label{isomorphism iff induced isomorphism}
      The morphism~$f$ is an isomorphism if and only if the induced algebra homomorphism~$f^*$ is an isomorphism.
    \item
      \label{induced surjective iff closed embedding}
      The induced map~$f^*$ is surjective if and only if~$X'$ is closed in~$Y$ and~$f$ is an isomorphism onto~$X'$.% make footnote not notice this linebreak
      \footnote{The author linkes to think about~$f$ as a closed embedding, but is not sure if this is how algebraic geometers use this term.}
  \end{enumerate}
\end{corollary}


\begin{proof}
  Parts~\ref*{affine sets isomorphic iff coordinate rings isomorphic} and~\ref*{isomorphism iff induced isomorphism} follows from \cref{equivalence for affine sets}.
  
  To show part~\ref*{induced surjective iff closed embedding} first suppose that~$X'$ is closed in~$Y \subseteq \Aff^m$ and that~$f$ restricts to an isomorphism~$X \to X'$.
  To show that~$f^*$ surjective we may assume that~$X = X' \subseteq Y$ and that~$f$ is the inclusion~$f \colon X \inclusion Y$.
  The induced algebra homomorphism~$f \colon \coord(Y) \to \coord(X)$ maps every~$\varphi \in \coord(Y)$ to~$\varphi \circ f$, which is the restriction of~$\varphi$ to~$X$.
  The homomorphism~$f^*$ is therefore surjective because every regular function on~$X$ is the restriction of a regular function on~$\Aff^m$, and therefore also of a regular function on~$Y$.
  
  Suppose now that~$f^*$ is surjective.
  The ideal~$I \defined \ker(f^*)$ is radical because the quotient~$\coord(Y)/I \cong \coord(X)$ is reduced.
  It follows from \hyperref[nullstellensatz 3]{Hilbert’s Nullstellensatz} that~$X'' \defined \vset(I)$ is an affine set~$X'' \subseteq Y$ with~$\coord(X'') = \coord(Y)/I \cong \coord(X)$.
  We show that~$X' = X''$ and that~$f$ restricts to an isomorphism~$X \to X'$:
  
  That~$0 = f^*(\varphi) = \varphi \circ f$ for every~$\varphi \in I$ means that
  \[
              X'
    =         \im(f)
    \subseteq \vset(I)
    =         X''
  \]
  It follows that~$f$ restricts to a morphism of affine varieties~$\tilde{f} \colon X \to X''$ which fits in the following commutative diagram:
  \[
    \begin{tikzcd}[column sep = large]
        {}
      & Y
      \\
        X
        \arrow{ru}[above left]{f}
        \arrow{r}[below]{\tilde{f}}
      & X''
        \arrow[hook]{u}
    \end{tikzcd}
  \]
  This induces on coordinate rings the following commutative diagram:
  \[
    \begin{tikzcd}[column sep = large]
        {}
      & \coord(Y)
        \arrow{dl}[above left]{f^*}
        \arrow{d}
      \\
        \coord(X)
      & \coord(X'')
        \arrow{l}[below]{\tilde{f}^*}
    \end{tikzcd}
  \]
  The homomorphism~$\coord(Y) \to \coord(X'')$ is the restriction homomorphism~$\varphi \mapsto \restrict{\varphi}{X''}$, which is by the above observations surjective with kernel~$\videal(X'') = \videal(\vset(I)) = I$.
  
  Both~$f^*$ and the restriction homomorphism~$\coord(Y) \to \coord(X'')$ are surjective algebra homomorphisms with the same kernel so it follows that there exists a unique algebra homomorphism~$\coord(X'') \to \coord(X)$ which makes the above diagram commute, and that it is an isomorphism.
  This shows that~$\tilde{f}^*$ is an isomorphism.
  It follows from part~\ref*{isomorphism iff induced isomorphism} that~$\tilde{f}$ is an isomorphism from~$X$ to~$X''$.
  The inclusion
  \[
              \im(f)
    =         X'
    \subseteq X''
    =         \im(\tilde{f})
    =         \im(f)
  \]
  is therefore already an equality~$X' = X''$, and~$f$ restricts to the isomorphism~$\tilde{f}$.
\end{proof}





\subsection{Products of Affine Sets}


\begin{lemma}
  \label{product of affine sets is an affine set}
  For any two affine sets~$X \subseteq \Aff^n$ and~$Y \subseteq \Aff^m$ the set
  \[
              X \times Y
    \subseteq \Aff^n \times \Aff^m
    =         \Aff^{n+m}
  \]
  is again an affine set.
\end{lemma}


\begin{proof}
  We may label the coordinates of~$\Aff^n$ by~$x_1, \dotsc, x_n$ while labeling the coordinates of~$\Aff^m$ by~$x_{n+1}, \dotsc, x_{n+m}$.
  The affine set~$X$ is then cut out by an ideal~$I \idealleq k[x_1, \dotsc, x_n]$ while~$Y$ is cut out by an ideal~$J \idealleq k[x_{n+1}, \dotsc, x_{n+m}]$.
  The set~$X \times Y$ is then cut out by the generated ideal~$\genideal{I,J} \idealleq k[x_1, \dotsc, x_{n+m}]$.
\end{proof}


\begin{definition}
  For any two affine sets~$X \subseteq \Aff^n$,~$Y \subseteq \Aff^m$ the affine set~$X \times Y \subseteq \Aff^{n+m}$ is the \emph{product}\index{product!of affine sets} of~$X$ and~$Y$.
\end{definition}


\begin{example}
  It holds that~$\Aff^n \times \Aff^m = \Aff^{n+m}$ as affine sets.
\end{example}


\begin{warning}
  \label{zariski finer than product topology}
  The Zariski topology on~$X \times Y$ is finer than the product topology (i.e.\ it has more open sets) and in general strictly so.
  
  To show that the Zariski topology on~$X \times Y$ is finer than the product topology it sufficies to consider the case~$X = \Aff^n$,~$Y = \Aff^m$ because both the Zariski topology and product topology on~$X \times Y$ are inherited from~$\Aff^n \times \Aff^m = \Aff^{n+m}$.
  It further sufficies to show that the sets~$U \times V$ for open subsets~$U \subseteq \Aff^n$,~$V \subseteq \Aff^m$ are open in the Zariski topology because these form a basis of the product topology.
  This holds because
  \[
      \Aff^{n+m} \setminus \, (U \times V)
    = \left( (\Aff^n \setminus U) \times \Aff^m \right)
      \cup
      \left( \Aff^n \times (\Aff^m \setminus V) \right)
  \]
  is Zariski closed by \cref{product of affine sets is an affine set}.
  
  To show that the Zariski toplogy on~$X \times Y$ is in general strictly finer than the product topology we consider the case~$X = Y = \Aff^1$.
  The diagonal
  \[
              \Delta
    =         \{(x,x) \suchthat x \in \Aff^1\}
    =         V(x_1 - x_2)
    \subseteq \Aff^2
  \]
  is then Zariski closed.
  But~$\Delta$ cannot be closed in the product topology because~$\Aff^1$ is not Hausdorff\footnote{Here we use the well-known fact from point set topology that a toplogical space~$X$ is Hausdorff if and only if the diagonal~$\Delta = \{(x,x) \suchthat x \in X\}$ is a closed subset of~$X \times X$.}, as it is an infinite set endowed with the cofinite topology.
\end{warning}


\begin{proposition}
  Let~$X, X_1, X_2, Y_1, Y_2$ be affine sets.
  \begin{enumerate}
    \item
      The projections~$\pi_i \colon X_1 \times X_2 \to X_i$ are morphisms of affine sets.
    \item
      A map~$f \colon X \to Y_1 \times Y_2$ given by~$f = (f_1, f_2)$ with~$f_i \colon X \to Y_i$ is a morphism of affine sets if and only if both~$f_1, f_2$ are morphisms of affine sets.
  \end{enumerate}
  This shows that the product of two affine sets is their categorical product in the category of affine sets.
  \begin{enumerate}[resume]
    \item
      If~$f \colon X \to X'$ and~$g \colon Y \to Y'$ are two morphisms of affine sets then the induced map~$f \times g \colon X \times Y \to X' \times Y'$ is again a morphism of affine sets.
    \qed
  \end{enumerate}
\end{proposition}


\begin{proposition}
  \label{coordinate ring of product of affine sets}
  For any two affine sets~$X,Y$ the map
  \[
            \coord(X) \tensor_k \coord(Y)
    \to     \coord(X \times Y) \,,
    \quad   f \tensor g
    \mapsto \bigl[ (x,y) \mapsto f(x)g(y) \bigr]
  \]
  is a well-defined natural isomorphism of~\dash{$k$}{algebras}.
  \qed
\end{proposition}









\section{Quasi-Affine Sets}





\subsection{Definition}


\begin{definition}
  If~$X$ is an affine set and~$X' \subseteq X$ is a Zariski open subset then~$X'$ is a \emph{{\qaffine} \textup(algebraic\textup) set}.
  The \emph{Zariski topology} on~$X'$ is the subspace topology induced by the Zariski topology on~$X$.
\end{definition}


\begin{remark}
  A subset~$X \subseteq \Aff^n$ is a {\qaffine} set if and only if it is of the form~$X = C \cap U$ for some Zariski closed subset~$C \subseteq \Aff^n$ and Zariski open subset~$U \subseteq \Aff^n$.
\end{remark}


\begin{corollary}
  Let~$X, X_1, \dotsc, X_n \subseteq \Aff^n$ be {\qaffine} sets.
  \begin{enumerate}
    \item
      Every Zariski open subsets of~$X$ is again a {\qaffine} set.
    \item
      Every Zariski closed subsets of~$X$ is again a {\qaffine} set.
    \item
      The finite intersection~$X_1 \cap \dotsb \cap X_n$ is again a {\qaffine} set.
  \end{enumerate}
\end{corollary}


\begin{proof}
  Let~$C, C_1, \dotsc, C_n \subseteq \Aff^n$ be Zariski closed and~$U, U_1, \dotsc, U_n \subseteq \Aff^n$ Zariski open such that~$X = C \cap U$ and~$X_i = C_i \cap U_i$ for every~$i$.
  \begin{enumerate}
    \item
      If~$V \subseteq X$ is Zariski open then there exists a Zariski open subset~$V' \subseteq \Aff^n$ with~$V = V' \cap X$.
      It then follows that
      \[
          V
        = V' \cap X
        = V' \cap C \cap U
        = C \cap (V' \cap U)
      \]
      with~$V' \cap U \subseteq \Aff^n$ being Zariski open.
    \item
      This can be shown in the same way as above.
    \item
      It follows that
      \[
          X_1 \cap \dotsb \cap X_n
        = (C_1 \cap \dotsb \cap C_n) \cap (U_1 \cap \dotsb \cap U_n)
      \]
      with~$C_1 \cap \dotsb \cap C_n \subseteq \Aff^n$ being Zariski closed and~$U_1 \cap \dotsb \cap U_n \subseteq \Aff^n$ being Zariski open.
    \qedhere
  \end{enumerate}
\end{proof}


\begin{example}
  \leavevmode
  \begin{enumerate}
    \item
      Every affine set is a {\qaffine} set.
    \item
      If~$X$ is an affine set then~$\Dopen_X(f)$ is a {\qaffine} set for every~$f \in \coord(X)$.
    \item
      It follows from the previous example with~$X = \Aff^{n^2}$ and~$f = \det$ that~$\GL_n(k) = \Dopen(\det)$ is a {\qaffine} set.
  \end{enumerate}
\end{example}





\subsection{Morphisms of Quasi-Affine Sets}


\begin{definition}
  \label{regular for quasiaffine}
  Let~$X$ be a {\qaffine} set and let~$f \colon X \to k$ be a function.
  \begin{enumerate}
    \item
      The function~$f$ is \emph{regular at~$x \in X$} if~$f$ is a rational function in some neighbourhood of~$x$, i.e.\ if there exist~$U \subseteq X$ open and~$g, h \in \coord(X)$ such that~$x \in U$,~$h(y) \neq 0$ for every~$y \in U$ and~$f(y) = g(y)/h(y)$ for all~$y \in U$.
    \item
      The function~$f$ is \emph{regular} if it is regular at every point~$x \in X$.
      The set of all rational functions~$X \to k$ is denoted by~$\rational(X)$.
  \end{enumerate}
\end{definition}


\begin{notation}
  If~$X$ is a set,~$U \subseteq X$ is a subset and~$f, f', g, g' \colon X \to k$ are functions with~$g(x), g'(x) \neq 0$ for every~$x \in U$ and~$f(x)/g(x) = f'(x)/g'(x)$ for all~$x \in U$ then we say that~$f/g \equiv f'/g'$~on~$U$.
\end{notation}


\begin{proposition}
  \label{regular on affine is polynomial}
  Let~$X$ be an affine set and let~$f \colon X \to k$ be a regular function in the sense of \cref{regular for quasiaffine}.
  The map~$f$ is then already a polynomial, and thus regular in the sense of \cref{regular for affine}.
  Thus both definitions agree for affine sets, and~$\rational(X) = \coord(X)$.
\end{proposition}


\begin{proof}
  There exists an open cover~$(U_i)_{i \in I}$ of~$X$ such that~$f$ is given on each~$U_i$ by a rational function~$f_i/g_i$ for suitable~$f_i, g_i \in \coord(X)$.
  We may assume that each~$U_i$ is of the form~$U_i = \Dopen(h_i)$ for some~$h_i \in \coord(X)$ as these sets form a basis for the Zariski topology on~$X$.
  
  It follows for all~$i,j \in I$ that~$f_i/g_i \equiv f_j/g_j$ on~$U_i \cap U_j$, and therefore that~$f_i g_j \equiv f_j g_i$ on~$U_i \cap U_j$.
  The set~$U_i \cap U_j$ is given by
  \[
    U_i \cap U_j = \Dopen(h_i) \cap \Dopen(h_j) = \Dopen(h_i h_j) \,.
  \]
  It follows that by multiplying the above equality with~$h_i h_j$ we arrive at the equality
  \begin{equation}
  \label{extended equality on intersections}
      h_i h_j f_i g_j
    = h_i h_j f_j g_i \,,
  \end{equation}
  which holds both on~$U_i \cap U_j = \Dopen(h_i hj)$ and on~$X \setminus (U_i \cap U_j) = \vset(h_i h_j)$, i.e.\ on the whole of~$X$.
  
  We may assume that~$h_i = g_i$:
  It holds that~$\Dopen(h_i) \subseteq \Dopen(g_i)$ because the rational function~$f_i/g_i$ is defined on~$U_i = \Dopen(h_i)$.
  It follows from \cref{containment of D} that~$h_i \in \rad{\genideal{g_i}}$ and therefore that~$h_i^n = a g_i$ for some~$a \in \coord(X)$,~$n \geq 0$.
  It follows that~$a(x) \neq 0$ forevery~$x \in U_i = \Dopen(h_i)$ and therefore 
  \[
            \frac{f_i}{g_i}
    \equiv  \frac{a f_i}{a g_i}
    \equiv   \frac{a f_i}{h_i^n}
  \]
  on~$U_i$.
  By replacing~$f_i$ with~$a f_i$ and both~$g_i$ and~$h_i$ with~$h_i^n$ the claim follows.
  Note that \Cref{extended equality on intersections} can now be rewritten as
  \begin{equation}
  \label{equality on intersections}
        f_i g_j \cdot g_i g_j
      = f_j g_i \cdot g_i g_j \,.
  \end{equation}
  So we may swap the indices in the term~$f_i g_j$ if the factor~$g_i g_j$ is present.
  
  The denominators~$g_i$,~$i \in I$ have no common zeros because
  \[
              \vset(g_i \suchthat i \in I)
    =         \bigcap_{i \in I} \vset(g_i)
    =         X \setminus \bigcup_{i \in I} \Dopen(g_i)
    =         X \setminus X
    =         \emptyset \,,
  \]
  The squares~$g_i^2$,~$i \in I$ do therefore also have no common zeroes (because~$g_i$ and~$g_i^2$ have the same zeroes).
  It follows from \hyperref[nullstellensatz 2]{Hilbert’s~Nullstellensatz} that there exists a linear combination
  \begin{equation}
  \label{unit as linear combination}
      1
    = \sum_{i \in I} a_i g_i^2
  \end{equation}
  for suitable~$a_i \in \coord(X)$.
  By using \Cref{equality on intersections} and \Cref{unit as linear combination} it follows that it holds on~$U_j$ that
  \[
            f
    \equiv  \frac{f_j}{g_j}
    \equiv  \frac{f_j}{g_j} \sum_{i \in I} a_i g_i^2
    \equiv  \sum_{i \in I} \frac{a_i f_j g_i^2}{g_j}
    \equiv  \sum_{i \in I} \frac{a_i f_j g_i^2 g_j}{g_j^2}
    \equiv  \sum_{i \in I} \frac{a_i f_i g_j^2 g_i}{g_j^2}
    \equiv  \sum_{i \in I} a_i f_i g_i \,.
  \]
  This shows that~$f \equiv \sum_{i \in I} a_i f_i g_i$ on~$U_j$ for every~$j \in I$ and therefore~$f = \sum_{i \in I} a_i f_i g_i$.
\end{proof}


% TODO: Understand this proof better.


\begin{definition}
  \label{regular of qaffine}
  Let~$X,Y$ be {\qaffine} sets.
  \begin{enumerate}[resume]
    \item
      A map~$f \colon X \to \Aff^m$ is \emph{regular} if it is regular in each coordinate.
    \item
      A map~$f \colon X \to Y$ is \emph{regular} if it is the restriction of a regular map~$X \to \Aff^m$.
  \end{enumerate}
  A map~$X \to Y$ is a \emph{morphism} of {\qaffine} sets if it is regular.
\end{definition}


\begin{remark}
  \Cref{regular on affine is polynomial} shows that \cref{regular of qaffine} agrees with \cref{regular for affine} for affine sets.
\end{remark}


\begin{lemma}
  Let~$X \subseteq X' \subseteq \Aff^n$ be a {\qaffine} set with~$X'$ an affine set.
  \begin{enumerate}
    \item
      The sets~$\Dopen_X(f)$ with~$f \in \coord(X')$ are a basis of the Zariski topology on~$X$.
    \item
      The sets~$\Dopen_X(f)$ with~$f \in \rational(X)$ are open in~$X$, and also form a basis of the Zariski topology on~$X$.
  \end{enumerate}
\end{lemma}


\begin{proof}
  \leavevmode
  \begin{enumerate}
    \item
      This follows from the fact that the sets~$\Dopen_{X'}(f)$ with~$f \in \coord(X')$ are a basis for the Zariski topology on~$X'$ and that~$\Dopen_X(f) = \Dopen_{X'}(f) \cap X$ for every~$f \in \coord(X)$.
    \item
      It remains to show that every~$f \in \rational(X)$ is continuous.
      If~$f$ is a globally defined rational function given by~$f = g/h$ for~$f, g \in \Aff(X')$ then~$\Dopen_X(f) = \Dopen_X(g)$ is open.
      
      If more generally~$f$ is regular then there exists an open cover~$(U_i)_{i \in I}$ of~$X$ such that~$\restrict{f}{U_i}$ is rational for every~$i \in I$.
      It then follows by the above that~$\restrict{f}{U_i}$ is continuous for every~$i \in I$ (where we use that~$U_i$ is again {\qaffine} and~$\restrict{f}{U_i}$ is again regular) and therefore that~$f$ is continuous (because continuity is a local property).
    \qedhere
  \end{enumerate}
\end{proof}


\begin{corollary}
  Let~$X,Y$ be {\qaffine} sets and let~$f \colon X \to Y$ be a morphism of {\qaffine} sets.
\ \begin{enumerate}
    \item
      It holds for every~$\varphi \in \rational(Y)$ that~$f^{-1}(\Dopen_Y(\varphi)) = \Dopen_X(\varphi \circ f)$.
    \item
      The map~$f$ is continuous with respect to the Zariski topologies on~$X,Y$.
    \qed
  \end{enumerate}
\end{corollary}


\begin{lemma}
  Let~$X, Y, Z$ be {\qaffine} sets.
  \begin{enumerate}
    \item
      The identity map~$\id_X \colon X \to X$ is a morphism.
    \item
      For every two morphisms~$f \colon X \to Y$ and~$g \colon Y \to Z$ their composition~$g \circ f \colon X \to Z$ is again a morphism.
  \end{enumerate}
\end{lemma}


\begin{proof}
  \leavevmode
  \begin{enumerate}[start=2]
    \item
      We need to show that~$g \circ f$ is regular at every point~$x \in X$.
      It follows from the regularity of~$g$ that there exists an open neighbourhood~$V \subseteq Y$ of~$f(x)$ on which~$g$ is given by a rational function~$g_1/g_2$.
      It follows from the continuity of~$f$ that there exists an open neighbourhood~$U$ of~$x$ with~$f(U) \subseteq V$.
      By using the regularity of~$f$ and shrinkening~$V$ if necessary we may assume that~$f$ is given by a rational function~$f_1/f_2$ on~$U$.
      It then follows that
      \[
          (g \circ f)(x)
        = \frac{ g_1\left( \frac{f_1(x)}{f_2(x)} \right) }{ g_2\left( \frac{f_1(x)}{f_2(x)} \right) }
      \]
      for every~$x \in U$, which shows that~$g \circ f$ is given by a rational function on~$U$.
      (Recall that the composition of rational functions is again rational.)
    \qedhere
  \end{enumerate}
\end{proof}



\begin{lemma}
  Let~$X, Y, Z$ be {\qaffine} sets
  \begin{enumerate}
    \item
      If~$f \colon X \to Y$ is a morphism of {\qaffine} sets then the map
      \[
                  f^*
        \colon    \rational(Y)
        \to       \rational(X),
        \quad     \varphi
        \mapsto   \varphi \circ f
      \]
      is a homomorphism of~\dash{$k$}{algebras}.
    \item
      It holds that~$\id_X^* = \id_{\rational(X)}$.
    \item
      If~$f \colon X \to Y$,~$g \colon Y \to Z$ are morphisms of {\qaffine} sets then~$(g \circ f)^* = f^* \circ g^*$.
    \qed
  \end{enumerate}
\end{lemma}




\subsection{Products of Quasi-Affine Sets}


\begin{lemma}
  If~$X \subseteq \Aff^n$ and~$Y \subseteq \Aff^m$ are {\qaffine} sets then the set~$X \times Y \subseteq \Aff^{n+m}$ is again {\qaffine}.
\end{lemma}


\begin{proof}
  Let~$X' \subseteq \Aff^n$ and~$Y' \subseteq \Aff^m$ be affine sets such that~$X \subseteq X'$ and~$Y \subseteq Y'$ are open subsets.
  Then~$X \times Y$ is an open subset of the affine set~$X' \times Y' \subseteq \Aff^{n+m}$ because the Zariski topology on~$X' \times Y'$ is finer than the product topology.
\end{proof}



\begin{definition}
  For {\qaffine} sets~$X \subseteq \Aff^n$,~$Y \subseteq \Aff^m$ their \emph{product} is the {\qaffine} set~$X \times Y \subseteq \Aff^{n+m}$.
\end{definition}


\begin{fluff}
  Note that for affine sets~$X, Y$ their product as affine sets is the same as their product of {\qaffine} sets.
  We therefore do not need to specify which kind of product of we are talking about when dealing with affine sets.
\end{fluff}


\begin{lemma}
  Let~$X, X_1, X_2, X_2, Y_1, Y_2$ be {\qaffine} sets.
  \begin{enumerate}
    \item
      The projections~$\pi_i \colon X_1 \times X_2 \to X_i$ are morphisms of {\qaffine} sets.
    \item
      A map~$f \colon X \to Y_1 \times Y_2$ given by~$f = (f_1, f_2)$ with~$f_i \colon X \to Y_i$ is a morphism of {\qaffine} sets if and only if both~$f_1, f_2$ are morphisms of {\qaffine} sets.
  \end{enumerate}
  This shows that the product of two {\qaffine} sets is their categorical product in the category of {\qaffine} sets.
  \begin{enumerate}[resume]
    \item
      If~$f \colon X \to X'$ and~$g \colon Y \to Y'$ are to morphisms of {\qaffine} sets then the induced map~$f \times g \colon X \times Y \to X' \times Y'$ is again a morphism of {\qaffine} sets.
    \qed
  \end{enumerate}
\end{lemma}





\subsection{Affine Varieties}


\begin{definition}
  An \emph{affine~variety} is a {\qaffine} set which is isomorphic to an affine set (as a {\qaffine} set).
  A \emph{{\qaffine}~variety} is just a {\qaffine} set.% make footnote not nice the line break
  \footnote{We will later on extend this notions to projective and {\qprojective} sets.}
\end{definition}


\begin{notation}
  \leavevmode
  \begin{enumerate}
    \item
      We often just say than a {\qaffine} variety is \emph{affine} to mean that it is an quasi-affine variety.
      Note that this does not necessarily mean that it is affine as a set!
    \item
      For an affine variety~$X$ we often write~$\coord(X) \defined \rational(X)$.
      \Cref{regular on affine is polynomial} shows that this is well-defined when~$X$ is an affine set.
  \end{enumerate}
\end{notation}


\begin{remark}
  Many authors additionally require affine varieties to be irreducible.
  The author tries to avoid this restriction.
\end{remark}


\begin{example}
  \leavevmode
  \begin{enumerate}
    \item
      Every affine set is an affine variety.
    \item
      Let~$X$ be an affine variety and let~$f \in \coord(X)$.
      Then the open subset~$\Dopen_X(f)$ is again an affine variety:
      
      We may assume that~$X \subseteq \Aff^n$ is an affine set.
      For the set
      \[
          Y
        = \left\{
                (x, t)
            \in \Aff^n \times \Aff^1
            =   \Aff^{n+1}
          \suchthat*
            x \in X,
            x t = 1
          \right\}
      \]
      the map
      \[
                \varphi
        \colon  \Dopen_X(f)
        \to     Y,
        \quad   x
        \mapsto \left( x, \frac{1}{f(x)} \right)
      \]
      is a bijection.
      If~$X$ is cut out by some ideal~$I \idealleq k[x_1, \dotsc, x_n]$ then~$Y$ is cut out by the ideal~$I$ together with the polynomial~$f x_{n+1} - 1$, which shows that~$Y$ is again an affine set.
      The map~$\varphi$ is rational in each coordinate and therefore a morphism.
      The inverse of~$\varphi$ is given by projection onto the first~$n$-th coordinates, which is also a morphism.
      This shows that~$\varphi$ is an isomorphism, which shows that claim.
      
      It follows in particular that the isomorphism~$\varphi$ of affine varieties induces an isomorphism of \dash{$k$}{algebras}~$\varphi^* \colon \coord(Y) \to \coord(\Dopen_X(f))$.
      It follows that
      \begin{align*}
                \coord(\Dopen_X(f))
        &\cong  \coord(Y) \\
        &=      k[x_1, \dotsc, x_n, x_{n+1}]/(I, f x_{n+1} - 1) \\
        &\cong  ( k[x_1, \dotsc, x_n]/I )[x_{n+1}]/(f x_{n+1} - 1) \\
        &=      \coord(X)[x_{n+1}]/(f x_{n+1} - 1) \\
        &\cong  \coord(X)[f^{-1}]
      \end{align*}
      is the localization of~$\coord(X)$ at~$f$.
      This shows that every regular function on~$\Dopen_X(f)$ is of the form~$g/f^n$ for some~$g \in \coord(X)$ and~$n \geq 0$.
    \item
      As an instance of the previous example we find that~$\GL_n(k) = \Dopen(\det) \subseteq \Aff^{n^2}$ is an affine variety with
      \[
          \coord(\GL_n(k))
        = \coord\left( \Aff^{n^2} \right)\left[ {\det}^{-1} \right]
        = k\left[ x_{11}, \dotsc, x_{nn}, {\det}^{-1} \right] \,.
      \]
      As an example we have for~$n = 2$ that
      \[
          \coord(\GL_2(k))
        = k\left[a, b, c, d, \frac{1}{ad-bc}\right] \,.
      \]
  \end{enumerate}
\end{example}


\begin{lemma}
  Let~$X, X_1, X_2$ be affine varieties.
  \begin{enumerate}
    \item
      Every closed subset~$C \subseteq X$ is again an affine variety.
    \item
      If~$X_1$ and~$X_2$ are both affine 
  \end{enumerate}
\end{lemma}


\begin{lemma}
  \label{coordinate ring of product of qaffine}
  For affine varieties~$X,Y$ the map
  \[
            \coord(X) \otimes_k \coord(Y)
    \to     \coord(X \times Y) \,,
    \quad   f \otimes g
    \mapsto [(x, y) \mapsto f(x) g(y)]
  \]
  is a \dash{well}{defined} natural isomorphism of~\dash{$k$}{algebras}.
\end{lemma}


\begin{proof}
  To see that the proposed map~$\Phi = \Phi_{X,Y}$ is \dash{well}{defined} let~$f \in \coord(X)$ and~$g \in \coord(Y)$.
  We need to show that~$\Phi(f \otimes g)$ is regular at every point~$(x,y) \in X \times Y$.
  There exist a neighbourhood~$U \subseteq X$ of~$x$ on which~$f$ is given by a rational function~$f_1/f_2$, and similarly a neighbourhood~$V \subseteq Y$ of~$y$ on which~$g$ is given by a rational function~$g_1/g_2$.
  It follows that~$U \times V$ is an open neighbourhood of~$(x,y)$ in~$X \times Y$ because the Zariski topology on~$X \times Y$ is finer than the product topology.
  In this neighbourhood the map~$\Phi(f \otimes g)$ is given by the rational function~$(f_1 g_1) / (f_2 g_2)$.
  This shows that~$\Phi(f \otimes g)$ is regular at~$(x,y)$.
  
  To show that naturality of~$\Phi$ let~$X', Y'$ be another pair of affine varieties and consider a pair of morphisms~$\varphi \colon X \to X'$,~$\psi \colon Y \to Y'$.
  For the naturialty of~$\Phi$ we need the diagram
  \begin{equation}
  \label{naturality for product of coordinate rings}
    \begin{tikzcd}[sep = large]
        \coord(X) \otimes_k \coord(Y)
        \arrow{r}[above]{\Phi_{X,Y}}
      & \coord(X \times Y)
      \\
        \coord(X') \otimes_k \coord(Y')
        \arrow{u}[left]{\varphi^* \otimes \psi^*}
        \arrow{r}[above]{\Phi_{X',Y'}}
      & \coord(X' \times Y')
        \arrow{u}[right]{(\varphi \times \psi)^*}
    \end{tikzcd}
  \end{equation}
  to commutes.
  This holds because
  \begin{align*}
        \Phi_{X,Y}( (\varphi^* \times \psi^*)(f \otimes g) )(x,y)
    &=  \Phi_{X,Y}( \varphi^*(f) \otimes \psi^*(g) )(x,y)         \\
    &=  \varphi^*(f)(x) \psi^*(g)(y)                              \\
    &=  (f \circ \varphi)(x) (g \circ \psi)(y)                    \\
    &=  f(\varphi(x)) g(\psi(y))                                  \\
    &=  \Phi_{X',Y'}(f \otimes g)(\varphi(x), \psi(y))            \\
    &=  \Phi_{X',Y'}(f \otimes g)( (\varphi \times \psi)(x,y) )   \\
    &=  (\varphi \times \psi)^*( \Phi_{X',Y'}(f \otimes g) )(x,y)
  \end{align*}
  for all~$(x,y) \in X \times Y$, and therefore
  \[
      \Phi_{X,Y}( (\varphi^* \otimes \psi^*)(f \otimes g) )
    = (\varphi \times \psi)^*( \Phi_{X',Y'}(f \otimes g ) )
  \]
  for every simple tensor~$f \otimes g \in \coord(X') \otimes \coord(Y')$, and thus overall
  \[
      \Phi_{X,Y} \circ (\varphi^* \otimes \psi^*)
    = (\varphi \times \psi)^* \circ \Phi_{X',Y'} \,.
  \]

  
  To show that~$\Phi$ is an isomorphism let~$X' \subseteq \Aff^n$ and~$Y' \subseteq \Aff^m$ be affine sets for which there exists isomorphisms~$\varphi \colon X \to X'$ and~$\psi \colon Y \to Y'$.
  Then in the resulting commutative diagram~\eqref{naturality for product of coordinate rings} the vertical arrows are both isomorphisms and the lower horizontial map~$\Phi_{X', Y'}$ is an isomorphism by~\Cref{regular functions on product of affine sets}.
  It follows that the upper horizontal arrow, which can be expressed as
  \begin{equation}
  \label{isomorphism for affine varieties}
      \Phi_{X,Y}
    = (\varphi \times \psi)^* \circ \Phi_{X',Y'} \circ ( \varphi^* \otimes \psi^* )^{-1} \,,
  \end{equation}
  is also an isomomorphism.
\end{proof}


\begin{remark}
  That~$\Phi_{X,Y}$ is well-defined for affine varieties~$X,Y$ also follows from~\Cref{isomorphism for affine varieties}.
  The above argumentation actually shows that~$\Phi_{X,Y}$ is well-defined whenever~$X,Y$ are {\qaffine} varietes, and it is explained in~\cite{MO267198} that~$\Phi_{X,Y}$ is then again an isomorphism.
  (One uses that that the isomorphism holds for affine varieties, and then uses that {\qaffine} varities are \enquote{locally affine} and that regularity is a local condition.)
% TODO: Add a proper explanation how and why quasi-affine varietes are locally affine.
\end{remark}


% TODO: Calculate some induced morphisms used for algebraic groups


\begin{fluff}
  The notion of {\qaffine} sets and the above generalization of affine sets to affine varieties were originally not given in the lecture.
  {\Qaffine} sets were only introduced much later on together with projective and {\qprojective} sets.
  We have choosen to include these concept earlier on so that~$\GL_n(k)$ becomes an affine variety.
  This places the upcoming discussion of affine algebraic groups on a more formally solid foundation.
  
  The idea of considering~{\qaffine} sets and the regular maps between them was inspired by~\cite[2.2]{frankeAlg1}.
  The proof of \cref{regular on affine is polynomial} is taken from~\cite[Lemma~3.10]{milneAG}.
\end{fluff}








% \section{(Quasi-)Projective Sets}


\begin{definition}
  The \emph{projective~\dash{$n$}{space}} over~$k$ is~$\Proj^n \defined \Proj^n_k \defined ( k^{n+1} \setminus \{0\} )/{\sim}$ where~$\sim$ is the equivalence relation defined by
  \[
          x \sim y
    \iff  \exists\, \lambda \in k^\times : x = \lambda y
    \iff  \gen{x}_k = \gen{y}_k \,.
  \]
  The equivalence class of~$(x_0, \dotsc, x_n) \in k^{n+1} \setminus \{0\}$ in~$\Proj^n$ is denoted by~$[x_0 \hd \dotsb \hd x_n]$, and the tupel~$(x_0, \dotsc, x_n)$ are \emph{homogeneous coordinates of~$[x_0 \hd \dotsb \hd x_n]$}.
\end{definition}


\begin{lemma}
  \label{characterization of homogeneous polynomials}
  For a polynomial~$f \in k[x_0, \dotsc, x_n]$ and a degree~$d \geq 0$ the following conditions are equivalent:
  \begin{enumerate}
    \item
      $f$ contains only monomials of degree~$d$.
    \item
      $f(\lambda x_0, \dotsc, \lambda x_n) = \lambda^d f(x_0, \dotsc, x_n)$ for every~$\lambda \in k$.
    \item
      $f(\lambda x) = \lambda^d f(x)$ for all~$\lambda \in k$ and~$x \in k^{n+1}$.
  \end{enumerate}
\end{lemma}


\begin{definition}
  A polynomial~$f \in k[x_0, \dotsc, x_n]$ which satisfies one (and thus all) of the conditions from \cref{characterization of homogeneous polynomials} is \emph{homogeneous of degree~$d$}.
  The set of homogeneous polynomials~$f \in k[x_0, \dotsc, x_n]$ of degree~$d$ is denoted by~$k[x_0, \dotsc, x_n]^d$.
\end{definition}


\begin{lemma}
  Let~$f, g\in k[x_0, \dotsc, x_n]$.
  \begin{enumerate}
    \item
      If~$f,g$ are homogeneous of degree~$d \geq 0$ then the linear combination~$\lambda f + \mu g$ is again homogeneous of degree~$d$ for all~$\lambda, \mu \in k$.
    \item
      If~$f$ is homogeneous of degree~$d_1$ and~$g$ is homogeneous of degree~$d_2$ then~$f \cdot g$ is homogeneous of degree~$d_1 + d_2$.
    \item
      It holds that~$k[x_0, \dotsc, x_n] = \bigoplus_{d \geq 0} k[x_0, \dotsc, x_n]_d$.
  \end{enumerate}
  Together this shows that~$k[x_0, \dotsc, x_n]$ is an~\dash{($\mathbb{N}$}{)graded}~\dash{$k$}{algebra} with homegenous components~$k[x_0, \dotsc, x_n]^d$.
\end{lemma}


\begin{definition}
  For~$f \in k[x_0, \dotsc, x_n]$ the unique polynomials~$f_d \in k[x_0, \dotsc, x_n]^d$ for~$d \in \Natural$ with~$f = \sum_{d \geq 0} f_d$ are the \emph{homogeneous parts of~$f$}.
\end{definition}


\begin{lemma}
  \label{characterization of homogeneous ideals}
  For an ideal~$I \idealleq k[x_0, \dotsc, x_n]$ the following conditions are equivalent:
  \begin{enumerate}
    \item
      $I$ is generated by homogeneous elements.
    \item
      $I = \bigoplus_{d \geq 0} I \cap k[x_0, \dotsc, x_n]^d$.
    \item
      $I = \bigoplus_{d \geq 0} I_d$ for some~\dash{$k$}{linear} subspaces~$I_d \moduleleq k[x_0, \dotsc, x_n]^d$.
  \end{enumerate}
\end{lemma}


\begin{definition}
  An ideal~$I \idealleq k[x_0, \dotsc, x_n]$ is \emph{homogeneous} if it satisfies one (and thus all) of the conditions in \cref{characterization of homogeneous ideals}.
\end{definition}


\begin{fluff}
  For a polynomial~$f \in k[x_0, \dotsc, x_n]$ and a point~$[x] \in \Proj^n$ the value~$f([x])$ is not well-defined, but if~$f$ is homogeneous then the condition~$f(x) = 0$ is well-defined.
  That allows us to define vanishing sets in projective space.
\end{fluff}


\begin{definition}
  \leavevmode
  \begin{enumerate}
    \item
      Let~$S \subseteq k[x_0, \dotsc, x_n]$ consist of homogeneous polynomials.
      Then the set
      \[
                  \vset_{\Proj}(S)
        \defined  \{
                    [x] \in \Proj^n
                  \suchthat
                    \text{$f(x) = 0$ for every~$f \in S$}
                  \}
      \]
      is the \emph{vanishing locus} of~$S$ or \emph{projective set} given by~$S$.
    \item
      For any homogeneous ideal~$I \idealleq k[x_0, \dotsc, x_n]$ the set
      \begin{align*}
                    \vset_{\Proj}(I)
        \defined{}& \{
                      [x] \in \Proj^n
                    \suchthat
                      \text{$f(x) = 0$ for every homogeneous~$f \in I$}
                    \} \\
        ={}&        \{
                      [x] \in \Proj^n
                    \suchthat
                      \text{$f(x) = 0$ for every~$f \in I$}
                    \}
      \end{align*}
      is the \emph{vanishing locus} of~$I$ or \emph{projective set} given by~$I$.
  \end{enumerate}
\end{definition}





\chapter{Affine Algebraic Groups}


\section{Definition and First Examples}


\begin{definition}
  An \emph{affine algebraic group} is an affine variety~$G$ together with a group structure such that both the multiplication map~$G \times G \to G$,~$(g_1, g_2) \mapsto g_1 g_2$ and inversion map~$G \to G$,~$g \mapsto g^{-1}$ are morphisms of affine varieties.
  
  If~$G, H$ are affine algebraic groups then a map~$f \colon G \to H$ is a \emph{homomorphism of affine algebraic groups} if it is both a morphism of affine varieties and a group homomorphism.
\end{definition}


\begin{example}
  The \emph{additive groups}~$\Gadd \defined (\Aff^1, +)$ is an affine algebraic group since the addition
  \[
            {+}
    \colon  \Aff^1 \times \Aff^1
    =       \Aff^2
    \to     \Aff^1 \,,
    \quad   (x,y)
    \mapsto x+y
  \]
  and inversion
  \[
            {-}
    \colon  \Aff^1
    \to     \Aff^1 \,,
    \quad   x
    \mapsto -x
  \]
  are regular.
\end{example}


\begin{example}
  The \emph{general linear group}~$\GL_n(k)$ together with the usual matrix multiplication is an affine algebraic group.
  We have seen in \cref{principal open sets of affine are again affine} that~$\GL_n(k)$ is an affine variety, the multiplication map~$\GL_n(k) \times \GL_n(k) \to \GL_n(k)$ is polynomial and therefore regular, and the inversion map~$\GL_n(k) \to \GL_n(k)$ is a rational function (because the entries of~$A^{-1}$ are rational functions in the entries of~$A$ by Cramer’s rule) and therefore also regular.
  
  It follows for~$n = 1$ that the \emph{multiplicative group}~$\Gmult \defined \GL_1(k) = k^\times$ is an affine algebraic group.
\end{example}


\begin{lemma}
  Every Zariski closed subgroup of an affin algebraic group is again an affine algebraic group.
\end{lemma}


\begin{proof}
  Let~$G$ be an affine algebraic group and let~$H \groupleq G$ be a Zariski closed subgroup.
  It follows from \cref{closed of affine is again affine} that~$H$ is again an affine variety, and the multiplication~$H \times H \to H$ and inversion~$H \to H$ are morphisms because they are restrictions of the multiplication~$G \times G \to G$ and inversion~$G \to G$.
\end{proof}


\begin{corollary}
  \label{closed subgroups of affine algebraic group}
  Every Zariski closed subgroup of~$\GL_n(k)$ is an affine algebraic group.
  \qed
\end{corollary}


\begin{example}
  It follows from \cref{closed subgroups of affine algebraic group} that the following are affine algebraic groups:
  \begin{enumerate}
    \item
      The \emph{special linear group}~$\SL_n(k) = \{ A \in \GL_n(k) \suchthat \det A = 1 \}$.
    \item
      The \emph{orthogonal groups}~$\Orth_n(k) = \{ A \in \GL_n(k) \suchthat A^T A = I \}$.
    \item
      The \emph{special orthogonal group}~$\SOrth_n(k) = \SL_n(k) \cap \Orth_n(k)$.
    \item
      The \emph{symplectic group}~$\Symp_{2n}(k) = \{ A \in \GL_n(k) \suchthat A^T J A = J \}$ where~$
          J
        = \begin{psmallmatrix}
             0    & I_n \\
            -I_n  & 0
          \end{psmallmatrix}
      $.
    \item
      Every finite group, when regarded as an affine set and thus an affine variety, is an affine algebraic group.
      The multiplication map~$G \times G \to G$ and inversion map~$G \to G$ are morphisms of afffine varieties because both~$G$ and~$G \times G$ are finite, and therefore all maps~$G \times G \to G$ and~$G \to G$ are morphisms.
    \item
      The \emph{group of diagonal matrices}~$
          \Diag_n(k)
        = \left\{
            \begin{psmallmatrix}
              * &        &    \\
                & \ddots &    \\
                &        & *
            \end{psmallmatrix}
            \in \GL_n(k)
          \right\}
      $.
    \item
      The \emph{group of upper triangular matrices}~$
          \Triag_n(k)
        = \left\{
            \begin{psmallmatrix}
              * & \cdots & *      \\
                & \ddots & \vdots \\
                &        & *
            \end{psmallmatrix}
            \in \GL_n(k)
          \right\} \,.
      $
    \item
      The \emph{group of unipotent matrices}~$
          \Uni_n(k)
        = \left\{
            \begin{psmallmatrix}
              1 & \cdots & *      \\
                & \ddots & \vdots \\
                &        & 1
            \end{psmallmatrix}
            \in \GL_n(k)
          \right\} \,.
      $
  \end{enumerate}
\end{example}


% For embeddings: Motivate by G_a.


\begin{remark}
  One may rephrase the definition of an affine algebraic groups as saying that~$G$ is an affine variety together with morphisms
  \[
            m
    \colon  G \times G
    \to     G
    \quad\text{and}\quad
            i
    \colon  G
    \to     G
  \]
  and an element~$e \in G$ such that the following diagrams commute:
  \[
    \renewcommand{\arraystretch}{1.5}
    \renewcommand{\arraycolsep}{3.3pt}
    \begin{array}{ccc}
        \begin{tikzcd}[column sep = 30pt, row sep = 35pt]
            G \times G \times G
            \arrow{r}[above]{m \times \id}
            \arrow{d}[left]{\id \times m}
          & G \times G
            \arrow{d}[right]{m}
          \\
            G \times G
            \arrow{r}[below]{m}
          & G
        \end{tikzcd}
      &
        \begin{tikzcd}[column sep = 10pt, row sep = 9pt]
            {}
          & G \times G
            \arrow{dd}[right]{m}
          & {}
          \\
            G \times \{e\}
            \arrow[hook]{ur}
            \arrow[two heads]{dr}[below left]{\pr_1}
          & {}
          & \{e\} \times G
            \arrow[hook']{ul}
            \arrow[two heads]{dl}[below right]{\pr_2}
          \\
            {}
          & G
          & {}
        \end{tikzcd}
      &
        \begin{tikzcd}[column sep = 3pt, row sep = 9pt]
            G
            \arrow{rr}[above]{(i,\id)}
            \arrow{dd}[left]{(\id,i)}
            \arrow{dr}
          & {}
          & G \times G
            \arrow{dd}[right]{m}
          \\
            {}
          & \{e\}
            \arrow[hook]{dr}
          & {}
          \\
            G \times G
            \arrow{rr}[below]{m}
          & {}
          & G
        \end{tikzcd}
    \\
        \text{associativity}
      & \text{neutral element}
      & \text{inverse}
    \end{array}
  \]

  If more generally~$\mc{C}$ is any category with finite products, including a terminal object~$\ast$, then a \emph{group object} in~$\mc{C}$ is an object~$G \in \mc{C}$ together with morphisms
  \[
    m \colon G \times G \to G \,,
    \qquad
    i \colon G \to G \,,
    \qquad
    e \colon \ast \to G
  \]
  such that the following diagrams commute:
  \[
    \renewcommand{\arraystretch}{1.5}
    \renewcommand{\arraycolsep}{4pt}
    \begin{array}{ccc}
        \begin{tikzcd}[row sep = 36pt, column sep = 30pt]
            G \times G \times G
            \arrow{r}[above]{m \times \id}
            \arrow{d}[left]{\id \times m}
          & G \times G
            \arrow{d}[right]{m}
          \\
            G \times G
            \arrow{r}[below]{m}
          & G
        \end{tikzcd}
      &
        \begin{tikzcd}[row sep = 10pt, column sep = 15pt]
            {}
          & G \times G
            \arrow{dd}[right]{m}
          & {}
          \\
            G \times *
            \arrow{ur}[above left]{\id \times e}
            \arrow{dr}[below left]{\pr_1}
          & {}
          & * \times G
            \arrow{ul}[above right]{e \times \id}
            \arrow{dl}[below right]{\pr_2}
          \\
            {}
          & G
          & {}
        \end{tikzcd}
      &
        \begin{tikzcd}[row sep = 12pt, column sep = 10pt]
            G
            \arrow{rr}[above]{(i,\id)}
            \arrow{dd}[left]{(\id,i)}
            \arrow{dr}
          & {}
          & G \times G
            \arrow{dd}[right]{m}
          \\
            {}
          & *
            \arrow{dr}{e}
          & {}
          \\
            G \times G
            \arrow{rr}[below]{m}
          & {}
          & G
        \end{tikzcd}
    \\
        \text{associativity}
      & \text{neutral \enquote{element}}
      & \text{inverse}
    \end{array}
  \]
\end{remark}


\begin{example}
  \leavevmode
  \begin{enumerate}
    \item
      The group objects in the category of sets are just groups.
    \item
      The group objects in the category of topological spaces are topological groups.
    \item
      The group objects in the category of smooth real manifolds are real Lie groups.
    \item
      The group objects in the category of affine varieties are affine algebraic groups.
  \end{enumerate}
\end{example}


\begin{warning}
  An affine algebraic group~$G$ is in general not a topological group.
  While the multiplication~$m \colon G \times G \to G$ is a morphism of affine varieties and therefore continuous, it is so with respect to the Zariski topology on~$G \times G$.
  For~$G$ to be a topological group we would need~$m$ to be continuous with respect to the product topololgy of~$G \times G$, which is in general coarser than the Zariski topology (see \cref{zariski finer than product topology}).
\end{warning}


\begin{definition}
  For an affine algebraic group~$G$ the connected component of the identity~$1 \in G$ is denoted by~$G^0$.
\end{definition}


\begin{proposition}
  Let~$G$ be a linear algebraic group.
  \begin{enumerate}
    \item
      The connected components of~$G$ coincide with its irreducible components.
    \item
      The connected/irreducible component~$G^0$ is a normal subgroup of~$G$.
    \item
      The connected/irreducible components of~$G$ are the cosets of~$G^0$.
    \item
      The group~$G^0$ has finite index in~$G$.
  \end{enumerate}
\end{proposition}


\begin{proof}
  \leavevmode
  \begin{enumerate}
    \item
      It follows from \cref{quasi-affine have only finitely many irreducible} that~$G$ has only finitely many irreducible components~$C_1, \dotsc, C_n$.
      It holds that~$C_1 \nsubseteq C_2 \cup \dotsb \cup C_n$ because it would otherwise follows from the irreducibility of~$C_1$ that~$C_1 \subseteq C_i$ for some~$i \geq 2$, which would contradict~$C_1, \dotsc, C_n$ being the irreducible components of~$G$.
      Let~$g \in C_1$ with~$x \notin C_2 \cup \dotsc \cup C_n$.
      
      The element~$x$ is contained in precisely one irreducible component.
      For every~$g \in G$ the map
      \[
                G
        \to     G \,,
        \quad   h
        \mapsto g h x^{-1}
      \]
      is regular and therefore a homeomorphism, and maps~$x$ to~$g$.
      It follows that every~$g \in G$ is contained in precisely one irreducible component.
      The irreducible components of~$G$ are therefore disjoint.
      
      Each irreducible component~$C$ is closed.
      The complement of~$C$ is also closed because it is the union of all other irreducible components, and therefore a finite union of closed sets.
      This shows that each irreducible component~$C$ of~$G$ is actually clopen.
      
      It follows that each connected component of~$G$ is contained in an irreducible component.
      It also holds that every irreducible component is contained in a connected component because irreducible topological spaces are connected.
      It follows that the connected and irreducible components coincide.
    \item
      It holds that~$1 \in G^0$ by the definition of~$G^0$.
      For every~$g \in G^0$ the left multiplication
      \[
                \lambda_g
        \colon  G
        \to     G \,,
        \quad   h
        \mapsto gh
      \]
      is an isomorphism of varieties and thus a homeomorphism.
      It therefore maps the component~$G^0$ of the identity~$1$ onto the component of~$\lambda_g(1) = g$.
      It follows from~$g$ being contained in~$G^0$ that this component is~$G^0$ and therefore that~$g G^0 = \lambda_g( G^0 ) = G^0$.
      
      It follows similarly that~$(G^0)^{-1}$ is the component of~$G$ which contains~$1^{-1} = 1$ and therefore that~$(G^0)^{-1} = G^0$.
      
      Together this shows that~$G^0$ is a subgroups of~$G$.
      To show that~$G^0$ is normal in~$G$ let~$g \in G$.
      The conjugation map
      \[
                c_g
        \colon  G
        \to     G \,,
        \quad   h
        \mapsto g h g^{-1}
      \]
      is regular and therefore a homeomorphism.
      It follows that~$g G^0 g^{-1} = c_g(G^0)$ is the component containing~$c_g(1) = 1$ and therefore that~$g G^0 g^{-1} = G^0$.
    \item
      We find in the above notation that the component of~$g \in G$ is given by~$\lambda_g( G^0 ) = g G^0$.
    \item
      The index~$[G : G^0]$ is the number of components of~$G$.
    \qedhere
  \end{enumerate}
\end{proof}





\section{Hopf Algebra Structure on the Coordinate Ring}


\begin{fluff}
  If~$X$ is an affine variety then the coordinate ring~$\coord(X)$ is a~\dash{$k$}{algebra}, with addition and multiplication coming from~$k$.
  In this section we will see that if~$G$ is an affine algebraic group then the additional group structure of~$G$ gives the coordinate ring~$\coord(G)$ the additional structure of a Hopf algebra. 
\end{fluff}


\begin{fluff}
  A~\dash{$k$}{algebra}~$A$ may be defined as a~{\kvs}~$A$ together with two~\dash{$k$}{linear} maps
  \[
    m \colon A \tensor A \to A
    \quad\text{and}\quad
    u \colon k \to A
  \]
  such that the following diagrams commute:
  \[
    \begin{tikzcd}[column sep = 30pt, row sep = 35pt]
        A \tensor A \tensor A
        \arrow{r}[above]{m \tensor \id}
        \arrow{d}[left]{\id \tensor m}
      & A \tensor A
        \arrow{d}[right]{m}
      \\
        A \tensor A
        \arrow{r}[below]{m}
      & A
    \end{tikzcd}
    \quad\quad
    \begin{tikzcd}[column sep = 10pt, row sep = 10pt]
        {}
      & A \tensor A
        \arrow{dd}[right]{m}
      & {}
      \\
        A \tensor k
        \arrow{ur}{\id \tensor u}
        \arrow{dr}[below left]{\sim}
      & {}
      & k \tensor A
        \arrow{ul}[above right]{u \tensor \id}
        \arrow{dl}[below right]{\sim}
      \\
        {}
      & A
      & {}
    \end{tikzcd}
  \]
  The commutativity of the first diagram encodes the associativity of the multiplication~$m$ and the commutativity of the second diagram encodes that the elemente~$u(1)$ is the unit with respect to this multiplication.
  
  By reversing the arrows in these diagrams we arrive at the notion of a coalgebra:
\end{fluff}


\begin{definition}
  A~\dash{($k$}{)coalgebra}\index{coalgebra} is a~{\kvs}~$C$ together with~\dash{$k$}{linear} maps
  \[
    \Delta \colon A \to A \tensor A
    \quad\text{and}\quad
    \varepsilon \colon A \to k
  \]
  such that the diagrams
  \[
    \begin{tikzcd}[column sep = 30pt, row sep = 35pt]
        C
        \arrow{r}[above]{\Delta}
        \arrow{d}[left]{\Delta}
      & C \tensor C
        \arrow{d}[right]{\id \tensor \Delta}
      \\
        C \tensor C
        \arrow{r}[below]{\Delta \tensor \id}
      & C \tensor C \tensor C
    \end{tikzcd}
    \quad\quad
    \begin{tikzcd}[column sep = 10pt, row sep = 10pt]
        {}
      & C
        \arrow{dl}[above left]{\sim}
        \arrow{dd}{\Delta}
        \arrow{dr}[above right]{\sim}
      & {}
      \\
        C \tensor k
      & {}
      & k \tensor C
      \\
        {}
      & C \tensor C
        \arrow{ul}[below left]{\id \tensor \varepsilon}
        \arrow{ur}[below right]{\varepsilon \tensor \id}
      & {}
    \end{tikzcd}
  \]
  commute.
  The map~$\Delta$ is the \emph{comultiplication}\index{comultiplication} of~$C$ and~$\varepsilon$ is its \emph{counit}\index{counit}.
  The commutativity of the first diagram is the \emph{coassociativity}\index{coassociativity} of~$\Delta$ and the commutativity of the second diagram states that~$\varepsilon$ is \emph{counitial}.
\end{definition}


% \begin{fluff}[Sweedler notation]
%   Let~$C$ be a~\dash{$k$}{coalgebra}.
%   For~$x \in C$ the element~$\Delta(x) \in C$ can be written as a sum of simple tensors
%   \begin{equation}
%   \label{prepreswedler notation}
%       \Delta(x)
%     = \sum_{i=1}^n x_{1,i} \tensor x_{i,2} \,.
%   \end{equation}
%   For calculations it is often not important what the number~$n$ of summands are, and one may write instead
%   \[
%       \Delta(x)
%     = \sum_i x_{1,i} \tensor x_{i,2} \,.
%   \]
%   This expression can further be simplified by dropping the index~$i$ and simply writing
%   \[
%       \Delta(x)
%     = \sum_{(x)} x_{(1)} \tensor x_{(2)} \,.
%   \] 
%   This is known as \emph{Sweedler’s notation}.
%   One can think about Sweedler’s notation as giving a blueprint for how to construct the original expression~\eqref{prepreswedler notation}.
%   
%   The coassociativity
%   \[
%       (\Delta \tensor \id)( \Delta(x) )
%     = (\id \tensor \Delta)( \Delta(x) )
%   \]
%   can in Sweedler’s notation be expressed as
%   \[
%       \sum_{(x)} \sum_{(x_{(1)})} x_{(1)(1)} \tensor x_{(1)(2)} \tensor x_{(2)}
%     = \sum_{(x)} x_{(1)} \tensor \sum_{(x_{(2)})} x_{(2)(1)} \tensor x_{(2)(2)} \,,
%   \]
%   and the counital axiom
%   \[
%     ()
%   \]
% 
% 
% \end{fluff}


\begin{definition}
  A \emph{\dash{$k$}{bialgebra}}\index{bialgebra} is a~\dash{$k$}{algebra}~$B$ together with~\dash{$k$}{linear} maps~$\Delta \colon B \to B \tensor B$ and~$\varepsilon \colon B \to k$ such that~$(B,\Delta,\varepsilon)$ is a~\dash{$k$}{coalgebra} and~$\Delta$,~$\varepsilon$ are algebra homomorphisms.
\end{definition}


\begin{definition}
  A~\dash{$k$}{Hopf algebra}\index{Hopf algebra} is a~\dash{$k$}{bialgebra}~$H$ together with a~\dash{$k$}{linear} map~$S \colon H \to H$ which makes the diagram
  \[
    \begin{tikzcd}[column sep = small]
        {}
      & H \tensor H
        \arrow{rr}[above]{S \tensor \id}
      & {}
      & H \tensor H
        \arrow{dr}[above right]{m}
      & {}
      \\
        H
        \arrow{ur}[above left]{\Delta}
        \arrow{rr}[above]{\varepsilon}
        \arrow{dr}[below left]{\Delta}
      & {}
      & k
        \arrow{rr}[above]{u}
      & {}
      & H
      \\
        {}
      & H \tensor H
        \arrow{rr}[above]{\id \tensor S}
      & {}
      & H \tensor H
        \arrow{ur}[below right]{m}
      & {}
    \end{tikzcd}
  \]
  commutes.
  The map~$S$ is the \emph{antipode}\index{antipode} of~$H$.
\end{definition}


\begin{remark}
  If~$C$ is a~\dash{$k$}{coalgebra} and~$A$ is a~\dash{$k$}{algebra} then one can define on~$\Hom_k(C,A)$ the structure of a~\dash{$k$}{algebra} via the \emph{convolution product}\index{convolution product}~$*$ given by
  \[
              f * g
    \defined  m_A \circ (f \tensor g) \circ \Delta_C
  \]
  for all~$f, g \in \Hom_k(C,A)$.
  The convolution unit of~$\Hom_k(C,A)$ is given by~$u_A \circ \varepsilon_C$.
  If~$B$ is a~\dash{$k$}{bilalgebra} then it follows that~$\End_k(B)$ is a~\dash{$k$}{algebra} with respect to the convolution product, and an antipode for~$B$ is precisely a convolution inverse to the identity~$\id_B \in \End_k(B)$.
  It follows in particular that for every~\dash{$k$}{bialgebra}~$B$ there exists at most one possible antipode map which makes it into a Hopf algebra.
  
  The antipode of a Hopf algebra is an antimorphism of both~\dash{$k$}{algebras} and~\dash{$k$}{coalgebras}.
\end{remark}


\begin{fluff}
  Let~$G$ be an affine algebraic group with multiplication~$m \colon G \times G \to G$, inversion~$G \to G$ and neutral element~$e \in G$;
  let~$j \colon \{e\} \to G$ be the inclusion.
  The induced algebra homomorphism~$m^* \colon \coord(G) \to \coord(G \times G)$ is given by
  \begin{equation}
  \label{action of mstar}
      m^*(f)(x_1, x_2)
    = f(x_1 x_2)
  \end{equation}
  for all~$f \in \coord(G)$,~$x_1, x_2 \in G$, the induced homomorphism~$i^* \colon \coord(G) \to \coord(G)$ is given by
  \[
      i^*(f)(x)
    = f(x^{-1})
  \]
  for all~$f \in \coord(G)$,~$x \in G$
  
  We define~$\Delta \colon \coord(G) \to \coord(G) \tensor \coord(G)$ to be the composition of the algebra homomorphism~$m^* \colon \coord(G) \to \coord(G \times G)$ with the natural isomorphism~$\coord(G \times G) \cong \coord(G) \tensor \coord(G)$ from \cref{coordinate ring of product of qaffine}.
  The homomorphism~$\Delta$ is thus given by~$\Delta(f) = \sum_{i=1}^n f_1 \tensor f_2$ such that
  \[
      m^*(f)(x_1, x_2)
    = \sum_{i=1}^n f_1(x_1) f_2(x_2)
  \]
  for all~$x_1, x_2 \in G$.
  \Cref{action of mstar} then becomes
  \[
      f(x_1 x_2)
    = \sum_{i=1}^n f_1(x_1) \tensor f_2(x_2)
  \]
  for all~$f \in \coord(G)$,~$x_1, x_2 \in G$.  
  We further define~$\varepsilon$ to be the composition of~$j^* \colon \coord(G) \to \coord(\{e\})$ with the (unique) isomorphism (of \dash{$k$}{algebras})~$\coord(\{e\}) \cong k$.
  The homomorphism~$\varepsilon$ is given by
  \[
      \varepsilon(f)
    = f(e)
  \]
  for all~$f \in \coord(G)$.
  Lastly we define~$S \defined i^*$, which is given by
  \[
      S(f)(x)
    = f(x^{-1})
  \]
  for all~$f \in \coord(G)$,~$x \in G$.
  
  That~$m, i, e$ give a group structure on~$G$ can be encoded in the commutativity of the following diagrams:
  \[
    \begin{tikzcd}[column sep = 30pt, row sep = 35pt]
        G \times G \times G
        \arrow{r}[above]{m \times \id}
        \arrow{d}[left]{\id \times m}
      & G \times G
        \arrow{d}[right]{m}
      \\
        G \times G
        \arrow{r}[below]{m}
      & G
    \end{tikzcd}
    \quad
    \begin{tikzcd}[column sep = 10pt, row sep = 9pt]
        {}
      & G \times G
        \arrow{dd}[right]{m}
      & {}
      \\
        G \times \{e\}
        \arrow{ur}[above left]{\id \times j}
        \arrow{dr}[below left]{\pr_1}
      & {}
      & \{e\} \times G
        \arrow{ul}[above right]{j \times \id}
        \arrow{dl}[below right]{\pr_2}
      \\
        {}
      & G
      & {}
    \end{tikzcd}
    \quad
    \begin{tikzcd}[column sep = 3pt, row sep = 9pt]
        G
        \arrow{rr}[above]{(i,\id)}
        \arrow{dd}[left]{(\id,i)}
        \arrow{dr}
      & {}
      & G \times G
        \arrow{dd}[right]{m}
      \\
        {}
      & \{e\}
        \arrow{dr}{j}
      & {}
      \\
        G \times G
        \arrow{rr}[below]{m}
      & {}
      & G
    \end{tikzcd}
  \]
  By applying the contravariant functor~$\coord(-)$ to these diagrams we get commutative diagrams involving~$\coord(G)$,~$\Delta$,~$\varepsilon$ and~$S$, and which will show that these homomorphisms give~$\coord(G)$ the structure of a Hopf-algebra.
  \begin{itemize}
    \item
      By applying~$\coord(-)$ to the first diagram we get the following commutative diagram:
      \[
        \begin{tikzcd}[sep = large]
            \coord(G)
            \arrow{r}[above]{m^*}
            \arrow{d}[left]{m^*}
          & \coord(G \times G)
            \arrow{d}[right]{(\id \times m)^*}
          \\
            \coord(G \times G)
            \arrow{r}[below]{(m \times \id)^*}
          & \coord(G \times G \times G)
        \end{tikzcd}
      \]
      By using the natural (!) isomorphism~$\coord(G \times G) \cong \coord(G) \tensor \coord(G)$ this becomes the following commutative diagram:
      \begin{equation}
      \label{coassociativity}
        \begin{tikzcd}[sep = large]
            \coord(G)
            \arrow{r}[above]{\Delta}
            \arrow{d}[left]{\Delta}
          & \coord(G) \tensor \coord(G)
            \arrow{d}[right]{\id \tensor \Delta}
          \\
            \coord(G) \tensor \coord(G)
            \arrow{r}[below]{\Delta \tensor \id}
          & \coord(G) \tensor \coord(G) \tensor \coord(G)
        \end{tikzcd}
      \end{equation}
      This diagram gives the coassociativity of~$\Delta$.
    \item
      By applying~$\coord(-)$ to the second diagram we get the following commutative diagram:
      \[
        \begin{tikzcd}
            {}
          & \coord(G)
            \arrow{dl}[above left]{\pr_1^*}
            \arrow{dd}[right]{m^*}
            \arrow{dr}[above right]{\pr_2^*}
          & {}
          \\
            \coord(G \times \{e\})
          & {}
          & \coord(\{e\} \times G)
          \\
            {}
          & \coord(G \times G)
            \arrow{ul}[below left]{(\id \times i)^*}
            \arrow{ur}[below right]{(i \times \id)^*}
          & {}
        \end{tikzcd}
      \]
      By applying the natural isomorphism~$\coord(G \times \{e\}) \cong \coord(G) \tensor k$ the algebra homomorphisms~$\pr_1^*$ becomes the natural isomorphism~$\coord(G) \cong \coord(G) \tensor k$, and similarly for~$\pr_2^*$.
      We thus get the following commutative diagram:
      \[
        \begin{tikzcd}
            {}
          & \coord(G)
            \arrow{dl}[above left]{\sim}
            \arrow{dd}[right]{\Delta}
            \arrow{dr}[above right]{\sim}
          & {}
          \\
            \coord(G) \tensor k
          & {}
          & k \tensor \coord(G)
          \\
            {}
          & \coord(G) \tensor \coord(G)
            \arrow{ul}[below left]{\id \tensor \varepsilon}
            \arrow{ur}[below right]{\varepsilon \tensor \id}
          & {}
        \end{tikzcd}
      \]
      This diagram gives that~$\varepsilon$ is a counit.
  \end{itemize}
  Together this shows that~$\Delta$,~$\varepsilon$ endow~$\coord(G)$ with the structure of a~\dash{$k$}{bilalgebra}.
  \begin{itemize}[resume]
    \item
      Before we apply~$\coord(-)$ to the third diagram we rewrite this diagram as follows, where~$d$ denotes the diagonal map:
      \[
        \begin{tikzcd}[column sep = small]
            {}
          & G \times G
            \arrow{rr}[above]{i \times \id}
          & {}
          & G \times G
            \arrow{dr}[above right]{m}
          & {}
          \\
            G
            \arrow{ur}[above left]{d}
            \arrow{rr}[above]{}
            \arrow{dr}[below left]{d}
          & {}
          & \{e\}
            \arrow{rr}[above]{j}
          & {}
          & G
          \\
            {}
          & G \times G
            \arrow{rr}[above]{\id \times S}
          & {}
          & G \times G
            \arrow{ur}[below right]{m}
          & {}
        \end{tikzcd}
      \]
      By applying the functor~$\coord(-)$ to this diagram we get the following commutative diagram:
      \[
        \begin{tikzcd}[column sep = tiny]
            {}
          & \coord(G \times G)
            \arrow{rr}[above]{(i \times \id)^*}
          & {}
          & \coord(G \times G)
            \arrow{dr}[above right]{d^*}
          & {}
          \\
            \coord(G)
            \arrow{ur}[above left]{m*}
            \arrow{rr}[above]{j^*}
            \arrow{dr}[below left]{m*}
          & {}
          & \coord(\{e\})
            \arrow{rr}
          & {}
          & \coord(G)
          \\
            {}
          & \coord(G \times G)
            \arrow{rr}[above]{(\id \times i)^*}
          & {}
          & \coord(G \times G)
            \arrow{ur}[below right]{d^*}
          & {}
        \end{tikzcd}
      \]
      (The unlabeled arrow is the unique homomorphism of~\dash{$k$}{algebras}.)
      By applying the natural isomorphism~$\coord(G \times G) \cong \coord(G) \tensor \coord(G)$ and the isomorphism~$\coord(\{e\}) \cong k$ the induced morphism~$d^*$ becomes the multiplication~$\coord(G) \tensor \coord(G) \to \coord(G)$,~$f \tensor g \mapsto fg$.
      We thus get the following commutative diagram:
            \[
        \begin{tikzcd}[column sep = tiny]
            {}
          & \coord(G) \tensor \coord(G)
            \arrow{rr}[above]{S \mathbin{\otimes} \id}
          & {}
          & \coord(G) \tensor \coord(G)
            \arrow{dr}[above right]{\text{mult}}
          & {}
          \\
            \coord(G)
            \arrow{ur}[above left]{\Delta}
            \arrow{rr}[above]{\varepsilon}
            \arrow{dr}[below left]{\Delta}
          & {}
          & k
            \arrow{rr}
          & {}
          & \coord(G)
          \\
            {}
          & \coord(G) \tensor \coord(G)
            \arrow{rr}[above]{\id \tensor S}
          & {}
          & \coord(G) \tensor \coord(G)
            \arrow{ur}[below right]{\text{mult}}
          & {}
        \end{tikzcd}
      \]
      The commutativity of this diagram shows that~$S$ is an antipode for the bialgebra~$(\coord(G), \Delta, \varepsilon)$.
  \end{itemize}

  Altogether we have seen and shown that the groups structure on the affine variety~$G$ induces on its coordinate ring~$\coord(G)$ the structure of a Hopf algebra with
  \begin{itemize}
    \item
      the comultiplication~$\Delta$ of~$\coord(G)$ being induced by the multiplication and~$G$ and given by~$\Delta(f) = \sum_{i=1}^n f_1 \otimes f_2$ such that
      \begin{equation}
      \label{explicit description of comultiplication}
          f(x_1 x_2)
        = \sum_{i=1}^n f_1(x_1) f_2(x_2)
      \end{equation}
      for all~$f \in \coord(G)$,~$x_1, x_2 \in G$;
    \item
      the counit~$\varepsilon$ of~$\coord(G)$ being induced by by the neutral element of~$G$ and given by evaluation at~$e$, i.e.\ by
      \begin{equation}
      \label{explicit description of counit}
          \varepsilon(f)
        = f(e)
      \end{equation}
      for all~$f \in \coord(G)$;
    \item
      and the antipode~$S$ of~$\coord(G)$ being induced by the inversion of~$G$ and given by
      \begin{equation}
      \label{explicit description of antipode}
          S(f)(x)
        = f(x^{-1})
      \end{equation}
      for all~$f \in \coord(G)$,~$x \in G$.
  \end{itemize}
\end{fluff}


\begin{example}
  Let us consider the additive groups~$\Gadd = (\Aff^1, +)$.
  The coordinate ring of~$\Gadd$ is given by
  \[
      \coord(\Gadd)
    = \coord(\Aff^1)
    = k[x] \,.
  \]
  To determine the action of the comultiplication~$\Delta \colon k[x] \to k[x] \otimes k[x]$ on the algebra generator~$x \in k[x]$ we note that
  \[
      x(y_1 + y_2)
    = y_1 + y_2
    = x(y_1) \cdot 1 + 1 \cdot x(y_2)
  \]
  for all~$y_1, y_2 \in G$.
  It follows from the explicit description of the comultiplication given in~\eqref{explicit description of comultiplication} that
  \[
      \Delta(x)
    = x \otimes 1 + 1 \otimes x \,.
  \]
  The counit~$\varepsilon \colon k[x] \to k$ is by~\eqref{explicit description of counit} given by~$\varepsilon(f) = f(0)$ for every~$f \in k[x]$.
  It is on the algebra generator~$x \in k[x]$ given by
  \[
      \varepsilon(x)
    = x(0)
    = 0 \,.
  \]
  The antipode~$S \colon k[x] \to k[x]$ is by~\eqref{explicit description of antipode} given by~$S(f(x)) = f(-x)$ for all~$f \in k[x]$.
  It is on the algebra generator~$x \in k[x]$ given by
  \[
      S(x)
    = -x \,.
  \]
\end{example}


\begin{example}
  The coordinate ring of the general linear group~$\GL_n(k)$ is given by
  \[
      \coord(\GL_n(k))
    = k\left[ x_{11}, \dotsc, x_{nn}, \det^{-1} \right]
  \]
  where the element~$\det \in k[x_{11}, \dotsc, x_{nn}]$ is given by
  \[
      \det
    = \sum_{\sigma \in \symm_n} \sgn(\sigma) x_{1\sigma(1)} \dotsm x_{n\sigma(n)} \,.
  \]
  
  To determine the action of the comultiplication~$\Delta$ on the algebra generators~$x_{ij}$ we note that
  \[
      x_{ij}(A_1 A_2)
    = \sum_{\ell=1}^n (A_1)_{i \ell} (A_2)_{\ell j}
    = \sum_{\ell=1}^n x_{i \ell}(A_1) x_{\ell j}(A_2)
  \]
  for all~$A_1, A_2 \in \GL_n(k)$.
  It follows from the explicit description of the comultiplication given in~\eqref{explicit description of comultiplication} that
  \[
      \Delta(x_{ij})
    = \sum_{\ell = 1}^n x_{i \ell} x_{\ell j} \,.
  \]
  To determine the action of~$\Delta$ on the generator~$\det^{-1}$ we note that
  \[
      \det^{-1}(A_1 A_2)
    = \frac{1}{\det(A_1 A_2)}
    = \frac{1}{\det(A_1) \det(A_2)}
    = \frac{1}{\det(A_1)} \cdot \frac{1}{\det(A_2)}
    = \det^{-1}(A_1) \cdot \det^{-1}(A_2)
  \]
  for all~$A_1, A_2 \in \GL_n(k)$.
  It follows that
  \[
      \Delta\left( \det^{-1} \right)
    = \det^{-1} \otimes \det^{-1} \,.
  \]
  
  The action of the counit~$\varepsilon$ is by~\eqref{explicit description of counit} given by~$\varepsilon(f) = f(I)$ for all~$f \in k[x_{11}, \dotsc, x_{nn}, \det^{-1}]$.
  It follows that the action of~$\varepsilon$ on the algebra generators~$x_{ij}$ is given by
  \[
      \varepsilon(x_{ij})
    = x_{ij}(I)
    = \delta_{ij}
  \]
  and that the action of~$\varepsilon$ on the algebra generator~$\det^{-1}$ is given by
  \[
      \varepsilon\left( \det^{-1} \right)
    = \det^{-1}(I)
    = \frac{1}{\det(I)}
    = \frac{1}{1}
    = 1 \,.
  \]
  
  The action of the antipode~$S$ is by~\eqref{explicit description of antipode} given by~$S(f)(A) = f(A^{-1})$ for all~$f \in \coord(\GL_n(k))$ and all~$A \in \GL_n(k)$.
  It therefore follows from
  \[
      \det^{-1}\left( A^{-1} \right)
    = \frac{1}{\det(A^{-1})}
    = \frac{1}{\det(A)^{-1}}
    = \det(A)
  \]
  that~$S(\det^{-1}) = \det$.
  The action of~$S$ on the algebra generators~$x_{ij}$ is messier to write down, as it boils down to explicitely expressing the coordinates of~$A^{-1}$ in terms of the coordinates of~$A$ via Cramer’s rule.
  One finds that
  \begin{multline*}
      S(x_{ij})
    = (-1)^{ij} \det^{-1}
      \sum_{\sigma \in \symm_{n-1}}
      \sgn(\sigma)
      \cdot
      \left(
        \sum_{\substack{p,q=1,\dotsc,i-1 \\ p < i, \sigma(q) < j}}
        x_{i,\sigma(j)}
        +
        \sum_{\substack{p,q=1,\dotsc,n-1 \\ p \geq i, \sigma(q) < j}}
        x_{i+1,\sigma(j)}
      \right.
    \\
      \left.
        +
        \sum_{\substack{p,q=1,\dotsc,n-1 \\ p < i, \sigma(q) \geq j}}
        x_{i,\sigma(j)+1}
        +
        \sum_{\substack{p,q=1,\dotsc,n-1 \\ p \geq i, \sigma(q) \geq j}}
        x_{i+1,\sigma(j)+1}
        \right)
  \end{multline*}
\end{example}


\begin{example}
  As a special case of the previous example we find that the coordinate ring of the multilplicative group~$\Gmult = (k^\times, \cdot) = \GL_1(k)$ is given by
  \[
      \coord(\Gmult)
    = k[x, x^{-1}]
  \]
  with the comultiplication~$\Delta$, counit~$\varepsilon$ and antipode~$S$ being given on the generators~$x$,$~x^{-1}$ by
  \[
      \Delta\left( x^{\pm 1} \right)
    = x^{\pm 1} \otimes x^{\pm 1} \,,
    \qquad
      \varepsilon\left( x^{\pm 1} \right)
    = 1 \,,
    \qquad
      S\left( x^{\pm 1} \right)
    = x^{\mp 1} \,.
  \]
\end{example}


% TODO: A(G x H) = A(G) ⊗ A(H) as Hopf algebras -> A^n and D_n
% TODO: Tn, Un ?


\begin{remark}
  One can show that the contravariant equivalence of categories
  \begin{align*}
            \coord(-)
    \colon  \{
              \text{affine varieties}
            \}
    \longto \left\{
              \begin{tabular}{@{}c@{}}
                finitely generated,   \\
                commutative, reduced  \\
                \dash{$k$}{algebras}
              \end{tabular}
            \right\}
  \end{align*}
  from \cref{equivalence for affine varieties} induces a contravariant equivalence of categories
  \begin{align*}
            \coord(-)
    \colon  \{
              \text{affine algebraic groups}
            \}
    \longto \left\{
              \begin{tabular}{@{}c@{}}
                finitely generated, \\
                commutative, reduced  \\
                \dash{$k$}{Hopf algebras}
              \end{tabular}
            \right\} \,.
  \end{align*}
  In this sense a group structure on an affine variety~$G$ (which is given by morphisms of affine varieties) is \enquote{the same} as a Hopf algebra structure on its coordinate ring~$\coord(G)$.
% TODO: Give brief overview of how to retrieve the group structure from the coordinate ring.
\end{remark}













