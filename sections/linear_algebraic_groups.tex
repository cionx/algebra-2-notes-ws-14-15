\section{Linear Algebraic Groups}


\begin{conventions}
  Throughout these notes $k$ denotes an algebraically closed field.
\end{conventions}


\begin{definition}
  A \emph{linear algebraic group} is a subgroup $G \groupleq \GL_n(k)$ (for some $n$) which is cut out by polynomials, i.e.\ there exist $f_1, \dotsc, f_r \in k[x_{11}, \dotsc, x_{nn}]$ such that
  \[
      G
    = \{
        A \in \GL_n(k)
      \suchthat
        f_1(A) = \dotsb = f_n(A) = 0
      \} \,.
  \]
\end{definition}


\begin{example}
  \leavevmode
  \begin{enumerate}
    \item
      The \emph{general linear group}.
    \item
      The \emph{multiplicative group} $\Gmult \defined \GL_1(k) = k^\times$.
    \item
      The \emph{additive group} $\Gadd \defined (K,+)$ can be regarded as a linear algebraic group by identifying it with
      \[
                  \Gadd
        \cong     \left\{
                    \begin{pmatrix}
                      1 & x \\
                      0 & 1
                    \end{pmatrix}
                  \suchthat*
                    x \in k
                  \right\}
        \groupleq \GL_2(k)
      \]
    \item
      The \emph{special linear group} $\SL_n(k) = \{ A \in \GL_n(k) \suchthat \det A = 1 \}$.
    \item
      The \emph{orthogonal groups} $\Orth_n(k) = \{ A \in \GL_n(k) \suchthat A^T A = I \}$.
    \item
      The \emph{special orthogonal group} $\SOrth_n(k) = \SL_n(k) \cap \Orth_n(k)$.
    \item
      The \emph{symplectic group} $\Symp_{2n}(k) = \{ A \in \GL_n(k) \suchthat A^T J A = J \}$ where
      $
          J
        = \begin{psmallmatrix}
             0    & I_n \\
            -I_n  & 0
          \end{psmallmatrix}
      $.
    \item
      Every finite subgroup of $\GL_n(k)$ is a linear algebraic group.
    \item
      The \emph{group of diagonal matrices}
      $
          \Diag_n(k)
        = \left\{
            \begin{psmallmatrix}
              * &         &   \\
                & \sddots &   \\
                &         & *
            \end{psmallmatrix}
            \in \GL_n(k)
          \right\}
      $.
    \item
      The \emph{group of upper triangular matrices}
      $
          \Triag_n(k)
        = \left\{
            \begin{psmallmatrix}
              * & \scdots & *       \\
                & \sddots & \svdots \\
                &         & *
            \end{psmallmatrix}
            \in \GL_n(k)
          \right\} \,.
      $
    \item
      The \emph{group of unipotent matrices}
      $
          \Uni_n(k)
        = \left\{
            \begin{psmallmatrix}
              1 & \scdots & *       \\
                & \sddots & \svdots \\
                &         & 1
            \end{psmallmatrix}
            \in \GL_n(k)
          \right\} \,.
      $
  \end{enumerate}
\end{example}





\subsection{Affine Algebraic Geometry}


\begin{definition}
  The \emph{affine $n$-space over $k$} is $\Aff^n \defined \Aff^n_k \defined k^n$.
\end{definition}


\begin{definition}
  For every subset $S \subseteq k[x_1, \dotsc, x_n]$ the set
  \[
              \Vset(S)
    \defined  \{
                x \in \Aff^n
              \suchthat
                \text{$f(x) = 0$ for every $f \in S$}
              \}
  \]
  is the \emph{vanishing locus} of $S$ or \emph{algebraic set} given by $S$ or \emph{affine variety} given by $S$.
\end{definition}


\begin{lemma}
  \leavevmode
  \begin{enumerate}
    \item
      It holds for every subset $S \subseteq k[x_1, \dotsc, x_n]$ that $\Vset(S) = \Vset( (S) )$.
    \item
      It holds for all ideals $I_1 \idealleq I_2 \idealleq k[x_1, \dotsc, x_n]$ that $\Vset(I_1) \supseteq \Vset(I_2)$.
    \item
      It holds for all ideals $I_1, I_2 \idealleq k[x_1, \dotsc, x_n]$ that
      \[
          \Vset(I_1) \cup \Vset(I_2)
        = \Vset(I_1 \cap I_2)
        = \Vset(I_1 \cdot I_2) \,.
      \]
    \item
      It holds for every family $I_\lambda$, $\lambda \in \Lambda$ of ideals $I_\lambda \idealleq k[x_1, \dotsc, x_n]$ that
      \[
          \bigcap_{\lambda \in \Lambda} \Vset(I_\lambda)
        = \Vset\left( \bigcup_{\lambda \in \Lambda} I_\lambda \right)
        = \Vset\left( \sum_{\lambda \in \Lambda} I_\lambda \right) \,.
      \]
    \item
      It holds that $\Vset(1) = \emptyset$.
    \item
      It holds that $\Vset(0) = \Aff^n$.
    \qed
  \end{enumerate}
\end{lemma}


\begin{corollary}
  \label{corollary: existence of zariski topology}
  There exists a unique topology on $\Aff^n$ whose closed subsets are the subsets of the form $\Vset(S)$ for subsets $S \subseteq k[x_1, \dotsc, x_n]$.
  \qed
\end{corollary}


\begin{definition}
  The topology from Corollary~\ref{corollary: existence of zariski topology} is the \emph{Zariski topology} on $\Aff^n$.
\end{definition}


\begin{remark}
  Linear algebra groups are precisely those subgroups of $\GL_n(k)$ for some $n$ which are closed with respect to (the subspace topology of $\GL_n(k)$ induced by) the Zariski topology.
\end{remark}


\begin{theorem}[Hilbert]
  Every ideal $I \idealleq k[x_1, \dotsc, x_n]$ is finitely generated.
  \qed
\end{theorem}


\begin{corollary}
  It holds for every subset $S \subseteq k[x_1, \dotsc, x_n]$ that
  \[
      \Vset(S)
    = \Vset(f_1, \dotsc, f_m)
    = \Vset(f_1) \cap \dotsb \cap \Vset(f_m) \,.
  \]
  for some finitely many $f_1, \dotsc, f_m \in S$.
  \qed
\end{corollary}


\begin{theorem}[Hilbert’s~Nullstellensatz, version~1]
  It holds for every proper ideal $I \ideallneq k[x_1, \dotsc, x_n]$ that $\Vset(I) \neq \emptyset$.
  \qed
\end{theorem}


\begin{definition}
  For every subset $X \subseteq \Aff^n$ the set
  \[
              \Ideal(X)
    \defined  \{
                f \in k[x_1, \dotsc, x_n]
              \suchthat
                \text{$f(x) = 0$ for every $x \in X$}
              \}
  \]
  is the \emph{vanishing ideal of $X$}.
\end{definition}


\begin{lemma}
  \leavevmode
  \begin{enumerate}
    \item
      For every subset $X \subseteq \Aff^n$ the vanishing ideal $\Ideal(X)$ is an ideal in $k[x_1, \dotsc, x_n]$.
    \item
      It holds for all subsets $X_1 \subseteq X_2 \subseteq \Aff^n$ that $\Ideal(X_1) \idealgeq \Ideal(X_2)$.
    \item
      It holds for every family $(X_\lambda)_{\lambda \in \Lambda}$ of subsets $X_\lambda \subseteq \Aff^n$ that
      \[
          \Ideal\left( \bigcup_{\lambda \in \Lambda} X_\lambda \right)
        = \bigcap_{\lambda \in \Lambda} \Ideal(X_\lambda) \,.
      \]
    \item
      It holds that $\Ideal(\emptyset) = (1) = k[x_1, \dotsc, x_n]$.
    \item
      It holds that $\Ideal(\Aff^n) = 0$ (because $k$ is infinite).
    \qed
  \end{enumerate}
\end{lemma}


\begin{definition}
  The \emph{coordinate ring} of an affine variety $X \subseteq \Aff^n$ is the $k$-algebra
  \[
              \coord(X)
    \defined  k[x_1, \dotsc, x_n]/{\Ideal(X)} \,.
  \]
\end{definition}


\begin{definition}
  Let $R$ be a commutative ring.
  \begin{enumerate}
    \item
      The ring $R$ is \emph{reduced} if $0 \in R$ is the only nilpotent element of $R$.
    \item
      The \emph{radical} of an ideal $I \idealleq R$ is
      \[
                  \rad{I}
        \defined  \{
                    f \in R
                  \suchthat
                    \text{$f^n \in I$ for some $n \geq 0$}
                  \} \,.
      \]
    \item
      An ideal $I \idealleq R$ is \emph{radical} if $I = \rad{I}$.
  \end{enumerate}
\end{definition}


\begin{lemma}
  Let $R$ be a commutative ring.
  \begin{enumerate}
    \item
      For every ideal $I \idealleq R$ its radical $\rad{I}$ is again an ideal in $R$.
    \item
      An ideal $I \idealleq R$ is radical if and only if the quotient $R/I$ is reduced.
    \qed
  \end{enumerate}
\end{lemma}


\begin{lemma}
  For every subset $X \subseteq \Aff^n$ its vanishing ideal $\Ideal(X) \idealleq k[x_1, \dotsc, x_n]$ is radical.
  \qed
\end{lemma}


\begin{corollary}
  For every affine variety $X$ its coordinate ring $\coord(X)$ is reduced.
\end{corollary}


\begin{definition}
  Let $X$ be a topological space.
  \begin{enumerate}
    \item
      The space $X$ is \emph{reducible} if $X = C_1 \cup C_2$ for some closed subsets $C_1, C_2 \subsetneq X$.
    \item
      The space $X$ is \emph{irreducible} if it is nonempty and not reducible.
  \end{enumerate}
\end{definition}


\begin{lemma}
  For a nonempty topological space $X$ the following conditions are equivalent:
  \begin{enumerate}
    \item
      The space $X$ is irreducible.
    \item
      Every two nonempty open subsets of $X$ intersect nontrivially.
    \item
      Every nonemtpy open subset of $X$ is dense.
    \qed
  \end{enumerate}
\end{lemma}


\begin{lemma}
  \label{lemma: irreducible is connected}
  Every irreducible space is connected.
  \qed
\end{lemma}


\begin{example}
  The converse to Lemma~\ref{lemma: irreducible is connected} does not hold:
  The vanishing set $\Vset(xy) \subseteq \Aff^2$ is connected but reducible.
\end{example}


\begin{proposition}
  \label{proposition: existence of irreducible components}
  If $X$ is a topological space then there exist closed irreducible subsets $C_\lambda \subseteq X$, $\lambda \in \Lambda$ with $X = \bigcup_{\lambda \in \Lambda} C_\lambda$ and $C_\lambda \nsubseteq C_\mu$ for $\lambda \neq \mu$, and the sets $C_\lambda$, $\lambda \in \Lambda$ are unique up to permutation.
  \qed
\end{proposition}


\begin{definition}
  The sets $C_\lambda$, $\lambda \in \Lambda$ from Proposition~\ref{proposition: existence of irreducible components} are the irreducible components of $X$.
\end{definition}


\begin{definition}
  A topological space $X$ is \emph{noetherian} if every descending sequence
  \[
              C_1
    \supseteq C_2
    \supseteq C_3
    \supseteq \dotsb
  \]
  of closed subsets $C_i \subseteq X$ stabilizes, or equivalently if every ascending sequence
  \[
              U_1
    \subseteq U_2
    \subseteq U_3
    \subseteq \dotsb
  \]
  of open subsets $U_i \subseteq X$ stabilizes.
\end{definition}


\begin{lemma}
  If a topological space $X$ is noetherian then $X$ has only finitely many irreducible components.
  \qed
\end{lemma}


\begin{lemma}
  Any affine variety is noetherian.
  \qed
\end{lemma}


\begin{corollary}
  Any affine variety has only finitely many irreducible components.
  \qed
\end{corollary}


\begin{theorem}[Hilbert’s~Nullstellensatz, version~2]
  The maps $\Vset, \Ideal$ restrict to the following bijections:
  \[
    \begin{matrix}
        \left\{
          \begin{tabular}{c}
              affine varieties \\
              $X \subseteq \Aff^n$
          \end{tabular}
        \right\}
      & \begin{tikzcd}[column sep = large]
            {}
            \arrow[shift left]{r}{\Ideal}
          & {}
            \arrow[shift left]{l}{\Vset}
        \end{tikzcd}
      & \left\{
          \begin{tabular}{c}
            radical ideals \\
            $I \idealleq k[x_1, \dotsc, x_n]$
          \end{tabular}
        \right\}
      \\
        {}
      & {}
      & {}
      \\
        \rotatebox[origin=c]{90}{$\subseteq$}
      & {}
      & \rotatebox[origin=c]{90}{$\subseteq$}
      \\
        {}
      & {}
      & {}
      \\
        \left\{
          \begin{tabular}{c}
              irreducible \\
              affine varieties \\
              $X \subseteq \Aff^n$
          \end{tabular}
        \right\}
      & \begin{tikzcd}[column sep = large]
            {}
            \arrow[shift left]{r}{\Ideal}
          & {}
            \arrow[shift left]{l}{\Vset}
        \end{tikzcd}
      & \left\{
          \begin{tabular}{c}
            prime ideals \\
            $\mf{p} \idealleq k[x_1, \dotsc, x_n]$
          \end{tabular}
        \right\}
      \\
        {}
      & {}
      & {}
      \\
        \rotatebox[origin=c]{90}{$\subseteq$}
      & {}
      & \rotatebox[origin=c]{90}{$\subseteq$}
      \\
        {}
      & {}
      & {}
      \\
        \left\{
          \begin{tabular}{c}
            points $p \in \Aff^n$
          \end{tabular}
        \right\}
      & \begin{tikzcd}[column sep = large]
            {}
            \arrow[shift left]{r}{\Ideal}
          & {}
            \arrow[shift left]{l}{\Vset}
        \end{tikzcd}
      & \left\{
          \begin{tabular}{c}
            maximal ideals \\
            $\mf{m} \idealleq k[x_1, \dotsc, x_n]$
          \end{tabular}
        \right\}
    \end{matrix}
  \]
  For every point $p = (p_1, \dotsc, p_n) \in \Aff^n$ the corresponding maximal ideal is given by $\mf{m}_p = (x_1 - p_1, \dotsc, x_n - p_n)$.
\end{theorem}


\begin{definition}
  For every linear algebraic group $G$ the connected component of $1 \in G$ is denoted by $G^0$.
\end{definition}


\begin{proposition}
  Let $G$ be a linear algebraic group.
  \begin{enumerate}
    \item
      The connected components of $G$ coincide with the irreducible components.
    \item
      The connected/irreducible component $G^0$ is a normal subgroup of $G$.
    \item
      The connected/irreducible components of $G$ are the cosets of $G^0$.
    \item
      The group $G^0$ has finite index in $G$.
  \end{enumerate}
\end{proposition}











