\section{Jordan--Chevalley for Linear Algebraic Groups}


\begin{lemma}
  \label{properties of tensor product of local endomorphisms}
  Let~$g \colon V \to V$ and~$h \colon W \to W$ be endomorphism of~\dash{$k$}{vector spaces}~$V$ and~$W$.
  \begin{enumerate}
    \item
      If both~$g$ and~$h$ are semisimple then the endomorphism~$g \tensor h$ is again semisimple.
    \item 
      If~$g$ or~$h$ is locally nilpotent then~$g \tensor h$ is again locally nilpotent.
    \item
      If~$g$ and~$h$ are locally unipotent then~$g \tensor h$ is again locally unipotent.
  \end{enumerate}
\end{lemma}


\begin{proof}
  \leavevmode
  \begin{enumerate}
    \item
      If~$V$ and~$W$ have eigenspace decomposition~$V = \bigoplus V_\lambda(g)$ and~$W = \bigoplus_\mu W_\mu(h)$ then
      \[
          V \tensor W
        =         \left( \bigoplus_\lambda V_\lambda(g) \right)
          \tensor \left( \bigoplus_\mu W_\mu(h) \right)
        = \bigoplus_{\kappa}
          \underbrace{
          \left( \bigoplus_{\lambda + \mu = \kappa} V_\lambda(g) \tensor W_\mu(h) \right)
          }_{= (V \tensor W)_\kappa(g \tensor h)}
      \]
      is the decomposition of~$V \tensor W$ into eigenspaces of~$g \tensor h$.
    \item
      We consider only the case that~$g$ is locally nilpotent.
      For every simple tensor~$v \tensor w \in V \tensor W$ there then exists some power~$n$ for which~$f^n(v) = 0$, and for which it then follows that
      \[
          (g \tensor h)^n(v \tensor h)
        = g^n(v) \tensor h^n(v)
        = 0 \,.
      \]
      An arbitrary element~$x \in V \tensor W$ can be written as a sum of simple tensors~$x = \sum_{i=1}^r v_i \tensor w_i$.
      There then exists some power~$n$ with~$g^n(v_i) = 0$ for all~$i = 1, \dotsc, r$, and it follows from the above that~$(g \tensor h)^n(x) = 0$.
    \item
      It holds that
      \begin{align*}
           &  g \tensor h - \id_{V \tensor W} \\
        ={}&  g \tensor h - {\id_V} \tensor {\id_W} \\
        ={}&  g \tensor h - {\id_V} \tensor h + {\id_V} \tensor h - {\id_V} \tensor {\id_W} \\
        ={}&  (g - \id_V) \tensor h + {\id_V} \tensor (h - \id_W) \,.
      \end{align*}
      The endomorphism~$g - \id_V$ and~$h - \id_W$ are locally nilpotent, and it follows from the above that the summand~$(g - \id_V) \tensor h$ is again locally nilpotent.
      The summands~$(g - \id_V) \tensor h$ and~${\id_V} \tensor (h - \id_W)$ commute factorwise and therefore commute with each other.
      It follows from \cref{combination of locally potent endomorphisms} that~$g \tensor h - \id_{V \tensor W}$ is locally nilpotent.
    \qedhere
  \end{enumerate}
\end{proof}


\begin{lemma}
  Let~$V$,~$W$ be~\dash{$k$}{vector spaces} and let~$g \in \GL(V)$,~$h \in \GL(W)$ be locally finite.
  Then~$g \tensor h \in \GL(V \tensor W)$ is locally finite with
  \[
      (g \tensor h)_s
    = g_s \tensor h_s
    \quad\text{and}\quad
      (g \tensor h)_u
    = g_u \tensor h_u \,.
  \]
\end{lemma}


\begin{proof}
  It holds that
  \[
      g \tensor h
    = (g_s g_u) \tensor (h_s h_u)
    = (g_s \tensor h_s) (g_u \tensor h_u)
  \]
  and the factors~$g_s \tensor h_s$ and~$g_u \tensor h_u$ commute with each other because they do so factorwise.
  It follows from \cref{properties of tensor product of local endomorphisms} that the factor~$g_s \tensor h_s$ is again semisimple and that the factor~$g_u \tensor h_s$ is again locally unipotent.
\end{proof}


\begin{lemma}
  \label{homomorphisms of Gsets}
  Let~$G$ be a group and let~$\varphi \colon G \to G$ be a homomorphism of left~\dash{$G$}{sets}, i.e.\ it holds for all~$g, x \in G$ that~$\varphi(g x) = g \varphi(x)$.
  Then there exists a unique element~$h \in G$ such that~$\varphi$ is given by right multiplication with~$h$, and~$h$ is given by~$h = \varphi(1)$.
\end{lemma}


\begin{proof}
  The uniqueness of~$h$ follows from~$\varphi(1) = 1 \cdot h = h$ and the existence follows from
  \[
      \varphi(g)
    = \varphi(g \cdot 1)
    = g \cdot \varphi(1)
    = g \cdot h \,,
  \]
  as claimed.
\end{proof}


\begin{theorem}
  \label{abstract jcd}
  Let~$G$ be a linear algebraic group.
  \begin{enumerate}
    \item
      There exists for every~$g \in G$ unique elements~$g_s, g_u \in G$ with~$(\rho_g)_s = \rho_{g_s}$ and~$(\rho_g)_u = \rho_{g_u}$.
    \item
      It holds that~$g = g_s g_u$ and the elements~$g_s$ and~$g_u$ commute.
  \end{enumerate}
\end{theorem}


\begin{proof}
  \leavevmode
  \begin{enumerate}
    \item
      Let~$\mu \colon \coord(G) \tensor \coord(G) \to \coord(G)$ be the multiplication map.
      The map~$\rho_g$ is an algebra automorphism so that
      \[
          \mu \circ ( \rho_g \tensor \rho_g )
        = \rho_g \circ \mu \,.
      \]
      It follows from \cref{properties of local mjcd} that also
      \[
          \mu \circ ( \rho_g \tensor \rho_g )_s
        = ( \rho_g )_s \circ \mu
      \]
      and by \cref{properties of tensor product of local endomorphisms} therefore that
      \[
          \mu \circ \left( (\rho_g)_s \tensor (\rho_g)_s \right)
        = ( \rho_g )_s \circ \mu \,.
      \]
      This shows that~$(\rho_g)_s$ is again an algebra automorphism.% don’t let footnote see this line break
      \footnote{That~$(\rho_g)_s$ preserves the unit~$1$ follows from~$(\rho_g)_s$ being additive, multiplicative and bijective, and thus an isomorphism of not\nobreakdash-necessarily\nobreakdash-commutative rings.}
      It follows that there exists a unique homomorphism of affine varieties~$\varphi \colon G \to G$ with~$(\rho_g)_s = \varphi^*$.
      
      The map~$\varphi$ is already a homomorphism of left~\dash{$G$}{sets}:
      For every~$h \in G$ let~$l_h \colon G \to G$ denote the left multiplication with~$h$.
      Then~$l_h$ commutes with~$r_g$ for every~$h \in G$, from which it follows that $\lambda_h = l_h^*$ commutes with~$\rho_g = r_g^*$ for every~$h \in G$.
      It then follows that~$\lambda_h$ also commutes with the semisimple part~$(\rho_g)_s$ for every~$h \in G$, from which it then follows that~$\varphi$ commutes with~$l_h$ for every~$h \in G$.
      
      It follows from \cref{homomorphisms of Gsets} that there exists a unique element~$g_s \in G$ with~$\varphi = r_{g_s}$, which is equivalent to~$(\rho_g)_s = \varphi^* = r_{g_s}^* = \rho_{g_s}$.
      
      It follows in the same way that there exists a unique element~$g_u \in G$ with~$(\rho_g)_u = \rho_{g_u}$.
    \item
      It holds that
      \[
          \rho_{g_s g_u}
        = r_{g_s g_u}^*
        = (r_{g_u} r_{g_s})^*
        = r_{g_s}^* r_{g_u}^*
        = \rho_{g_s} \, \rho_{g_u}
        = (\rho_g)_s (\rho_g)_u
        = \rho_{g} \,,
      \]
      therefore~$r_{g_s g_u} = r_g$ and thus~$g = g_s g_u$.
      It also follows in the same way that~$g = g_u g_s$.
  \end{enumerate}
\end{proof}


\begin{definition}
  \label{abstract jcd definition}
  Let~$G$ be a linear algebraic group and let~$g \in G$
  The decomposition~$g = g_s g_u$ from \cref{abstract jcd} is the \emph{{\JCD}}\index{Jordan--Chevalley decomposition!for elements of a lin.\ alg.\ group} of~$g$.
  The factor~$g_s$ is the \emph{semisimple part}\index{semisimple!part of!an element of a lin.\ alg.\ group} of~$g$ and the factor~$g_u$ is the \emph{unipotent part}\index{unipotent!part of!an element of a lin.\ alg.\ group} of~$g$.
\end{definition}


\begin{proposition}
  For~$g \in \GL(V)$ the \enquote{abstract} {\JCD} of \cref{abstract jcd definition} coincides with the \enquote{concrete} one from \cref{mjcd definition}.
\end{proposition}


\begin{proof}
  Let~$G \defined \GL(V)$.
  
  We start by embedding~$V$ into~$\coord(G)$.
  Let~$f \in V^*$ be nonzero and consider the~\dash{$k$}{linear} map
  \[
            \varphi
    \colon  V
    \to     \coord(G) \,,
    \quad   v
    \mapsto [g \mapsto f(gv)] \,.
  \]
  
  This map is injective:
  There exists some~$v' \in V$ with~$f(v') \neq 0$.
  Because~$G = \GL(V)$ acts transitively on~$V \setminus \{0\}$ it follows that there exists for every~$v \in V$ with~$v \neq 0$ some~$g \in \GL(V)$ with~$gv = v'$;
  it then holds that
  \[
          \varphi(v)(g)
    =     f(gv)
    =     f(v')
    \neq  0
  \]
  and therefore that~$\varphi(v) \neq 0$.

  The map~$\varphi$ is also~\dash{$G$}{equivariant} because it holds for all~$g,h \in G$, and~$v \in V$ that
  \[
      \varphi(gv)(h)
    = f(hgv)
    = \varphi(v)(hg)
    =\rho_g(\varphi(v))(h)
  \]
  and therefore that~$\varphi(gv) = \rho_g(\varphi(v))$.      
  Together this shows that~$\varphi$ is an embedding of representations of~$G$.
  Let~$W \defined \varphi(V)$.
    
  Let~$g = g_s g_u$ be the \enquote{abstract} {\JCD} of~$g \in G$.
  Then~$\rho_g = \rho_{g_s} \, \rho_{g_u}$ is the {\JCD} of~$\rho_g$.
  It follows that~$\restrict{\rho_g}{W} = (\restrict{\rho_{g_s}}{W}) (\restrict{\rho_{g_u}}{W})$ is the {\JCD} of~$\restrict{\rho_g}{W}$, so that~$g_s$ acts semisimple on~$W$ and~$g_u$ acts unipotent on~$W$.
  It follows from~$\varphi$ being an isomorphism~$V \to W$ of representations of~$G$ that~$g_s$ also acts semisimple on~$V$ and that~$g_u$ acts unipotent on~$V$.
  But this means that~$g_s$ is semisimple as an endomorphism~$V \to V$ and that~$g_u$ is unipotent as an endomorphism~$V \to V$.
  The decomposition~$g = g_s g_u$ is therefore also the \enquote{concrete} {\JCD} of~$g$.
\end{proof}


\begin{lemma}[Functoriality of the {\JCD}]
  Let~$G$ and~$H$ be linear algebraic groups and let~$f \colon G \to H$ be a homomorphism of linear algebraic groups.
  Then
  \[
      f(g)_s
    = f(g_s)
    \quad\text{and}\quad
      f(g)_u
    = f(g_u)
  \]
  for every~$g \in G$.
\end{lemma}


\begin{proof}
  For every~$g \in G$ the diagram
  \[
    \begin{tikzcd}
        H
        \arrow{r}[above]{r_{f(g)}}
      & H
      \\
        G
        \arrow{r}[above]{r_g}
        \arrow{u}[left]{f}
      & G
        \arrow{u}[right]{f}
    \end{tikzcd}
  \]
  commutes, which by dualizing results in the following commutative diagram:
  \[
    \begin{tikzcd}
        \coord(H)
        \arrow{d}[left]{f^*}
      & \coord(H)
        \arrow{l}[above]{\rho_{f(g)}}
        \arrow{d}[right]{f^*}
      \\
        \coord(G)
      & \coord(G)
        \arrow{l}[above]{\rho_g}
    \end{tikzcd}
  \]
  It follows from \cref{properties of local mjcd} that the diagram
  \[
    \begin{tikzcd}[column sep = large]
        \coord(H)
        \arrow{d}[left]{f^*}
      & \coord(H)
        \arrow{l}[above]{ (\rho_{f(g)})_s }
        \arrow{d}[right]{f^*}
      \\
        \coord(G)
      & \coord(G)
        \arrow{l}[above]{ (\rho_g)_s }
    \end{tikzcd}
  \]
  also commutes, which is by construction of the semisimple parts~$f(g)_s$ and~$g_s$ the same as the following diagram:
  \begin{equation}
    \label{comm diagram for coordinate rings}
    \begin{tikzcd}[column sep = large]
        \coord(H)
        \arrow{d}[left]{f^*}
      & \coord(H)
        \arrow{l}[above]{ \rho_{f(g)_s} }
        \arrow{d}[right]{f^*}
      \\
        \coord(G)
      & \coord(G)
        \arrow{l}[above]{ \rho_{g_s} }
    \end{tikzcd}
  \end{equation}
  This is the dual of the following diagram:
  \begin{equation}
    \label{comm diagram for groups}
    \begin{tikzcd}
        H
        \arrow{r}[above]{r_{f(g)_s}}
      & H
      \\
        G
        \arrow{u}[left]{f}
        \arrow{r}[above]{r_{g_s}}
      & G
        \arrow{u}[right]{f}
    \end{tikzcd}
  \end{equation}
  It follows from the commutativity of the diagram~\eqref{comm diagram for coordinate rings} that the diagram~\eqref{comm diagram for groups} commutes (because the functor~$\coord(-)$ is faithful).
  By evaluting the identity~$f \circ r_{g_s} = r_{f(g)_s} \circ f$ at the identity~$1 \in G$ it follows that
  \[
      f(g_s)
    = f(1 \cdot g_s)
    = \left( f \circ r_{g_s} \right)(1)
    = \left( r_{f(g)_s} \circ f \right)(1)
    = f(1) \cdot f(g)_s
    = 1 \cdot f(g)_s
    = f(g)_s \,.
  \]
  The equality~$f(g_u) = f(g)_u$ can be shown in the same way.
\end{proof}


\begin{corollary}
  \label{jcd independent of embedding}
  Let~$V$ be a \dash{finite}{dimensional}~\dash{$k$}{vector space}.
  \begin{enumerate}
    \item
      If~$G \colon G \inclusion \GL(V)$ is an embedding of linear algebraic groups then the {\JCD} in~$G$ coincides with the \enquote{concrete} {\JCD} (from \cref{mjcd definition}) in~$\GL(V)$.
    \item
      If~$G \subseteq \GL(V)$ is a closed subgroup then the {\JCD} in~$G$ coincides with the \enquote{concrete} {\JCD} in~$\GL(V)$.
    \qedhere
  \end{enumerate}
\end{corollary}


\begin{lemma}
  Let~$G$ be a linear algebraic group and let~$g_1, g_2 \in G$ be two group elements which commute with each other.
  Then
  \[
      (g_1 g_2)_s
    = (g_1)_s (g_2)_s
    \quad\text{and}\quad
      (g_1 g_2)_u
    = (g_1)_u (g_2)_u \,.
  \]
\end{lemma}


\begin{proof}
  We may assume that~$G$ is a closed subgroup~$\GL(V)$ for some \dash{finite}{dimensional}~\dash{$k$}{vector space}~$V$.
  The claim then follows from \cref{local mjcd for commuting endomorphisms} thanks to \cref{jcd independent of embedding}.
\end{proof}




