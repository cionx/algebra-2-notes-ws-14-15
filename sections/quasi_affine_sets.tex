\section{Quasi-Affine Sets}





\subsection{Definition}


\begin{definition}
  If~$X$ is an affine set and~$X' \subseteq X$ is a Zariski open subset then~$X'$ is a \emph{{\qaffine} \textup(algebraic\textup) set}.
  The \emph{Zariski topology} on~$X'$ is the subspace topology induced by the Zariski topology on~$X$.
\end{definition}


\begin{remark}
  A subset~$X \subseteq \Aff^n$ is a {\qaffine} set if and only if it is of the form~$X = C \cap U$ for some Zariski closed subset~$C \subseteq \Aff^n$ and Zariski open subset~$U \subseteq \Aff^n$.
\end{remark}


\begin{corollary}
  Let~$X, X_1, \dotsc, X_n \subseteq \Aff^n$ be {\qaffine} sets.
  \begin{enumerate}
    \item
      Every Zariski open subsets of~$X$ is again a {\qaffine} set.
    \item
      Every Zariski closed subsets of~$X$ is again a {\qaffine} set.
    \item
      The finite intersection~$X_1 \cap \dotsb \cap X_n$ is again a {\qaffine} set.
  \end{enumerate}
\end{corollary}


\begin{proof}
  Let~$C, C_1, \dotsc, C_n \subseteq \Aff^n$ be Zariski closed and~$U, U_1, \dotsc, U_n \subseteq \Aff^n$ Zariski open such that~$X = C \cap U$ and~$X_i = C_i \cap U_i$ for every~$i$.
  \begin{enumerate}
    \item
      If~$V \subseteq X$ is Zariski open then there exists a Zariski open subset~$V' \subseteq \Aff^n$ with~$V = V' \cap X$.
      It then follows that
      \[
          V
        = V' \cap X
        = V' \cap C \cap U
        = C \cap (V' \cap U)
      \]
      with~$V' \cap U \subseteq \Aff^n$ being Zariski open.
    \item
      This can be shown in the same way as above.
    \item
      It follows that
      \[
          X_1 \cap \dotsb \cap X_n
        = (C_1 \cap \dotsb \cap C_n) \cap (U_1 \cap \dotsb \cap U_n)
      \]
      with~$C_1 \cap \dotsb \cap C_n \subseteq \Aff^n$ being Zariski closed and~$U_1 \cap \dotsb \cap U_n \subseteq \Aff^n$ being Zariski open.
    \qedhere
  \end{enumerate}
\end{proof}


\begin{example}
  \leavevmode
  \begin{enumerate}
    \item
      Every affine set is a {\qaffine} set.
    \item
      If~$X$ is an affine set then~$\Dopen_X(f)$ is a {\qaffine} set for every~$f \in \coord(X)$.
    \item
      It follows from the previous example with~$X = \Aff^{n^2}$ and~$f = \det$ that~$\GL_n(k) = \Dopen(\det)$ is a {\qaffine} set.
  \end{enumerate}
\end{example}





\subsection{Morphisms of Quasi-Affine Sets}


\begin{definition}
  \label{regular for quasiaffine}
  Let~$X$ be a {\qaffine} set and let~$f \colon X \to k$ be a function.
  \begin{enumerate}
    \item
      The function~$f$ is \emph{regular at~$x \in X$} if~$f$ is a rational function in some neighbourhood of~$x$, i.e.\ if there exist~$U \subseteq X$ open and~$g, h \in \coord(X)$ such that~$x \in U$,~$h(y) \neq 0$ for every~$y \in U$ and~$f(y) = g(y)/h(y)$ for all~$y \in U$.
    \item
      The function~$f$ is \emph{regular} if it is regular at every point~$x \in X$.
      The set of all rational functions~$X \to k$ is denoted by~$\rational(X)$.
  \end{enumerate}
\end{definition}


\begin{notation}
  If~$X$ is a set,~$U \subseteq X$ is a subset and~$f, f', g, g' \colon X \to k$ are functions with~$g(x), g'(x) \neq 0$ for every~$x \in U$ and~$f(x)/g(x) = f'(x)/g'(x)$ for all~$x \in U$ then we say that~$f/g \equiv f'/g'$~on~$U$.
\end{notation}


\begin{proposition}
  \label{regular on affine is polynomial}
  Let~$X$ be an affine set and let~$f \colon X \to k$ be a regular function in the sense of \cref{regular for quasiaffine}.
  The map~$f$ is then already a polynomial, and thus regular in the sense of \cref{regular for affine}.
  Thus both definitions agree for affine sets, and~$\rational(X) = \coord(X)$.
\end{proposition}


\begin{proof}
  There exists an open cover~$(U_i)_{i \in I}$ of~$X$ such that~$f$ is given on each~$U_i$ by a rational function~$f_i/g_i$ for suitable~$f_i, g_i \in \coord(X)$.
  We may assume that each~$U_i$ is of the form~$U_i = \Dopen(h_i)$ for some~$h_i \in \coord(X)$ as these sets form a basis for the Zariski topology on~$X$.
  
  It follows for all~$i,j \in I$ that~$f_i/g_i \equiv f_j/g_j$ on~$U_i \cap U_j$, and therefore that~$f_i g_j \equiv f_j g_i$ on~$U_i \cap U_j$.
  The set~$U_i \cap U_j$ is given by
  \[
    U_i \cap U_j = \Dopen(h_i) \cap \Dopen(h_j) = \Dopen(h_i h_j) \,.
  \]
  It follows that by multiplying the above equality with~$h_i h_j$ we arrive at the equality
  \begin{equation}
  \label{extended equality on intersections}
      h_i h_j f_i g_j
    = h_i h_j f_j g_i \,,
  \end{equation}
  which holds both on~$U_i \cap U_j = \Dopen(h_i hj)$ and on~$X \setminus (U_i \cap U_j) = \vset(h_i h_j)$, i.e.\ on the whole of~$X$.
  
  We may assume that~$h_i = g_i$:
  It holds that~$\Dopen(h_i) \subseteq \Dopen(g_i)$ because the rational function~$f_i/g_i$ is defined on~$U_i = \Dopen(h_i)$.
  It follows from \cref{containment of D} that~$h_i \in \rad{\genideal{g_i}}$ and therefore~$h_i = a g_i^n$ for some~$a \in \coord(X)$,~$n \geq 0$.
  It follows that~$a(x) \neq 0$ forevery~$x \in U_i = \Dopen(h_i)$ and therefore 
  \[
            \frac{f_i}{g_i}
    \equiv  \frac{a f_i g_i^{n-1}}{a g_i^n}
    \equiv  \frac{a f_i g_i^{n-1}}{h_i}
  \]
  on~$U_i$.
  By replacing~$f_i$ with~$a f_i g_i^{n-1}$ and~$g_i$ with~$h_i$ the claim follows.
  Note that \Cref{extended equality on intersections} can now be rewritten as
  \begin{equation}
  \label{equality on intersections}
        f_i g_j \cdot g_i g_j
      = f_j g_i \cdot g_i g_j \,.
  \end{equation}
  So we may swap the indices in the term~$f_i g_j$ if the factor~$g_i g_j$ is present.
  
  The denominators~$g_i$,~$i \in I$ have no common zeros because
  \[
              \vset(g_i \suchthat i \in I)
    =         \bigcap_{i \in I} \vset(g_i)
    =         X \setminus \bigcup_{i \in I} \Dopen(g_i)
    =         X \setminus X
    =         \emptyset \,,
  \]
  The squares~$g_i^2$,~$i \in I$ do therefore also have no common zeroes (because~$g_i$ and~$g_i^2$ have the same zeroes).
  It follows from \hyperref[nullstellensatz 2]{Hilbert’s~Nullstellensatz} that there exists a linear combination
  \begin{equation}
  \label{unit as linear combination}
      1
    = \sum_{i \in I} a_i g_i^2
  \end{equation}
  for suitable~$a_i \in \coord(X)$.
  By using \Cref{equality on intersections} and \Cref{unit as linear combination} it follows that it holds on~$U_j$ that
  \[
            f
    \equiv  \frac{f_j}{g_j}
    \equiv  \frac{f_j}{g_j} \sum_{i \in I} a_i g_i^2
    \equiv  \sum_{i \in I} \frac{a_i f_j g_i^2}{g_j}
    \equiv  \sum_{i \in I} \frac{a_i f_j g_i^2 g_j}{g_j^2}
    \equiv  \sum_{i \in I} \frac{a_i f_i g_j^2 g_i}{g_j^2}
    \equiv  \sum_{i \in I} a_i f_i g_i \,.
  \]
  This shows that~$f \equiv \sum_{i \in I} a_i f_i g_i$ on~$U_j$ for every~$j \in I$ and therefore~$f = \sum_{i \in I} a_i f_i g_i$.
\end{proof}


% TODO: Understand this proof better.


\begin{definition}
  \label{regular of qaffine}
  Let~$X,Y$ be {\qaffine} sets.
  \begin{enumerate}[resume]
    \item
      A map~$f \colon X \to \Aff^m$ is \emph{regular} if it is regular in each coordinate.
    \item
      A map~$f \colon X \to Y$ is \emph{regular} if it is the restriction of a regular map~$X \to \Aff^m$.
  \end{enumerate}
  A map~$X \to Y$ is a \emph{morphism} of {\qaffine} sets if it is regular.
\end{definition}


\begin{remark}
  \Cref{regular on affine is polynomial} shows that \cref{regular of qaffine} agrees with \cref{regular for affine} for affine sets.
\end{remark}


\begin{lemma}
  Let~$X \subseteq X' \subseteq \Aff^n$ be a {\qaffine} set with~$X'$ an affine set.
  \begin{enumerate}
    \item
      The sets~$\Dopen_X(f)$ with~$f \in \coord(X')$ are a basis of the Zariski topology on~$X$.
    \item
      The sets~$\Dopen_X(f)$ with~$f \in \rational(X)$ are open in~$X$, and also form a basis of the Zariski topology on~$X$.
  \end{enumerate}
\end{lemma}


\begin{proof}
  \leavevmode
  \begin{enumerate}
    \item
      This follows from the fact that the sets~$\Dopen_{X'}(f)$ with~$f \in \coord(X')$ are a basis for the Zariski topology on~$X'$ and that~$\Dopen_X(f) = \Dopen_{X'}(f) \cap X$ for every~$f \in \coord(X)$.
    \item
      It remains to show that every~$f \in \rational(X)$ is continuous.
      If~$f$ is a globally defined rational function given by~$f = g/h$ for~$f, g \in \Aff(X')$ then~$\Dopen_X(f) = \Dopen_X(g)$ is open.
      
      If more generally~$f$ is regular then there exists an open cover~$(U_i)_{i \in I}$ of~$X$ such that~$\restrict{f}{U_i}$ is rational for every~$i \in I$.
      It then follows by the above that~$\restrict{f}{U_i}$ is continuous for every~$i \in I$ (where we use that~$U_i$ is again {\qaffine} and~$\restrict{f}{U_i}$ is again regular) and therefore that~$f$ is continuous (because continuity is a local property).
    \qedhere
  \end{enumerate}
\end{proof}


\begin{corollary}
  Let~$X,Y$ be {\qaffine} sets and let~$f \colon X \to Y$ be a morphism of {\qaffine} sets.
\ \begin{enumerate}
    \item
      It holds for every~$\varphi \in \rational(Y)$ that~$f^{-1}(\Dopen_Y(\varphi)) = \Dopen_X(\varphi \circ f)$.
    \item
      The map~$f$ is continuous with respect to the Zariski topologies on~$X,Y$.
    \qed
  \end{enumerate}
\end{corollary}


\begin{lemma}
  Let~$X, Y, Z$ be {\qaffine} sets.
  \begin{enumerate}
    \item
      The identity map~$\id_X \colon X \to X$ is a morphism.
    \item
      For every two morphisms~$f \colon X \to Y$ and~$g \colon Y \to Z$ their composition~$g \circ f \colon X \to Z$ is again a morphism.
  \end{enumerate}
\end{lemma}


\begin{proof}
  \leavevmode
  \begin{enumerate}[start=2]
    \item
      We need to show that~$g \circ f$ is regular at every point~$x \in X$.
      It follows from the regularity of~$g$ that there exists an open neighbourhood~$V \subseteq Y$ of~$f(x)$ on which~$g$ is given by a rational function~$g_1/g_2$.
      It follows from the continuity of~$f$ that there exists an open neighbourhood~$U$ of~$x$ with~$f(U) \subseteq V$.
      By using the regularity of~$f$ and shrinkening~$V$ if necessary we may assume that~$f$ is given by a rational function~$f_1/f_2$ on~$U$.
      It then follows that
      \[
          (g \circ f)(x)
        = \frac{ g_1\left( \frac{f_1(x)}{f_2(x)} \right) }{ g_2\left( \frac{f_1(x)}{f_2(x)} \right) }
      \]
      for every~$x \in U$, which shows that~$g \circ f$ is given by a rational function on~$U$.
      (Recall that the composition of rational functions is again rational.)
    \qedhere
  \end{enumerate}
\end{proof}



\begin{lemma}
  Let~$X, Y, Z$ be {\qaffine} sets
  \begin{enumerate}
    \item
      If~$f \colon X \to Y$ is a morphism of {\qaffine} sets then the map
      \[
                  f^*
        \colon    \rational(Y)
        \to       \rational(X),
        \quad     \varphi
        \mapsto   \varphi \circ f
      \]
      is a homomorphism of~\dash{$k$}{algebras}.
    \item
      It holds that~$\id_X^* = \id_{\rational(X)}$.
    \item
      If~$f \colon X \to Y$,~$g \colon Y \to Z$ are morphisms of {\qaffine} sets then~$(g \circ f)^* = f^* \circ g^*$.
    \qed
  \end{enumerate}
\end{lemma}




\subsection{Products of Quasi-Affine Sets}


\begin{lemma}
  If~$X \subseteq \Aff^n$ and~$Y \subseteq \Aff^m$ are {\qaffine} sets then the set~$X \times Y \subseteq \Aff^{n+m}$ is again {\qaffine}.
\end{lemma}


\begin{proof}
  Let~$X' \subseteq \Aff^n$ and~$Y' \subseteq \Aff^m$ be affine sets such that~$X \subseteq X'$ and~$Y \subseteq Y'$ are open subsets.
  Then~$X \times Y$ is an open subset of the affine set~$X' \times Y' \subseteq \Aff^{n+m}$ because the Zariski topology on~$X' \times Y'$ is finer than the product topology.
\end{proof}



\begin{definition}
  For {\qaffine} sets~$X \subseteq \Aff^n$,~$Y \subseteq \Aff^m$ their \emph{product} is the {\qaffine} set~$X \times Y \subseteq \Aff^{n+m}$.
\end{definition}


\begin{fluff}
  Note that for affine sets~$X, Y$ their product as affine sets is the same as their product of {\qaffine} sets.
  We therefore do not need to specify which kind of product of we are talking about when dealing with affine sets.
\end{fluff}


\begin{lemma}
  Let~$X, X_1, X_2, X_2, Y_1, Y_2$ be {\qaffine} sets.
  \begin{enumerate}
    \item
      The projections~$\pi_i \colon X_1 \times X_2 \to X_i$ are morphisms of {\qaffine} sets.
    \item
      A map~$f \colon X \to Y_1 \times Y_2$ given by~$f = (f_1, f_2)$ with~$f_i \colon X \to Y_i$ is a morphism of {\qaffine} sets if and only if both~$f_1, f_2$ are morphisms of {\qaffine} sets.
  \end{enumerate}
  This shows that the product of two {\qaffine} sets is their categorical product in the category of {\qaffine} sets.
  \begin{enumerate}[resume]
    \item
      If~$f \colon X \to X'$ and~$g \colon Y \to Y'$ are to morphisms of {\qaffine} sets then the induced map~$f \times g \colon X \times Y \to X' \times Y'$ is again a morphism of {\qaffine} sets.
    \qed
  \end{enumerate}
\end{lemma}





\subsection{Affine Varieties}


\begin{definition}
  An \emph{affine~variety} is a {\qaffine} set which is isomorphic to an affine set (as a {\qaffine} set).
  A \emph{{\qaffine}~variety} is just a {\qaffine} set.% make footnote not nice the line break
  \footnote{We will later on extend this notions to projective and {\qprojective} sets.}
\end{definition}


\begin{notation}
  \leavevmode
  \begin{enumerate}
    \item
      We often just say than a {\qaffine} variety is \emph{affine} to mean that it is an quasi-affine variety.
      Note that this does not necessarily mean that it is affine as a set!
    \item
      For an affine variety~$X$ we often write~$\coord(X) \defined \rational(X)$.
      \Cref{regular on affine is polynomial} shows that this is well-defined when~$X$ is an affine set.
  \end{enumerate}
\end{notation}


\begin{example}
  \leavevmode
  \begin{enumerate}
    \item
      Every affine set is an affine variety.
    \item
      Let~$X$ be an affine variety and let~$f \in \coord(X)$.
      Then the open subset~$\Dopen_X(f)$ is again an affine variety:
      
      We may assume that~$X \subseteq \Aff^n$ is an affine set.
      For the set
      \[
          Y
        = \left\{
                (x, t)
            \in \Aff^n \times \Aff^1
            =   \Aff^{n+1}
          \suchthat*
            x \in X,
            x t = 1
          \right\}
      \]
      the map
      \[
                \varphi
        \colon  \Dopen_X(f)
        \to     Y,
        \quad   x
        \mapsto \left( x, \frac{1}{f(x)} \right)
      \]
      is a bijection.
      If~$X$ is cut out by some ideal~$I \idealleq k[x_1, \dotsc, x_n]$ then~$Y$ is cut out by the ideal~$I$ together with the polynomial~$f x_{n+1} - 1$, which shows that~$Y$ is again an affine set.
      The map~$\varphi$ is rational in each coordinate and therefore a morphism.
      The inverse of~$\varphi$ is given by projection onto the first~$n$-th coordinates, which is also a morphism.
      This shows that~$\varphi$ is an isomorphism, which shows that claim.
      
      It follows in particular that the isomorphism~$\varphi$ of affine varieties induces an isomorphism of \dash{$k$}{algebras}~$\varphi^* \colon \coord(Y) \to \coord(\Dopen_X(f))$.
      It follows that
      \begin{align*}
                \coord(\Dopen_X(f))
        &\cong  \coord(Y) \\
        &=      k[x_1, \dotsc, x_n, x_{n+1}]/(I, f x_{n+1} - 1) \\
        &\cong  ( k[x_1, \dotsc, x_n]/I )[x_{n+1}]/(f x_{n+1} - 1) \\
        &=      \coord(X)[x_{n+1}]/(f x_{n+1} - 1) \\
        &\cong  \coord(X)[f^{-1}]
      \end{align*}
      is the localization of~$\coord(X)$ at~$f$.
      This shows that every regular function on~$\Dopen_X(f)$ is of the form~$g/f^n$ for some~$g \in \coord(X)$ and~$n \geq 0$.
    \item
      As an instance of the previous example we find that~$\GL_n(k) = \Dopen(\det) \subseteq \Aff^{n^2}$ is an affine variety with
      \[
          \coord(\GL_n(k))
        = \coord\left( \Aff^{n^2} \right)\left[ {\det}^{-1} \right]
        = k\left[ x_{11}, \dotsc, x_{nn}, {\det}^{-1} \right] \,.
      \]
      As an example we have for~$n = 2$ that
      \[
          \coord(\GL_2(k))
        = k\left[a, b, c, d, \frac{1}{ad-bc}\right] \,.
      \]
  \end{enumerate}
\end{example}


\begin{lemma}
  Let~$X, X_1, X_2$ be affine varieties.
  \begin{enumerate}
    \item
      Every closed subset~$C \subseteq X$ is again an affine variety.
    \item
      If~$X_1$ and~$X_2$ are both affine 
  \end{enumerate}
\end{lemma}


\begin{lemma}
  \label{coordinate ring of product of qaffine}
  For affine varieties~$X,Y$ the map
  \[
            \coord(X) \otimes_k \coord(Y)
    \to     \coord(X \times Y) \,,
    \quad   f \otimes g
    \mapsto [(x, y) \mapsto f(x) g(y)]
  \]
  is a \dash{well}{defined} natural isomorphism of~\dash{$k$}{algebras}.
\end{lemma}


\begin{proof}
  To see that the proposed map~$\Phi = \Phi_{X,Y}$ is \dash{well}{defined} let~$f \in \coord(X)$ and~$g \in \coord(Y)$.
  We need to show that~$\Phi(f \otimes g)$ is regular at every point~$(x,y) \in X \times Y$.
  There exist a neighbourhood~$U \subseteq X$ of~$x$ on which~$f$ is given by a rational function~$f_1/f_2$, and similarly a neighbourhood~$V \subseteq Y$ of~$y$ on which~$g$ is given by a rational function~$g_1/g_2$.
  It follows that~$U \times V$ is an open neighbourhood of~$(x,y)$ in~$X \times Y$ because the Zariski topology on~$X \times Y$ is finer than the product topology.
  In this neighbourhood the map~$\Phi(f \otimes g)$ is given by the rational function~$(f_1 g_1) / (f_2 g_2)$.
  This shows that~$\Phi(f \otimes g)$ is regular at~$(x,y)$.
  
  To show that naturality of~$\Phi$ let~$X', Y'$ be another pair of affine varieties and consider a pair of morphisms~$\varphi \colon X \to X'$,~$\psi \colon Y \to Y'$.
  For the naturialty of~$\Phi$ we need the diagram
  \begin{equation}
  \label{naturality for product of coordinate rings}
    \begin{tikzcd}[sep = large]
        \coord(X) \otimes_k \coord(Y)
        \arrow{r}[above]{\Phi_{X,Y}}
      & \coord(X \times Y)
      \\
        \coord(X') \otimes_k \coord(Y')
        \arrow{u}[left]{\varphi^* \otimes \psi^*}
        \arrow{r}[above]{\Phi_{X',Y'}}
      & \coord(X' \times Y')
        \arrow{u}[right]{(\varphi \times \psi)^*}
    \end{tikzcd}
  \end{equation}
  to commutes.
  This holds because
  \begin{align*}
        \Phi_{X,Y}( (\varphi^* \times \psi^*)(f \otimes g) )(x,y)
    &=  \Phi_{X,Y}( \varphi^*(f) \otimes \psi^*(g) )(x,y)         \\
    &=  \varphi^*(f)(x) \psi^*(g)(y)                              \\
    &=  (f \circ \varphi)(x) (g \circ \psi)(y)                    \\
    &=  f(\varphi(x)) g(\psi(y))                                  \\
    &=  \Phi_{X',Y'}(f \otimes g)(\varphi(x), \psi(y))            \\
    &=  \Phi_{X',Y'}(f \otimes g)( (\varphi \times \psi)(x,y) )   \\
    &=  (\varphi \times \psi)^*( \Phi_{X',Y'}(f \otimes g) )(x,y)
  \end{align*}
  for all~$(x,y) \in X \times Y$, and therefore
  \[
      \Phi_{X,Y}( (\varphi^* \otimes \psi^*)(f \otimes g) )
    = (\varphi \times \psi)^*( \Phi_{X',Y'}(f \otimes g ) )
  \]
  for every simple tensor~$f \otimes g \in \coord(X') \otimes \coord(Y')$, and thus overall
  \[
      \Phi_{X,Y} \circ (\varphi^* \otimes \psi^*)
    = (\varphi \times \psi)^* \circ \Phi_{X',Y'} \,.
  \]

  
  To show that~$\Phi$ is an isomorphism let~$X' \subseteq \Aff^n$ and~$Y' \subseteq \Aff^m$ be affine sets for which there exists isomorphisms~$\varphi \colon X \to X'$ and~$\psi \colon Y \to Y'$.
  Then in the resulting commutative diagram~\eqref{naturality for product of coordinate rings} the vertical arrows are both isomorphisms and the lower horizontial map~$\Phi_{X', Y'}$ is an isomorphism by~\Cref{regular functions on product of affine sets}.
  It follows that the upper horizontal arrow, which can be expressed as
  \begin{equation}
  \label{isomorphism for affine varieties}
      \Phi_{X,Y}
    = (\varphi \times \psi)^* \circ \Phi_{X',Y'} \circ ( \varphi^* \otimes \psi^* )^{-1} \,,
  \end{equation}
  is also an isomomorphism.
\end{proof}


\begin{remark}
  That~$\Phi_{X,Y}$ is well-defined for affine varieties~$X,Y$ also follows from~\Cref{isomorphism for affine varieties}.
  The above argumentation actually shows that~$\Phi_{X,Y}$ is well-defined whenever~$X,Y$ are {\qaffine} varietes, and it is explained in~\cite{MO267198} that~$\Phi_{X,Y}$ is then again an isomorphism.
  (One uses that that the isomorphism holds for affine varieties, and then uses that {\qaffine} varities are \enquote{locally affine} and that regularity is a local condition.)
% TODO: Add a proper explanation how and why quasi-affine varietes are locally affine.
\end{remark}


\begin{metaremark}
  The notion of {\qaffine} sets and the above generalization of affine sets to affine varieties was originally not given in the lecture.
  {\Qaffine} sets were only introduced much later on together with projective and {\qprojective} sets.
  We have choosen to include these concept earlier on so that~$\GL_n(k)$ becomes an affine variety.
  This places the upcoming discussion of affine algebraic groups on a more formally solid foundation.
\end{metaremark}


% TODO: Calculate some induced morphisms used for algebraic groups










