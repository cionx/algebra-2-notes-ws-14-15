\section{Diagonalizable Linear Algebraic Groups}


\begin{definition}
  A linear algebraic group~$G$ is \emph{diagonalizable}\index{diagonalizable lin.\ alg.\ group} if it is isomorphic to a closed subgroup of some~$\Diag_n(k) \cong \Gmult^n$.
  It is a torus if~$G \cong \Diag_n(k) \cong \Gmult^n$ for some~$n$.
\end{definition}


\begin{lemma}
  For a linear algebraic group~$G$ the following conditions are equivalent:
  \begin{enumerate}
    \item
      \label{diagbar}
      $G$ is a diagonalizable.
    \item
      \label{commutative and consists of ss}
      $G$ is commutative with~$G = G_s$.
  \end{enumerate}
\end{lemma}


\begin{proof}
  \leavevmode
  \begin{description}
    \item[\ref*{diagbar}~$\iff$~\ref*{commutative and consists of ss}:]
      The group~$\Diag_n(k)$ is commutative and every element of~$\Diag_n(k)$ is semisimple, so the same follows for every of its closed subgroups.
    \item[\ref*{commutative and consists of ss}~$\iff$~\ref*{diagbar}:]
      This follows from \cref{embedding for comm lag}.
    \qedhere
  \end{description}
\end{proof}


\begin{definition}
  A linear algebraic group~$G$ is a \emph{torus}\index{torus} if it is isomorphic to~$\Diag_n(k) \cong \Gmult^n$ for some~$n$.
\end{definition}


\begin{fluff}
  We will show that tori and diagonalizable linear algebraic groups can be characterized via their \emph{character group}.
\end{fluff}


\begin{definition}
  A \emph{character}\index{character!of a group}\index{group!character} of a group~$G$ (over~$k$) is a group homomorphism~$\chi \colon G \to k^\times$.
\end{definition}


\begin{lemma}[Dedekind\nobreakdash--Artin]
  \label{dedekind artin lemma}
  For any group~$G$ the set of characters~$G \to k$ is linearly independent in~$\Maps(G,k)$.
\end{lemma}


\begin{proof}
  We show that pairwise different characters~$\chi_1, \dotsc, \chi_n \colon G \to k^\times$ are linearly independent by induction over~$n$.
  The linear independence holds for~$n = 0$ and it holds for~$n = 1$ because every character is nonzero (as it maps~$1 \in G$ onto~$1 \in k^\times$).
  
  Let~$n \geq 2$ and let~$\lambda_1 \chi_1 + \dotsb + \lambda_n \chi_n = 0$ be a linear combination.
  It holds for all~$g, h \in G$ that
  \begin{gather}
    \notag
      0
    = \lambda_1 \chi_1(gh) + \dotsb + \lambda_n \chi_n(gh)
    = \lambda_1 \chi_1(g) \chi_1(h) + \dotsb + \lambda_n \chi_n(g) \chi_n(h)
  \intertext{and therefore}
    \label{first relation}
      0
    = \lambda_1 \chi_1(g) \chi_1 + \dotsb + \lambda_n \chi_n(g) \chi_n \,.
  \end{gather}
  It also holds that
  \begin{equation}
    \label{second relation}
      0
    = \chi_n(g) \cdot (\lambda_1 \chi_1 + \dotsb + \lambda_n \chi_n)
    = \lambda_1 \chi_n(g) \chi_1 + \dotsb + \lambda_n \chi_n(g) \chi_n \,.
  \end{equation}
  By subtracting the relation~\eqref{second relation} from the relation~\eqref{first relation} it follows that
  \[
      0
    =   \lambda_1 \left( \chi_1(g) - \chi_n(g) \right) \chi_1
      + \dotsb
      + \lambda_{n-1} (\chi_{n-1}(g) - \chi_n(g)) \chi_{n-1} \,.
  \]
  
  It follows from the induction hypothesis that for every~$i = 1, \dotsc, n-1$,~$\lambda_i = 0$ or~$\chi_i(g) = \chi_n(g)$.
  There exists for every~$i = 1, \dotsc, n$ some~$g \in G$ with~$\chi_i(g) \neq \chi_n(g)$ because~$\chi_i$ and~$\chi_n$ are distinct.
  It follows in combination that~$\lambda_1 = \dotsb = \lambda_{n-1} = 0$.
  It then also follows that~$\lambda_n = 0$ because~$\lambda_n \chi_n = 0$.
\end{proof}


\begin{definition}
  A \emph{rational character}\index{rational character}\index{character!rational} of a linear algebraic group~$G$ is a morphism of linear algebraic groups~$\chi \colon G \to \Gmult$.
\end{definition}


\begin{fluff}
  If~$G$ is a group and~$f_1, f_2 \colon G \to \Gmult$ are two group homomorphisms then their pointwise product
  \[
            f_1 f_2
    \colon  G
    \to     \Gmult \,,
    \quad   g
    \mapsto f_1(g) f_2(g)
  \]
  is again a group homomorphism because~$\Gmult$ is abelian.
  It also holds for every group homomorphism~$f \colon G \to \Gmult$ that the map
  \[
            f^{-1}
    \colon  G
    \to     \Gmult \,,
    \quad   g
    \mapsto f(g)^{-1}
  \]
  is again a group homomorphism.
  It follows that the set of group homomorphisms~$G \to \Gmult$ forms a group with respect to pointwise multiplication.
  The neutral element is given by the trivial group homomorphism and the inverse of a group homomorphism~$f \colon G \to \Gmult$ is given by~$f^{-1}$ as above.
  
  If~$G$ is a linear algebraic group and~$\chi_1, \chi_2 \colon G \to \Gmult$ are rational characters then the group homomorphism~$\chi_1 \chi_2 \colon G \to \Gmult$ is again a rational character because it is given by the composition
  \[
    \chi_1 \chi_2
    \colon
    G
    \xlongrightarrow{\Delta}
    G \times G
    \xlongrightarrow{\chi_1 \times \chi_2}
    \Gmult \times \Gmult
    \xlongrightarrow{\text{mult}}
    \Gmult \,.
  \]
  It similarly follow that for every rational character~$\chi \colon G \to \Gmult$ its inverse~$\chi^{-1} \colon g \to \Gmult$ is again a rational character becaus it is given by the composition
  \[
    \chi^{-1}
    \colon
    G
    \xlongrightarrow{\chi}
    \Gmult
    \xlongrightarrow{(-)^{-1}}
    \Gmult \,.
  \]
  It also holds that the trivial group homomorphism~$G \to \Gmult$ is a rational character.
  It follows that the set of rational characters~$G \to \Gmult$ forms a group with respect to pointwise multiplication.
\end{fluff}


\begin{definition}
  The \emph{character group}\index{character!group}\index{group!character} of a linear algebraic group~$G$ is the group of rational characters~$G \to \Gmult$ together with pointwise multiplication.
  It is denoted by~$\chargroup(G)$.
\end{definition}


\begin{remark}
  Other common notations for the character group are~$\widehat{G}$ and~$\chi(G)$.
\end{remark}


\begin{fluff}
  The character group~$\chargroup(-)$ defines a contravariant functor
  \begin{align*}
    \left\{
      \text{linear algebraic groups}
    \right\}
    &\longto
    \left\{
      \text{abelian groups}
    \right\} \,,
    \\
                  G
    &\longmapsto  \chargroup(G) \,,
    \\
                  [f \colon G \to H]
    &\longmapsto  \bigl[
                            f^*
                    \colon  \chargroup(H)
                    \to     \chargroup(G),
                            \chi
                    \mapsto \chi \circ f \,
                  \bigr] \,.
  \end{align*}
\end{fluff}


\begin{lemma}
  For every linear algebraic group~$G$ its character group~$\chargroup(G)$ is a subgroup of the unit group~$\coord(G)^\times$.
  \qed
\end{lemma}


\begin{lemma}
  \label{character group of product}
  Let~$G$ and~$H$ be linear algebraic groups.
  Then~$\chargroup(G \times H) \cong \chargroup(G) \times \chargroup(H)$.
\end{lemma}


\begin{proof}
  The canonical projections~$p \colon G \times H \to G$ and~$q \colon G \times H \to H$ are homomorphisms of linear algebraic groups and therefore induce group homomorphisms~$p^* \colon \chargroup(G) \to \chargroup(G \times H)$ and~$q^* \colon \chargroup(H) \to \chargroup(G \times H)$.
  These two group homomorphisms can be combined into a single group homomorphism
  \[
            \varphi
    \colon  \chargroup(G) \times \chargroup(H)
    \to     \chargroup(G \times H)
    \quad   (\chi_1, \chi_2)
    \mapsto p^*(\chi_1) q^*(\chi_2)
  \]
  because the group~$\chargroup(G \times H)$ is abelian.
  The rational character~$\varphi(\chi_1, \chi_2)$ is on elements of~$G \times H$ given by
  \[
      \varphi(\chi_1, \chi_2)(g,h)
    = p^*(\chi_1)(g,h) q^*(\chi_2)(g,h)
    = \chi_1(p(g,h)) \chi_2(q(g,h))
    = \chi_1(g) \chi_2(h) \,.
  \]
  
  To construct the inveres~$\psi$ of~$\varphi$ we consider the \enquote{inclusions}~$i \colon G \to G \times H$ and~$j \colon H \to G \times H$ which are homomorphisms of linear algebraic groups.
  The induced group homomorphisms~$i^* \colon \chargroup(G \times H) \to \chargroup(G)$ and~$j^* \colon \chargroup(G \times H) \to \chargroup(H)$ result in a single group homomorphism
  \[
              \psi
    \defined  (i^*, j^*)
    \colon    \chargroup(G \times H)
    \to       \chargroup(G) \times \chargroup(H) \,,
    \quad     \chi
    \mapsto   (i^*(\chi), j^*(\chi))
    =         (\chi \circ i, \chi \circ j) \,.
  \]
  
  The two group homomorphisms~$\varphi$ and~$\psi$ are mutually inverse to each other:
  It holds for every pair~$(\chi_1, \chi_2)$ of rational characters~$\chi_1 \in \chargroup(G)$,~$\chi_2 \in \chargroup(H)$ that
  \begin{align}
       &  \psi(\varphi(\chi_1, \chi_2)) \notag  \\
    ={}&  \psi( p^*(\chi_1) q^*(\chi_2) ) \notag  \\
    ={}&  ( i^*( p^*(\chi_1) q^*(\chi_2) ), j^*( p^*(\chi_1) q^*(\chi_2) ) )  \notag  \\
    ={}&  ( i^*(p^*(\chi_1)) i^*(q^*(\chi_2)) , j^*(p^*(\chi_1)) j^*(q^*(\chi_2)) ) \notag  \\
    ={}&  ( (pi)^*(\chi_1) (qi)^(\chi_2), (pj)^*(\chi_1) (qj)^*(\chi_2) ) \notag  \\
    ={}&  ( \chi_1, \chi_2) \label{last step}
  \end{align}
  where we use for the step~\eqref{last step} that~$pi = \id_G$ and~$qj = \id_H$ while the compositions~$pj$ and~$qi$ are the trivale homomorphisms.
  This shows that~$\varphi \psi = \id$.
  It also holds for every rational character~$\chi \in \chargroup(G \times H)$ and all group elements~$(g,h) \in G \times H$ that
  \begin{align*}
     {}&  \varphi(\psi(\chi))(g,h)  \\
    ={}&  \varphi(\chi \circ i, \chi \circ j)(g,h)  \\
    ={}&  (\chi \circ i)(g) (\chi \circ j)(h) \\
    ={}&  \chi((g,1)) \chi((1,h)) \\
    ={}&  \chi((g,1)(1,h))  \\
    ={}&  \chi(g,h) \,,
  \end{align*}
  which shows that~$\varphi \psi = \id$.
\end{proof}


\begin{lemma}
  \label{trivial torsion in char p}
  If~$G$ is a linear algebraic group and~$\ringchar(k) = p > 0$ then the character group~$\chargroup(G)$ has only trivial~\dash{$p$}{torsion}.
\end{lemma}


\begin{proof}
  If~$\ringchar(k) = p > 0$ then it holds for every~$x \in \Gmult$ that
  \[
      x^p - 1
    = (x - 1)^p
  \]
  and it therefore therefore holds that~$x^p = 1$ if and only if~$x = 1$.
  This shows that the multiplicative group~$\Gmult$ has only trivial~\dash{$p$}{torsion}.
  It follows that for every linear algebraic group~$G$ its character group~$\chargroup(G)$ has only trivial~\dash{$p$}{torsion}.
\end{proof}


\begin{fluff}
  We will now show that tori and diagonalizable linear algebraic groups can be characterized via they character groups.
  We start by showing that the character group of the torus~$\Diag_n(k)$ is the free abelian group~$\Integer^n$.
\end{fluff}


\begin{lemma}
  \label{units of laurant polynomials}
  Let~$R$ be an integral domain.
  Then the unit group of the ring of Laurant polynomials~$R[t,t^{-1}]$ is given by~$R[t,t^{-1}]^\times = \{ a t^n \suchthat a \in R^\times, n \in \Integer\}$.
\end{lemma}


\begin{proof}
  For every nonzero Laurant polynomial~$p = \sum_{i \in \Integer} p_i t^i \in k[t,t^{-1}]$ its \emph{lower degree} is given by
  \[
              \deg^-(p)
    \defined  \min \{ i \in \Integer \suchthat p_i \neq 0 \}
  \]
  and its \emph{upper degree} is given by
  \[
              \deg^+(p)
    \defined  \max \{ i \in \Integer \suchthat p_i \neq 0 \} \,.
  \]
  It holds that~$\deg^-(p) \leq \deg^+(p)$ with equality if and only if~$p$ is of the form~$p = a t^n$ for some nonzero~$a \in R$ and some~$n \in \Integer$.
  
  It follows from~$R$ being an integral domain that both~$\deg_-$ and~$\deg_+$ are additive in the sense that
  \[
    \deg_\pm(p_1 p_2) = \deg_\pm(p_1) + \deg_\pm(p_2)
  \]
  for any two nonzero Laurant polynomials~$p_1, p_2 \in R[t, t^{-1}]$.
  
  If~$p \in R[t, t^{-1}]$ is a unit with inverse~$q \in R[t, t^{-1}]$ then it follows that
  \[
      \deg^+(p) + \deg^+(q)
    = \deg^+(pq)
    = \deg^+(1)
    = 0
  \]
  and therefore that~$\deg^+(p) = -\deg^+(q)$.
  It follows similarly that~$\deg^-(p) = -\deg^-(q)$.
  This shows together with~$\deg^-(q) \leq \deg^+(q)$ that
  \[
          \deg^+(p)
    =     - \deg^+(q)
    \leq  - \deg^-(q)
    =     \deg^-(p) \,,
  \]
  which together with~$\deg^-(p) \leq \deg^+(p)$ shows that~$\deg^-(p) = \deg^+(p)$.
  The Lauraunt polynomial~$p$ is therefore of the form~$p = a t^n$ for some~$a \in R$,~$n \in \Integer$.
  By switching the roles of~$p$ and~$q$ it also follows that~$q$ is of the form~$q = b t^m$ for some~$b \in R$, $m \in \Integer$.
  
  It follows from~$p$ and~$q$ being inverse to each other that~$m = -n$ and that~$a, b \in R^\times$ with~$b = a^{-1}$.
\end{proof}


\begin{lemma}
  \label{structure theory for Dn}
  It holds for the group of diagonal matrices~$\Diag_n(k)$ that
  \begin{gather*}
      \coord(\Diag_n(k))
    = k[T_1, T_1^{-1}, \dotsc, T_n, T_n^{-1}]
  \shortintertext{where}
      T_i
      \left(
        \begin{pmatrix}
          d_1 &         &     \\
              & \ddots  &     \\
              &         & d_n
        \end{pmatrix}
      \right)
    = d_i \,,
  \end{gather*}
  for every~$i$, and it holds that
  \begin{align*}
          \coord\bigl( \Diag_n(k) \bigr)^\times
    &=    \bigcup_{\mindex{a} \in \Integer^n} k^\times T_1^{a_1} \dotsm T_n^{a_n}
    \cong k^\times \times \Integer^n \,,
  \shortintertext{and}
          \chargroup\bigl( \Diag_n(k) \bigr)
    &=    \bigcup_{\mindex{a} \in \Integer^n} T_1^{a_1} \dotsm T_n^{a_n}
    \cong \Integer^n \,.
  \end{align*}
  It holds in particular that~$\chargroup( \Diag_n(k) )$ is a~\dash{$k$}{basis} of~$\coord( \Diag_n(k) )$.
\end{lemma}


\begin{proof}
  It holds that
  \begin{align*}
           &  \coord(\Diag_n(k)) \\
        ={}&  \coord(\Gmult \times \dotsb \times \Gmult)  \\
    \cong{}&  \coord(\Gmult) \tensor \dotsb \tensor \coord(\Gmult) \\
        ={}&  k\left[ T_1, T_1^{-1} \right]
              \tensor \dotsb \tensor
              k\left[ T_n, T_n^{-1} \right]  \\
    \cong{}&  k\left[ T_1, T_1^{-1}, \dotsc, T_n T_n^{-1} \right]
  \end{align*}
  as claimed.
  This equality~$\coord(\Diag_n(k))^\times = \bigcup_{\mindex{a} \in \Integer^n} k^\times T_1^{a_1} \dotsm T_n^{a_n}$ follows from \cref{units of laurant polynomials} by induction over~$n$ on
  \[
      k\left[ T_1, T_1^{-1}, T_2, T_2^{-1}, \dotsc, T_n, T_n^{-1} \right]
    = k\left[ T_1, T_1^{-1} \right]
       \left[ T_2, T_2^{-1} \right]
       \dotsm
       \left[ T_n, T_n^{-1} \right] \,.
  \]
  
  The inclusion~$\bigcup_{\mindex{a} \in \Integer^n} T_1^{a_1} \dotsm T_n^{a_n} \subseteq \chargroup(\Diag_n(k))$ holds because every~$T_i \colon \Diag_n(k) \to \Gmult$ is a homomorphism of linear algebraic groups (namely the projection of~$\Diag_n(k) = \Gmult^n$ onto the~\dash{$i$}{th} factor).
  To convince ourselves of the inclusion~$\chargroup(\Diag_n(k)) \subseteq \bigcup_{\mindex{a} \in \Integer^n} T_1^{a_1} \dotsm T_n^{a_n}$ we note that every~$\chi \in \chargroup(\Diag_n(k))$ is a unit in~$\coord(\Diag_n(k))^\times$ and thus of the form~$\chi = \lambda T_1^{a_1} \dotsm T_n^{a_m}$ for some scalar~$\lambda \in k^\times$ and some multiindex~$\mindex{a} \in \Integer^n$.
  It holds that
  \[
    1 = \chi(1) = \lambda
  \]
  and therefore that~$\chi = T_1^{a_1} \dotsm T_n^{a_n}$.
\end{proof}


\begin{theorem}[Structure of diagonalizable groups]
  \label{diag groups via char group}
  For a linear algebraic group~$G$ the following conditions are equivalent:
  \begin{enumerate}
    \item
      \label{G is diagonalizable}
      The group $G$ is diagonalizable.
    \item
      \label{X(G) is nice}
      The character group~$\chargroup(G)$ is finitely generated and a \dash{$k$}{basis} of~$\coord(G)$.
    \item
      \label{G is product of things}
      It holds that~$G \cong \mu_{d_1} \times \dotsb \times \mu_{d_n} \times \Diag_r(k)$ where~$\mu_d \defined \{g \in \Gmult \suchthat g^d = 1\}$ is the group of the~\dash{$d$}{th} roots of unity.
  \end{enumerate}
\end{theorem}


\begin{proof}
  The proof proceeds in six steps.
  \begin{enumerate}[label = Step~\arabic*]
    \item
      Let~$G$ be a diagonalizable linear algebraic group and let~$i \colon G \inclusion \Diag_n(k)$ be a closed embedding for suitable~$n$.
      The induced algebra homomorphism~$i^* \colon \coord(\Diag_n(k)) \to \coord(G)$ is then a surjection (for example by \cref{induced is isomorphism or surjective}) and we have the following commutative diagram:
      \[
        \begin{tikzcd}
            \coord(\Diag_n(k))
            \arrow{r}[above]{i^*}
          & \coord(G)
          \\
            \chargroup(\Diag_n(k))
            \arrow{r}[above]{i^*}
            \arrow[hook]{u}
          & \chargroup(G)
            \arrow[hook]{u}
        \end{tikzcd}
      \]
      It follows from \cref{structure theory for Dn} that~$\chargroup(\Diag_n(k))$ is a~\dash{$k$}{basis} of~$\coord(\Diag_n(k))$, from which it then follows with the surjectivity of~$i^* \colon \coord(\Diag_n(k)) \to \coord(G)$ and the commutativity of the diagram that~$\coord(G)$ is generated by~$\chargroup(G)$ as a~\dash{$k$}{vector space}.
      It follows that the set~$\chargroup(G)$ is a~\dash{$k$}{basis} for~$\coord(G)$ because it is linearly independent by the \hyperref[dedekind artin lemma]{Dedekind\nobreakdash--Artin lemma}.
      
      This shows the implication \ref*{G is diagonalizable}~$\implies$~\ref*{X(G) is nice}.
      
    \item
      \label{every character a restriction}
      Note that in the above situation the image~$i^*(\chargroup(\Diag_n(k)))$ is is a~\dash{$k$}{generating} set of~$\coord(G)$ (because~$\chargroup(\Diag_n(k))$ is a~\dash{$k$}{basis} for~$\coord(\Diag_n(k))$ and the algebra homomorphism~$i^* \colon \coord(\Diag_n(k)) \to \coord(G)$ is surjective).
      This image is contained in the~\dash{$k$}{basis}~$\chargroup(G)$ of~$\coord(G)$, from which it follows that~$i^*(\chargroup(\Diag_n(k))) = \chargroup(G)$, i.e.\ that~$i^*$ is surjective.
      This shows that every character of~$G$ is the restriction of a character of~$\Diag_n(k)$.
    
    \item
      \label{fully faithful}
      Let~$G$ be any linear algebraic groups and~$H$ is a diagonalizable linear algebraic group then the map
      \[
                  \varepsilon
        \defined  (-)^*
        \colon    \Hom_{\,\text{lin.\ alg.\ groups}\,}(G, H)
        \to       \Hom_{\Ab}( \chargroup(H), \chargroup(G) )
      \]
      is bijective.
      
      To see that the map~$\varepsilon$ is injective we note that we have for every homomorphism of linear algebraic groups~$f \colon G \to H$ the following commutative diagram:
      \[
        \begin{tikzcd}
            \coord(H)
            \arrow{r}[above]{f^*}
          & \coord(G)
          \\
            \chargroup(H)
            \arrow{r}[above]{f^*}
            \arrow[hook]{u}
          & \chargroup(G)
            \arrow[hook]{u}
        \end{tikzcd}
      \]
      The homomorphism~$f$ is uniquely determined by the induced algebra homomorphism~$f^* \colon \coord(H) \to \coord(G)$ (because~$\coord(-)$ is faithful), which in turn is uniquely determined by its action on the~\dash{$k$}{basis}~$\chargroup(H)$ of~$\coord(H)$.
      This action is given by th induced group homomorphism~$\varepsilon(f) = f^* \colon \chargroup(H) \to \chargroup(G)$, which shows that~$\varepsilon$ is injective.
      
      To show that~$\varepsilon$ is surjective let~$f \colon \chargroup(H) \to \chargroup(G)$ be a group homomorphism.
      It follows from~$\chargroup(H)$ being a basis of~$\coord(H)$ that~$\chi$ extends uniquely to a~\dash{$k$}{linear} map~$F \colon \coord(H) \to \coord(G)$.
      The map~$F$ is multiplicative on the basis~$\chargroup(H)$ of~$\coord(H)$ and satisfies the equality~$F(1) = \chi(1) = 1$;
      this shows that~$F$ is an algebra homomorphism.
      It follows that there exists a unique morphism of affine varieties~$h \colon G \to H$ with~$F = f^*$.
      This morphim~$h$ is a group homomorphism:
      It holds that
      \begin{align}
            & \text{$h(g_1 g_2) = h(g_1) h(g_2)$ for all~$g_1, g_2 \in G$}  \notag  \\
      \iff{}& \text{$\varphi(h(g_1 g_2)) = \varphi( h(g_1) h(g_2) )$ for all~$g_1, g_2 \in G$,~$\varphi \in \coord(H)$} \notag  \\
      \iff{}& \text{$\chi'(h(g_1 g_2)) = \chi'(h(g_1) h(g_2))$ for all~$g_1, g_2 \in G$,~$\chi' \in \chargroup(H)$}
      \label{replace by basis}
      \,,
      \end{align}
      where the equivalence~\eqref{replace by basis} holds because~$\chargroup(H)$ is a~\dash{$k$}{basis} for~$\coord(H)$.
      The needed equality~$\chi'(f(g_1 g_2)) = \chi'(f(g_1) f(g_2))$ holds true because
      \begin{align}
            \chi'(h(g_1 g_2)) \notag
        &=  h^*\left( \chi' \right)(g_1 g_2) \notag \\
        &=  F(\chi')(g_1 g_2) \notag  \\
        &=  f(\chi')(g_1 g_2) \notag  \\
        &=  f(\chi')(g_1) f(\chi')(g_2) \label{is a group homo}  \\
        &=  F(\chi')(g_1) F(\chi')(g_2) \notag  \\
        &=  h^*(\chi')(g_1) h^*(\chi')(g_2) \notag  \\
        &=  \chi'(h(g_1)) \chi'(h(g_2)) \notag  \\
        &=  \chi'(h(g_1) h(g_2)) \notag \,,
      \end{align}
      where we use for the equality~\eqref{is a group homo} that~$f(\chi') \in \chargroup(G)$ is again a character, and thus a group homomorphism.
      With this we have shown that~$f$ is a group homomorphism, and therefore already a homomorphism of linear algebraic groups.
      It follows from the commutativity of the diagram
      \[
        \begin{tikzcd}
            \coord(H)
            \arrow{r}[above]{F = h^*}
          & \coord(G)
          \\
            \chargroup(H)
            \arrow{r}[above]{f}
            \arrow[hook]{u}
          & \chargroup(G)
            \arrow[hook]{u}
        \end{tikzcd}
      \]
      that~$f = h^* = \varepsilon(h)$, which shows that~$\varepsilon$ is surjective.
      
    \item
      Suppose that~$G$ is a linear algebraic group for which its character group~$\chargroup(G)$ is finitely generated, and a basis of~$\coord(G)$.
      It follows from~$\chargroup(G)$ being finitely generated that there exists for some~$n$ a surjective grouphomorphism~$\Integer^n \to \chargroup(G)$ and thus a surjective grouphomomorphism~$h \colon \chargroup(\Diag_n(k)) \to \chargroup(G)$.
      It then follows from \ref{fully faithful} that there exists a homomorphism of linear algebraic groups~$f \colon G \to \Diag_n(k)$ with~$h = f^*$.
      
      The homomorphism~$f$ is a closed embedding:
      We consider the following commutative diagram:
      \[
        \begin{tikzcd}
            \coord(\Diag_n(k))
            \arrow{r}[above]{f^*}
          & \coord(G)
          \\
            \chargroup(\Diag_n(k))
            \arrow{r}[above]{h = f^*}
            \arrow[hook]{u}
          & \chargroup(G)
            \arrow[hook]{u}
        \end{tikzcd}
      \]
      The lower horizontral arrow~$f^* \colon \chargroup(\Diag_n(k)) \to \chargroup(G)$ is surjective and~$\chargroup(G)$ is a~\dash{$k$}{basis} of~$\coord(G)$, so it follows that the image of the composition~$\chargroup(\Diag_n(k)) \to \coord(G)$ is a~\dash{$k$}{generating} set of~$\coord(G)$.
      It follows that the induction algebra homomorphism~$f^* \colon \coord(\Diag_n(k)) \to \coord(G)$ is surjective.
      This shows that~$f$ is closed embedding by \cref{induced is isomorphism or surjective}.
      
      This shows the implication \ref*{X(G) is nice}~$\implies$~\ref*{G is diagonalizable}.
      
    \item
      To show the implication \ref*{G is diagonalizable}~$\implies$~\ref*{G is product of things} let~$G$ be a diagonalizable linear algebraic group, which we may assume to be a closed subgroup of some~$\Diag_n(k) \defines T$.
      
      We have seen in \ref{every character a restriction} that every character of~$G$ is the restriction of a character of~$T$, i.e.\ that the group homomorphism~$i^* \colon \chargroup(T) \to \chargroup(G)$ induced by the inclusion~$i \colon G \inclusion T$ is surjective.
      Let~$\Gamma \defined \ker(i^*) \groupleq \chargroup(T)$ be its kernel.
      It then holds that~$G = \bigcap_{\chi \in \Gamma} \ker(\chi)$:
      
      The inclusion~$G \subseteq  \bigcap_{\chi \in \Gamma} \ker(\chi)$ holds by definition of~$\Gamma$.
      To convince ourselves of the other inclusion we note that~$G' \defined \bigcap_{\chi \in \Gamma} \ker(\chi)$ is a closed subgroup of~$T$ for which a character~$\chi \in \chargroup(T)$ is trivial on~$G$ (i.e.\ contained in~$\Gamma$) if and only if it is trivial on~$G'$.
      It follows that both~$G$ and~$G'$ have the same characters, i.e.\ that the inclusion~$j \colon G \inclusion G'$ induces an isomorphism of group~$j^* \colon \chargroup(G') \to \chargroup(G)$, since the characters on~$G$ and~$G'$ are the restrictions of the characters of~$T$ by \ref{every character a restriction}.
      This shows that the induced algebra homorphism~$j \colon \coord(G') \to \coord(G)$ maps the~\dash{$k$}{basis}~$\chargroup(G')$ of~$\coord(G')$ bijectively onto the~\dash{$k$}{basis}~$\chargroup(G)$ of~$\coord(G)$, and is therefore an algebra isomorphism.
      It follows that~$j$ is an isomorphism and therefore that~$G = G'$, as claimed.
      
      The group
      \[
                \chargroup(T)
        =       \left\{
                  T_1^{a_1} \dotsm T_n^{a_n}
                \suchthat
                  \mindex{a} \in \Integer^n
                \right\}
         \cong  \Integer^n
      \]
      is free abelian of finite rank so we may apply the theory of elementary divisors.
      With find that there exists a~\dash{$\Integer$}{basis}~$\tilde{T}_1, \dotsc, \tilde{T}_n$ of~$\chargroup(T)$ with respect to which the subgroup~$\Gamma$ of~$\chargroup(T)$ is generated by~$\tilde{T}_1^{d_1}, \dotsc, \tilde{T}_n^{d_n}$ for suitable~$d_1, \dotsc, d_n \in \Integer$.
      
      Let~$f \colon \chargroup(T) \to \chargroup(T)$ be the unique group automorphism which maps~$T_i$ to~$\tilde{T}_i$ for every~$i$.
      It follows from \ref{fully faithful} that there exists a (unique) homomorphism of linear algebraic groups~$h \colon T \to T$ with~$h^* = f$.
      The homomorphism~$h$ is again an automorphism:
      The inverse~$f^{-1} \colon \chargroup(T) \to \chargroup(T)$ is again a group homomorphism, and thus there exists a (unique) homomorphism of linear algebraic groups~$h' \colon T \to T$ with~$(h')^* = f^{-1}$.
      It follows from
      \[
          (h h')^*
        = (h')^* h^*
        = f^{-1} f
        = \id_{\chargroup(T)}
        = \id_T^*
      \]
      that~$h h' = \id_T$.
      We find similarly that~$h' h = \id_T$.
      
      The image~$G' \defined h(G)$ is again a closed subgroup of~$T$, and the linear algebraic groups~$G$ and~$G'$ are isomorphic via (the restriction of)~$h$.
      Let~$\Gamma'$ be the kernel of the restriction homomorphism~$\chargroup(T) \to \chargroup(G')$;
      it holds that
      \[
          G'
        = \bigcap_{\chi \in \Gamma'} \ker(\chi).
      \]
      It holds for every~$\chi \in \chargroup(G)$ that
      \begin{align*}
                \chi \in \Gamma'
        \iff{}& \restrict{\chi}{G'} = 1 \\
        \iff{}& \restrict{\chi}{h(G)} = 1 \\
        \iff{}& \restrict{(\chi \circ h)}{G} = 1  \\
        \iff{}& \chi \circ h \in \Gamma \\
        \iff{}& h^*(\chi) \in \Gamma  \\
        \iff{}& \chi \in (h^*)^{-1}(\Gamma) \,,
      \end{align*}
      which shows that
      \[
          \Gamma'
        = (h^*)^{-1}(\Gamma)
        = f^{-1}(\Gamma) \,.
      \]
      It holds that~$f^{-1}( \tilde{T}_i ) = T_i$ for every~$i$, and the group~$\Gamma$ is generated by~$\tilde{T}_1^{d_1}, \dotsc, \tilde{T}_n^{d_n}$.
      It follows that the group~$\Gamma'$ is generated by~$T_1^{d_1}, \dotsc, T_n^{d_n}$.
      
      Hence we may replace~$G$ by~$G'$, and~$\Gamma$ by~$\Gamma'$, and assume that
      \[
          \Gamma
        = \gen*{ T_1^{d_1}, \dotsc, T_n^{d_n} }
        = \left\{
            T_1^{a_1} \dotsm T_n^{a_n}
          \suchthat*
            \mindex{a} \in d_1 \Integer \times \dotsb \times d_n \Integer
          \right\} \,.
      \]
      It then follows that
      \begin{align*}
            G 
        &=  \bigcap_{\chi \in \Gamma} \ker(\Gamma)
         =  \ker\left( T_1^{d_1} \right)
            \cap \dotsb \cap
            \ker\left( T_n^{d_n} \right)  \\
        &=  ( \mu_{d_1} \times \Gmult \times \dotsb \times \Gmult )
            \cap \dotsb \cap
            ( \Gmult \times \dotsb \times \Gmult \times \mu_{d_n} ) \\
        &=  \mu_{d_1} \times \dotsb \times \mu_{d_n}
      \end{align*}
      with~$\mu_d = \Gmult$ for~$d = 0$.
      
      This shows the implication \ref*{G is diagonalizable}~$\implies$~\ref*{G is product of things}.
      
  \item
      To show the implication \ref*{G is product of things}~$\implies$~\ref*{G is diagonalizable} we may assume that~$g = \mu_{d_1} \times \dotsb \times \mu_{d_n} \times \Diag_r(k)$.
      It then follows from~$\mu_{d_i}$ being a closed subgroup of~$\Gmult$ for every~$i$ that~$G$ is a closed subgroup of
      \[
          \underbrace{\Gmult \times \dotsb \times \Gmult}_{n} \times \Diag_r(k)
        = \Diag_{n+r}(k) \,.
      \]
      This shows that~$G$ is diagonalizable.
     \qedhere
  \end{enumerate}
\end{proof}


\begin{corollary}
  The character group~$\chargroup(-)$ defines a contravariant equivalence of categories
  \begin{align*}
                  \left\{
                    \begin{tabular}{@{}c@{}}
                      diagonalizable  \\
                      linear algebraic groups
                    \end{tabular}
                  \right\}
    &\longto      \left\{
                    \begin{tabular}{@{}c@{}}
                      finitely generated abelian groups \\
                      (with trivial \dash{$p$}{torsion} if~$\ringchar(k) = p > 0$)
                    \end{tabular}
                  \right\} \,,  \\
                  G
    &\longmapsto  \chargroup(G) \,, \\
                  f
    &\longmapsto  f^* \,,
  \intertext{which restrict to a contravariant equivalence}
              \left\{
                \text{tori}
              \right\}
    &\longto  \left\{
                \text{free abelian groups of finite rank}
              \right\} \,.
  \end{align*}
\end{corollary}


\begin{proof}
  We denote the first of the proposed equivalences by~$F$.
  That~$F$ is \dash{well}{defined} follows from \cref{diag groups via char group} and \cref{trivial torsion in char p}.
  We have seen in \ref{fully faithful} of the above proof that~$F$ is fully faithful.
  
  To see that~$F$ is essentially surjective let~$A$ be a finitely generated abelian group, with trivial~\dash{$p$}{torsion} if~$\ringchar(k) = p > 0$.
  It follows from the classification of finitely generated abelian groups that
  \[
          A
    \cong \Integer/d_1 \times \dotsb \times \Integer/d_n \oplus \Integer^r
  \]
  for suitable integers~$d_1, \dotsc, d_n \geq 2$ and some~$r \geq 0$;
  if~$\ringchar(k) = p > 0$ then it follows from~$A$ having trivial~\dash{$p$}{torsion} that none of the integers~$d_1, \dotsc, d_n$ is a multiple of~$p$.
  The linear algebraic group
  \[
              G
    \defined  \mu_{d_1} \times \dotsb \times \mu_{d_n} \times \Diag_r(k)
  \]
  is a closed subgroup of~$\Diag_{n+r}(k)$ and it follows from \cref{character group of product} that
  \[
          \chargroup(G)
    \cong \chargroup(\mu_{d_1})
          \times \dotsb \times
          \chargroup(\mu_{d_n})
          \times
          \chargroup(\Diag_r(k)) \,.
  \]
  The factor~$\chargroup(\Diag_r(k))$ is given by~$\Integer^r$.
  It follows from~$p$ not dividing the integer~$d_i$ that~$\mu_{d_i} \cong \Integer/d_i$:
  The group~$\mu_{d_i}$ is cyclic because it is a finite subgroup of the multiplicative group~$k^\times$, and it has order~$d_i$ because the polynomial~$x^{d_i} - 1 \in k[x]$ has no multiple roots since it is separable.
  It follows that
  \[
          \chargroup(\mu_{d_i})
    =     \chargroup(\Integer/d_i)
    \cong \Integer/d_i
  \]
  because every group homomorphism~$\Integer/d_i \to \Gmult$ is already a homomorphism of linear algebraic groups.
  It follows altogether that
  \[
          \chargroup(G)
    \cong \Integer/d_1 \times \dotsb \times \Integer/d_n \oplus \Integer^r
    \cong A \,,
  \]
  which shows that~$F$ is essentially surjective.
  
  It follows from \cref{structure theory for Dn} that~$F$ restrict so the second equivalence of categories.
\end{proof}




