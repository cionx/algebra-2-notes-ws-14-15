\section{Definition and First Examples}


\begin{definition}
  An \emph{affine algebraic group} is an affine variety~$G$ together with a group structure such that both the multiplication map~$G \times G \to G$,~$(g_1, g_2) \mapsto g_1 g_2$ and inversion map~$G \to G$,~$g \mapsto g^{-1}$ are morphisms of affine varieties.
  
  If~$G, H$ are affine algebraic groups then a map~$f \colon G \to H$ is a \emph{homomorphism of affine algebraic groups} if it is both a morphism of affine varieties and a group homomorphism.
\end{definition}


\begin{example}
  The \emph{additive groups}~$\Gadd \defined (\Aff^1, +)$ is an affine algebraic group since the addition
  \[
            {+}
    \colon  \Aff^1 \times \Aff^1
    =       \Aff^2
    \to     \Aff^1 \,,
    \quad   (x,y)
    \mapsto x+y
  \]
  and inversion
  \[
            {-}
    \colon  \Aff^1
    \to     \Aff^1 \,,
    \quad   x
    \mapsto -x
  \]
  are regular.
\end{example}


\begin{example}
  The \emph{general linear group}~$\GL_n(k)$ together with the usual matrix multiplication is an affine algebraic group.
  We have seen in \cref{principal open sets of affine are again affine} that~$\GL_n(k)$ is an affine variety, the multiplication map~$\GL_n(k) \times \GL_n(k) \to \GL_n(k)$ is polynomial and therefore regular, and the inversion map~$\GL_n(k) \to \GL_n(k)$ is a rational function (because the entries of~$A^{-1}$ are rational functions in the entries of~$A$ by Cramer’s rule) and therefore also regular.
  
  It follows for~$n = 1$ that the \emph{multiplicative group}~$\Gmult \defined \GL_1(k) = k^\times$ is an affine algebraic group.
\end{example}


\begin{lemma}
  Every Zariski closed subgroup of an affin algebraic group is again an affine algebraic group.
\end{lemma}


\begin{proof}
  Let~$G$ be an affine algebraic group and let~$H \groupleq G$ be a Zariski closed subgroup.
  It follows from \cref{closed of affine is again affine} that~$H$ is again an affine variety, and the multiplication~$H \times H \to H$ and inversion~$H \to H$ are morphisms because they are restrictions of the multiplication~$G \times G \to G$ and inversion~$G \to G$.
\end{proof}


\begin{corollary}
  \label{closed subgroups of affine algebraic group}
  Every Zariski closed subgroup of~$\GL_n(k)$ is an affine algebraic group.
  \qed
\end{corollary}


\begin{example}
  It follows from \cref{closed subgroups of affine algebraic group} that the following are affine algebraic groups:
  \begin{enumerate}
    \item
      The \emph{special linear group}~$\SL_n(k) = \{ A \in \GL_n(k) \suchthat \det A = 1 \}$.
    \item
      The \emph{orthogonal groups}~$\Orth_n(k) = \{ A \in \GL_n(k) \suchthat A^T A = I \}$.
    \item
      The \emph{special orthogonal group}~$\SOrth_n(k) = \SL_n(k) \cap \Orth_n(k)$.
    \item
      The \emph{symplectic group}~$\Symp_{2n}(k) = \{ A \in \GL_n(k) \suchthat A^T J A = J \}$ where~$
          J
        = \begin{psmallmatrix}
             0    & I_n \\
            -I_n  & 0
          \end{psmallmatrix}
      $.
    \item
      Every finite group, when regarded as an affine set and thus an affine variety, is an affine algebraic group.
      The multiplication map~$G \times G \to G$ and inversion map~$G \to G$ are morphisms of afffine varieties because both~$G$ and~$G \times G$ are finite, and therefore all maps~$G \times G \to G$ and~$G \to G$ are morphisms.
    \item
      The \emph{group of diagonal matrices}~$
          \Diag_n(k)
        = \left\{
            \begin{psmallmatrix}
              * &        &    \\
                & \ddots &    \\
                &        & *
            \end{psmallmatrix}
            \in \GL_n(k)
          \right\}
      $.
    \item
      The \emph{group of upper triangular matrices}~$
          \Triag_n(k)
        = \left\{
            \begin{psmallmatrix}
              * & \cdots & *      \\
                & \ddots & \vdots \\
                &        & *
            \end{psmallmatrix}
            \in \GL_n(k)
          \right\} \,.
      $
    \item
      The \emph{group of unipotent matrices}~$
          \Uni_n(k)
        = \left\{
            \begin{psmallmatrix}
              1 & \cdots & *      \\
                & \ddots & \vdots \\
                &        & 1
            \end{psmallmatrix}
            \in \GL_n(k)
          \right\} \,.
      $
  \end{enumerate}
\end{example}


% For embeddings: Motivate by G_a.


\begin{remark}
  One may rephrase the definition of an affine algebraic groups as saying that~$G$ is an affine variety together with morphisms
  \[
            m
    \colon  G \times G
    \to     G
    \quad\text{and}\quad
            i
    \colon  G
    \to     G
  \]
  and an element~$e \in G$ such that the following diagrams commute:
  \[
    \renewcommand{\arraystretch}{1.5}
    \renewcommand{\arraycolsep}{3.3pt}
    \begin{array}{ccc}
        \begin{tikzcd}[column sep = 30pt, row sep = 35pt]
            G \times G \times G
            \arrow{r}[above]{m \times \id}
            \arrow{d}[left]{\id \times m}
          & G \times G
            \arrow{d}[right]{m}
          \\
            G \times G
            \arrow{r}[below]{m}
          & G
        \end{tikzcd}
      &
        \begin{tikzcd}[column sep = 10pt, row sep = 9pt]
            {}
          & G \times G
            \arrow{dd}[right]{m}
          & {}
          \\
            G \times \{e\}
            \arrow[hook]{ur}
            \arrow[two heads]{dr}[below left]{\pr_1}
          & {}
          & \{e\} \times G
            \arrow[hook']{ul}
            \arrow[two heads]{dl}[below right]{\pr_2}
          \\
            {}
          & G
          & {}
        \end{tikzcd}
      &
        \begin{tikzcd}[column sep = 3pt, row sep = 9pt]
            G
            \arrow{rr}[above]{(i,\id)}
            \arrow{dd}[left]{(\id,i)}
            \arrow{dr}
          & {}
          & G \times G
            \arrow{dd}[right]{m}
          \\
            {}
          & \{e\}
            \arrow[hook]{dr}
          & {}
          \\
            G \times G
            \arrow{rr}[below]{m}
          & {}
          & G
        \end{tikzcd}
    \\
        \text{associativity}
      & \text{neutral element}
      & \text{inverse}
    \end{array}
  \]

  If more generally~$\mc{C}$ is any category with finite products, including a terminal object~$\ast$, then a \emph{group object} in~$\mc{C}$ is an object~$G \in \mc{C}$ together with morphisms
  \[
    m \colon G \times G \to G \,,
    \qquad
    i \colon G \to G \,,
    \qquad
    e \colon \ast \to G
  \]
  such that the following diagrams commute:
  \[
    \renewcommand{\arraystretch}{1.5}
    \renewcommand{\arraycolsep}{4pt}
    \begin{array}{ccc}
        \begin{tikzcd}[row sep = 36pt, column sep = 30pt]
            G \times G \times G
            \arrow{r}[above]{m \times \id}
            \arrow{d}[left]{\id \times m}
          & G \times G
            \arrow{d}[right]{m}
          \\
            G \times G
            \arrow{r}[below]{m}
          & G
        \end{tikzcd}
      &
        \begin{tikzcd}[row sep = 10pt, column sep = 15pt]
            {}
          & G \times G
            \arrow{dd}[right]{m}
          & {}
          \\
            G \times *
            \arrow{ur}[above left]{\id \times e}
            \arrow{dr}[below left]{\pr_1}
          & {}
          & * \times G
            \arrow{ul}[above right]{e \times \id}
            \arrow{dl}[below right]{\pr_2}
          \\
            {}
          & G
          & {}
        \end{tikzcd}
      &
        \begin{tikzcd}[row sep = 12pt, column sep = 10pt]
            G
            \arrow{rr}[above]{(i,\id)}
            \arrow{dd}[left]{(\id,i)}
            \arrow{dr}
          & {}
          & G \times G
            \arrow{dd}[right]{m}
          \\
            {}
          & *
            \arrow{dr}{e}
          & {}
          \\
            G \times G
            \arrow{rr}[below]{m}
          & {}
          & G
        \end{tikzcd}
    \\
        \text{associativity}
      & \text{neutral \enquote{element}}
      & \text{inverse}
    \end{array}
  \]
\end{remark}


\begin{example}
  \leavevmode
  \begin{enumerate}
    \item
      The group objects in the category of sets are just groups.
    \item
      The group objects in the category of topological spaces are topological groups.
    \item
      The group objects in the category of smooth real manifolds are real Lie groups.
    \item
      The group objects in the category of affine varieties are affine algebraic groups.
  \end{enumerate}
\end{example}


\begin{warning}
  An affine algebraic group~$G$ is in general not a topological group.
  While the multiplication~$m \colon G \times G \to G$ is a morphism of affine varieties and therefore continuous, it is so with respect to the Zariski topology on~$G \times G$.
  For~$G$ to be a topological group we would need~$m$ to be continuous with respect to the product topololgy of~$G \times G$, which is in general coarser than the Zariski topology (see \cref{zariski finer than product topology}).
\end{warning}


\begin{definition}
  For an affine algebraic group~$G$ the connected component of the identity~$1 \in G$ is denoted by~$G^0$.
\end{definition}


\begin{proposition}
  Let~$G$ be a linear algebraic group.
  \begin{enumerate}
    \item
      The connected components of~$G$ coincide with its irreducible components.
    \item
      The connected/irreducible component~$G^0$ is a normal subgroup of~$G$.
    \item
      The connected/irreducible components of~$G$ are the cosets of~$G^0$.
    \item
      The group~$G^0$ has finite index in~$G$.
  \end{enumerate}
\end{proposition}


\begin{proof}
  \leavevmode
  \begin{enumerate}
    \item
      It follows from \cref{quasi-affine have only finitely many irreducible} that~$G$ has only finitely many irreducible components~$C_1, \dotsc, C_n$.
      It holds that~$C_1 \nsubseteq C_2 \cup \dotsb \cup C_n$ because it would otherwise follows from the irreducibility of~$C_1$ that~$C_1 \subseteq C_i$ for some~$i \geq 2$, which would contradict~$C_1, \dotsc, C_n$ being the irreducible components of~$G$.
      Let~$g \in C_1$ with~$x \notin C_2 \cup \dotsc \cup C_n$.
      
      The element~$x$ is contained in precisely one irreducible component.
      For every~$g \in G$ the map
      \[
                G
        \to     G \,,
        \quad   h
        \mapsto g h x^{-1}
      \]
      is regular and therefore a homeomorphism, and maps~$x$ to~$g$.
      It follows that every~$g \in G$ is contained in precisely one irreducible component.
      The irreducible components of~$G$ are therefore disjoint.
      
      Each irreducible component~$C$ is closed.
      The complement of~$C$ is also closed because it is the union of all other irreducible components, and therefore a finite union of closed sets.
      This shows that each irreducible component~$C$ of~$G$ is actually clopen.
      
      It follows that each connected component of~$G$ is contained in an irreducible component.
      It also holds that every irreducible component is contained in a connected component because irreducible topological spaces are connected.
      It follows that the connected and irreducible components coincide.
    \item
      It holds that~$1 \in G^0$ by the definition of~$G^0$.
      For every~$g \in G^0$ the left multiplication
      \[
                \lambda_g
        \colon  G
        \to     G \,,
        \quad   h
        \mapsto gh
      \]
      is an isomorphism of varieties and thus a homeomorphism.
      It therefore maps the component~$G^0$ of the identity~$1$ onto the component of~$\lambda_g(1) = g$.
      It follows from~$g$ being contained in~$G^0$ that this component is~$G^0$ and therefore that~$g G^0 = \lambda_g( G^0 ) = G^0$.
      
      It follows similarly that~$(G^0)^{-1}$ is the component of~$G$ which contains~$1^{-1} = 1$ and therefore that~$(G^0)^{-1} = G^0$.
      
      Together this shows that~$G^0$ is a subgroups of~$G$.
      To show that~$G^0$ is normal in~$G$ let~$g \in G$.
      The conjugation map
      \[
                c_g
        \colon  G
        \to     G \,,
        \quad   h
        \mapsto g h g^{-1}
      \]
      is regular and therefore a homeomorphism.
      It follows that~$g G^0 g^{-1} = c_g(G^0)$ is the component containing~$c_g(1) = 1$ and therefore that~$g G^0 g^{-1} = G^0$.
    \item
      We find in the above notation that the component of~$g \in G$ is given by~$\lambda_g( G^0 ) = g G^0$.
    \item
      The index~$[G : G^0]$ is the number of components of~$G$.
    \qedhere
  \end{enumerate}
\end{proof}


\begin{corollary}
  An affine algebraic group is connected if and only if it is irreducible.
  \qed
\end{corollary}





