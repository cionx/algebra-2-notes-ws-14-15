\section{Embedding Theorems}
\label{embedding theorems}


\begin{fluff}
  We have seen in \cref{closed subgroups of affine algebraic group} that every closed subgroup of~$\GL_n(k)$ is an affine algebraic group.
  In this \nameCref{embedding theorems} we show the converse:
  Every affine algebraic group is isomorphic to a closed subgroup of some~$\GL_n(k)$.
\end{fluff}


\begin{fluff}
  We motivate the next few statements by first considering a finite group~$G$.
  Then~$G$ acts on the group algebra~$k[G]$ via left multplication.
  This makes~$k[G]$ into a faithful representation of~$G$, resulting in an injective group homorophism~$G \inclusion \GL(k[G])$.
  The image of this inclusion is a finite, and thus closed subgroups of~$\GL_n(k)$ which is isomorphic to~$G$.
  
  For an affine algebraic group~$G$ modify this approach by replacing the group algebra~$k[G]$ with the coordinate ring~$\coord(G)$:
  For every~$g \in G$ the right multiplication
  \[
            r_g
    \colon  G
    \to     G \,,
    \quad   h
    \mapsto hg
  \]
  is an isomorphism of affine varieties, which induces an isomorphism of~\dash{$k$}{algebras}
  \[
            \rho_g
    \colon  \coord(G)
    \to     \coord(G) \,,
    \quad   f
    \mapsto f \circ r_g
    =       f((-) g) \,.
  \]
  It holds for all~$g_1, g_2 \in G$ that
  \[
      \rho_{g_1} \rho_{g_2}
    = r_{g_1}^* r_{g_2}^*
    = ( r_{g_2} r_{g_1} )^*
    = r_{g_1 g_2}^*
    = \rho_{g_1 g_2}
  \]
  so~$\rho \colon G \to \GL(\coord(G))$ is a group homomorphism.
  
  Since the coordinate ring~$\coord(G)$ is in general infinite-dimensional (unless~$G$ is finite) we start by showing that~$\coord(G)$ contains a \dash{finite}{dimensional}~\dash{$G$}{invariant} subspace.
  
  We will use the following observation:
  Dually to how the right multiplication multiplication~$r_g$ can be expressed via the multiplication~$m \colon G \times G \to G$ we can express the action~$\rho_g = r_g^*$ via~$m^*$ und thus via~$\Delta$.
  If~$\Delta(f) = \sum_{i=1}^n h_i \otimes a_i$ with~$f_i, a_i \in \coord(G)$ then it follows from~\eqref{explicit description of comultiplication} that
  \begin{equation}
  \label{expressing action via comultiplication}
      \rho_g(f)
    = r_g^*(f)
    = f \circ r_g
    = f((-) g)
    = \sum_{i=1}^n h_i(-) a_i(g)
    = \sum_{i=1}^n a_i(g) h_i
  \end{equation}
  is a linear combination of~$h_1, \dotsc, h_n$ with coefficients~$a_1(g), \dotsc, a_n(g)$.
\end{fluff}


\begin{lemma}
\label{invariant subspaces of coordinate ring}
  Let~$V \subseteq \coord(G)$ be a linear subspace.
  \begin{enumerate}
    \item
      The linear subspace~$V$ is~\dash{$G$}{invariant} if and only if~$\Delta(V) \subseteq V \tensor \coord(G)$.
    \item
      If~$V$ is \dash{finite}{dimensional} then there exists a \dash{finite}{dimensional} linear subspace~$W \subseteq \coord(G)$ with~$V \subseteq W$.
  \end{enumerate}
\end{lemma}


\begin{proof}
  \leavevmode
  \begin{enumerate}
    \item
      Let~$(h_i \suchthat i \in I)$ be a basis of~$V$ and extend this to a basis~$(h_j \suchthat j \in J)$ of~$\coord(G)$ with~$I \subseteq J$.
      For~$f \in \coord(G)$ the element~$\Delta(f)$ can be written as~$\Delta(f) = \sum_{j \in J} h_j \tensor a_{f,j}$ for unique elements~$a_{f,j} \in \coord(g)$, and it follows from~\eqref{expressing action via comultiplication} that
      \[
          \rho_g(f)
        = \sum_{j \in J} a_{f,j}(g) h_j \,.
      \]
      It follows that
      \begin{align*}
            {}& \text{$V$ is~\dash{$G$}{invariant}} \\
        \iff{}& \text{$\rho_g(f) \in V$ for all~$g \in G$,~$f \in V$} \\
        \iff{}& \text{$a_{f,j}(g) = 0$ for all~$j \in J \setminus I$,~$g \in G$,~$f \in V$} \\
        \iff{}& \text{$a_{f,j} = 0$ for all~$j \in J \setminus I$,~$f \in V$} \\
        \iff{}& \text{$\Delta(f) \in V \tensor \coord(G)$ for all~$f \in V$}  \\
        \iff{}& \Delta(V) \subseteq V \tensor \coord(G) \,,
      \end{align*}
      as desired.
    \item
      We may assume that~$V$ is \dash{one}{dimensional}.
      Let~$f \in V$ be nonzero.
      Then~$\Delta(f) = \sum_{i=1}^n f_i \tensor a_i$ for some~$f_i, a_i \in \coord(G)$ and it follows from~\eqref{expressing action via comultiplication} that
      \[
          \rho_g(f)
        = \sum_{i=1}^n a_i(g) f_i
      \]
      for every~$g \in G$.
      This shows for the \dash{finite}{dimensional} linear subspace~$W' \subseteq \coord(G)$ given by~$W' \defined \gen{f_1, \dotsc, f_n}_k$ that~$\rho_g(f) \in W'$ for every~$g \in G$.
      It follows for~$W \defined \gen{ \rho_g(f) \suchthat g \in G }_k$, which is the~\dash{$G$}{invariant} subspace generated by~$f$, from~$W \subseteq W'$ that~$W$ is also finite-dimensional.
    \qedhere
  \end{enumerate}
\end{proof}


\begin{theorem}[Embedding theorem]
  \label{embedding theorem}
  Every affine algebraic group~$G$ is isomorphic to a closed subgroup of some~$\GL_n(k)$.
\end{theorem}


\begin{proof}
  \label{embedding theorem proof}
  The coordinate ring~$\coord(G)$ is finitely generated as a~\dash{$k$}{algebra}.
  Let~$h_1, \dotsc, h_r \in \coord(G)$ be a set of~\dash{$k$}{algebra} generators and set~$V \defined \gen{h_1, \dotsc, h_r}_k$.
  It follows from \cref{invariant subspaces of coordinate ring} that~$V$ is contained in a \dash{finite}{dimensional}~\dash{$G$}{invariant} subspace~$W \subseteq \coord(G)$.
  We may assume that~$h_1, \dotsc, h_r$ are linearly independent and extend this family to a basis~$B = (h_1, \dotsc, h_n)$ of~$W$.
  For every~$g \in G$ let~$R(g) \in \GL_n(k)$ be the matrix representation of~$\rho_g$ with respect to the basis~$B$.
  In this way the representation~$\rho \colon G \to \GL(W)$ becomes a matrix representation (i.e.\ group homomorphism)~$R \colon G \to \GL_n(k)$.
  
  It follows by \cref{invariant subspaces of coordinate ring} from the~\dash{$G$}{invariance} of~$W$ that~$\Delta(W) \subseteq W \tensor \coord(G)$.
  We may therefore write for every~$j = 1, \dotsc, n$ the element~$\Delta(h_j)$ as
  \[
      \Delta(h_j)
    = \sum_{i=1}^n h_i \otimes a_{ij}
  \]
  for unique elements~$a_{ij} \in \coord(G)$.
  It follows from~\eqref{expressing action via comultiplication} that
  \[
      \rho_g(h_j)
    = \sum_{i=1}^n a_{ij}(g) h_i
  \]
  This show shows that the entries of~$R$ are given by the regular functions~$a_{ij}$, which shows that the group homomorphism~$R$ is regular and thus a morphism of affine algebraic groups.
  
  It remains to show that the image of~$R$ is closed in~$\GL_n(k)$ and that~$R$ restricts to an isomorphism of affine varieties~$G \to \im(R)$.
  According to \cref{induced is isomorphism or surjective} it sufficies to show that the induced algebra homomorphism~$f^*$ is surjective.
  It follows from~$f^*(x_{ij}) = a_{ij}$ that the coefficient functions~$a_{ij}$ are contained in the image of~$f^*$.
  It follows with
  \[
      h_j
    = h_j(1 \cdot (-))
    = \sum_{i=1}^n h_i(1) a_{ij}(-)
    = \sum_{i=1}^n h_i a_{ij}
  \]
  that~$h_1, \dotsc, h_n$ are contained in the image of~$f^*$.
  As the elements~$h_1, \dotsc, h_r$ generate~$\coord(G)$ as a~\dash{$k$}{algebra} it further follows that~$f^*$ is surjective.
\end{proof}


\begin{fluff}
  We have now seen that affine algebraic groups are (up to isomorphism) the same as closed subgroups of~$\GL_n(k)$.
  Affine algebraic groups are therefore also know as~\emph{linear algebraic groups}.
  We will therefore use this term (instead of \enquote{affine algebraic group}) throughout the rest of this notes.
\end{fluff}


\begin{remark}
  One may think about the \hyperref[embedding theorem]{embedding theorem} as anagalogous to
  \begin{itemize}
    \item
      Cayley’s theorem, which ensures that every finite group can be embedded into some~$S_n$;
    \item
      Whitney’s theorem, which ensures that every manifold~$M$ can be embedded into some~$\Real^n$;
    \item
      and Ado’s theorem, which ensures that every \dash{finite}{dimensional} Lie~algebra over a field~$k$ can be embedded into some~$\gl_n(k)$.
  \end{itemize}
\end{remark}


\begin{corollary}
  Let~$G$ be a linear algebraic group and let~$H \groupleq G$ be a closed subgroup.
  Then~$G$ is isomorphic to a closed subgroup~$G'$ of some~$\GL(W)$, for~$W$ a \dash{finite}{dimensional}-\dash{$k$}{vector space}, such that
  \[
      H
    = \Stab(W_H)
    = \{ g \in G \suchthat \rho_g(W_h) \subseteq W_H \} \,.
  \]
  for some~\dash{$k$}{linear subspace}~$W_H \subseteq W$.
\end{corollary}


\begin{proof}
  Let~$I \defined \videal_G(H)$.
  Let~$f_1, \dotsc, f_r$ be a linearly independent generating set for the ideal~$I$.
  We may extend this to a linearly independent algebra generating set~$f_1, \dotsc, f_s$ for~$\coord(G)$.
  By \cref{invariant subspaces of coordinate ring} there exists a \dash{finite}{dimensional}~\dash{$G$}{invariant} subspace~$W \subseteq \coord(G)$ which contains~$f_1, \dotsc, f_s$.
  We may extend~$f_1, \dotsc, f_s$ to a basis~$f_1, \dotsc, f_n$ of~$W$.
  We set~$W_H \defined W \cap I$.
  
  We find as in the  \hyperref[embedding theorem proof]{proof of the embedding theorem} that~$\rho \colon G \to \GL(W)$ restrict to an isomorphism~$G \to \im(\rho)$ with~$\im(\rho)$ being closed in~$\GL(W)$
  
  It follows from~$\vset(I) = \vset(\videal(H)) = H$ (because~$H$ is closed) for every~$x \in G$ that~$x \in H$ if and only if~$\varphi(x) = 0$ for every~$\varphi \in I$.
  It follows for every~$g \in G$ that
  \begingroup
  \allowdisplaybreaks
  \begin{align*}
        {}& g \in H \\
    \iff{}& \text{$gh \in H$ for every $h \in H$} \\
    \iff{}& \text{$r_g(h) \in H$ for every~$h \in H$} \\
    \iff{}& \text{$\varphi(r_g(h)) = 0$ for all~$\varphi \in I$,~$h \in H$} \\
    \iff{}& \text{$\rho_g(\varphi)(h) = 0$ for all~$\varphi \in I$,~$h \in H$}  \\
    \iff{}& \text{$\rho_g(\varphi) \in I$ for every~$\varphi \in I$}  \\
    \iff{}& \rho_g(I) \subseteq I \\
    \iff{}& \rho_g(W_H) \subseteq W_H
            \tag{\theequation} \addtocounter{equation}{1} \label{iff step} \\
    \iff{}& g \in \Stab(W_H) \,.
  \end{align*}
  \endgroup
  Note for the equivalence~\eqref{iff step} that if $\rho_g(I) \subseteq I$ then also
  \[
              \rho_g(W_H)
    =         \rho_g(W \cap I)
    \subseteq \rho_g(W) \cap \rho_g(I)
    \subseteq W \cap I
    =         W_H
  \]
  because~$W$ is~\dash{$G$}{invariant}.
  If on the other hand~$\rho_g(W_H) \subseteq W_H$ then the generators~$f_1, \dotsc, f_r$ of the ideal~$I$, which are contained in both~$W$ and~$I$ and therefore also in~$W_H$, are mapped into~$W_H \subseteq I$ by the algebra automorphism~$\rho_g$.
  It then follows that the whole of~$I$ is again mapped into~$I$ by~$\rho_g$.
\end{proof}




